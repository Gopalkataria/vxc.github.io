% Created 2019-02-18 Mon 15:16
% Intended LaTeX compiler: pdflatex
\documentclass[11pt]{article}
\usepackage[utf8]{inputenc}
\usepackage[T1]{fontenc}
\usepackage{graphicx}
\usepackage{grffile}
\usepackage{longtable}
\usepackage{wrapfig}
\usepackage{rotating}
\usepackage[normalem]{ulem}
\usepackage{amsmath}
\usepackage{textcomp}
\usepackage{amssymb}
\usepackage{capt-of}
\usepackage{hyperref}
\usepackage{listings}
\author{choppell}
\date{\today}
\title{References and Bibliography}
\hypersetup{
 pdfauthor={choppell},
 pdftitle={References and Bibliography},
 pdfkeywords={},
 pdfsubject={},
 pdfcreator={Emacs 25.3.2 (Org mode 9.1.9)}, 
 pdflang={English}}
\begin{document}

\maketitle
\tableofcontents


\section{Keywords}
\label{sec:orgea76e27}
\begin{description}
\item[{type}] conf-paper, conf-poster, journal-article, book,
book-chapter,

\item[{topic}] corp, dsl, edutech, concurrency, control, renarration,
semantic-web, sw-engg, systems, systems-engg,
web-accessibility, web-sec, vlabs

\item[{conf}] icalt, icse, isec, seke,
\end{description}

\section{2018}
\label{sec:orgfc35641}
\lstset{language=bibtex,label= ,caption= ,captionpos=b,numbers=none}
\begin{lstlisting}
@InProceedings{2018-seke-gollapudi-et-al,
  keywords = {unpublished, web-accessibility, renarration, semantic-web, seke},
  author = 		 {Sai VRJ Gollapudi and Venkatesh Choppella and
                  Sridhar Chimalakonda},
  title = 		 {Overlaying of Semantic Structures on Published Web
                  Pages (Poster)},
  OPTcrossref =  {},
  OPTkey = 		 {},
  booktitle = {Proc. International Conference on Software Engineering and
                  Knowledge Engineering.  San Francisco (SEKE 2018)},
  year = 	 {2018},
  OPTeditor = 	 {},
  OPTvolume = 	 {},
  OPTnumber = 	 {},
  OPTseries = 	 {},
  OPTpages = 	 {},
  month = 	 {July},
  OPTaddress = 	 {},
  OPTorganization = {},
  OPTpublisher = {},
  note = 	 {Accepted for publication.  Withdrawn on authors' request.},
  OPTannote = 	 {}
}


@InProceedings{2018-incose-natarajan-et-al,
  keywords = {conf-paper, systems, systems-engineering, sw-engg, seke},
  author = 		 {Swaminathan Natarajan and Kesav Vithal Nori and
                  Viswanath Kasturi and Anand Kumar and Venkatesh Choppella and Subhrojyoti Roy Chaudhuri},
  title = 		 {A Conceptual Model of Systems Engineering},
  OPTcrossref =  {},
  OPTkey = 		 {},
  booktitle = {28th Annual {INCOSE} International Symposium},
  year = 	 {2018},
  OPTeditor = 	 {},
  OPTvolume = 	 {},
  OPTnumber = 	 {},
  OPTseries = 	 {},
  OPTpages = 	 {},
  month = 	 {July},
  OPTaddress = 	 {},
  OPTorganization = {},
  OPTpublisher = {},
  note = 	 {Accepted for publication (poster)},
  OPTannote = 	 {}
}


@InProceedings{2018-icalt-swain-et-al,
  keywords={conf-paper, vlabs, edutech, icalt},
  author = 		 {Shovan Swain and Lalit Sanagavarapu and Venkatesh Choppella and Y. Raghu Reddy},
  title = 		 {Model Driven Approach for Virtual Lab Authoring - Chemical Sciences Labs},
  OPTcrossref =  {},
  OPTkey = 		 {},
  booktitle = {International Conference on Advanced Learning
                  Technologies (ICALT)},
  year = 	 {2018},
  OPTeditor = 	 {},
  OPTvolume = 	 {},
  OPTnumber = 	 {},
  OPTseries = 	 {},
  OPTpages = 	 {},
  OPTmonth = 	 jul,
  date = {2018-07},  
  OPTaddress = 	 {},
  OPTorganization = {},
  publisher = {IEEE},
  note = 	 {Accepted for publication},
  OPTannote = 	 {}
} 

@InProceedings{2018-icalt-kumar-et-al,
  keywords={conf-paper, vlabs, edutech, icalt},
  author = 		 {Mrityunjay Kumar and Jessica Emory and Venkatesh Choppella},
  title = 		 {Usability Analysis of Virtual Labs},
  OPTcrossref =  {},
  OPTkey = 		 {},
  booktitle = {International Conference on Advanced Learning
                  Technologies (ICALT)},
  year = 	 {2018},
  OPTeditor = 	 {},
  OPTvolume = 	 {},
  OPTnumber = 	 {},
  OPTseries = 	 {},
  OPTpages = 	 {},
  month = 	 jul,
  date = {2018-07},
  OPTaddress = 	 {},
  OPTorganization = {},
  publisher = {IEEE},
  note = 	 {Accepted for publication},
  OPTannote = 	 {}
} 

@Report{2018-arxiv-choppella-et-al,
  keywords =     {tech-report, concurrency, control, feedback, dining-philosophers, arxiv},
  author = 		 {Venkatesh Choppella and Kasturi Viswanath and Arjun Sanjeev},
  title = 		 {Generalized Dining Philosophers and Feedback Control},
  type = 		 {article},
  institution =  {{IIIT} {H}yderabad},
  year = 	 {2018},
  date = 	 {2018-05-05},
  OPTkey = 		 {},
  OPTsubtitle =  {},
  OPTtitleaddon = {},
  OPTlanguage =  {},
  number = 	 {ar{X}iv:1805.02010v1},
  OPTnote = 	 {},
  OPTlocation =  {},
  OPTmonth = 	 {},
  OPTisrn = 	 {},
  OPTchapter = 	 {},
  OPTpages = 	 {},
  OPTaddendum =  {},
  OPTpubstate =  {},
  OPTdoi = 		 {},
  OPTeprint = 	 {},
  OPTeprintclass = {},
  OPTeprinttype = {},
  url =   		 {https://arxiv.org/pdf/1805.02010.pdf},
  OPTurldate = 	 {},
  OPTannote = 	 {}
}

@InProceedings{2018-isec-prasad-et-al,
keywords = {Domain Specific Language (DSL), dsl, renarration,
                  web-accessibility, renarration, Web Page Transformation, isec},
  author = 		 {Prasad, Gollapudi VRJ Sai and Chimalakonda, Sridhar and Choppella, Venkatesh},
  title = 		 {Towards a Domain-Specific Language for the Renarration of Web Pages},
  year = 	 {2018},
  ALTdate = 	 {2018-02},
  OPTcrossref =  {},
  OPTkey = 		 {},
  booktitle = {Proceedings of the 11th Innovations in Software Engineering Conference},  
  OPTeditor = 	 {},
  OPTsubtitle =  {},
  OPTtitleaddon = {},
  OPTmaintitle = {},
  OPTmainsubtitle = {},
  OPTmaintitleaddon = {},
  OPTbooksubtitle = {},
  OPTbooktitleaddon = {},
  OPTeventtitle = {},
  OPTeventdate = {},
  OPTvenue = 	 {},
  OPTlanguage =  {},
  OPTvolume = 	 {},
  OPTpart = 	 {},
  OPTvolumes = 	 {},
  series = 	 {{ISEC '18}},
  OPTnumber = 	 {},
  OPTnote = 	 {},
  OPTorganization = {},
  publisher = {ACM},  
  location =  {Hyderabad, India},
  OPTmonth = 	 {feb},
  isbn = 	 {978-1-4503-6398-3},
  OPTchapter = 	 {},
  pages = 	 {3:1--3:10},
  OPTaddendum =  {},
  OPTpubstate =  {},
  doi = {10.1145/3172871.3172873},
  acmid = {3172873},  
  OPTeprint = 	 {},
  OPTeprintclass = {},
  OPTeprinttype = {},
  url = {http://doi.acm.org/10.1145/3172871.3172873},
  pdf={./papers/2018-isec-prasad-et-al.pdf},
  OPTurldate = 	 {},
  OPTannote = 	 {}

}

@InProceedings{2018-afmss-sanjeev-et-al,
  keywords={conf-paper, afmss, concurrency, control, isec},
  author = 		 {Arjun Sanjeev and Venkatesh Choppella and Viswanath Kasturi},
  title = 		 {Peterson's Mutual Exclusion Algorithm as Feedback Control},
  OPTcrossref =  {},
  OPTkey = 		 {},
  booktitle = {2nd Symposium on Application of Formal Methods for
               Safety and Security of Safety Critical Systems (AFMSS 2018)},
  year = 	 {2018},
  OPTeditor = 	 {},
  OPTvolume = 	 {},
  OPTnumber = 	 {},
  OPTseries = 	 {},
  OPTpages = 	 {},
  OPTmonth = 	 feb,
  OPTdate = {2018-02},
  OPTaddress = 	 {},
  OPTorganization = {},
  publisher = {Springer},
  note = 	 {\textbf{Best Paper Award: 2nd Prize}},
  OPTannote = 	 {},
  pdf = {./papers/2018-afmss-sanjeev-et-al.pdf}
}

@inproceedings{2018-isec-banerjee-et-al,
 author = {Banerjee, Amar and Choppella, Venkatesh and Kasturi, Viswanath and Natarajan, Swaminathan and Nistala, Padmalata V. and Nori, Kesav},
 title = {An Attempt at Explicating the Relationship Between Knowledge, Systems and Engineering},
 booktitle = {Proceedings of the 11th Innovations in Software Engineering Conference},
 series = {ISEC '18},
 month={February}, 
 year = {2018},
 isbn = {978-1-4503-6398-3},
 location = {Hyderabad, India},
 pages = {5:1--5:11},
 articleno = {5},
 numpages = {11},
 url = {http://doi.acm.org/10.1145/3172871.3172875},
 doi = {10.1145/3172871.3172875},
 acmid = {3172875},
 publisher = {ACM},
 address = {New York, NY, USA},
 pdf={./papers/2018-isec-prasad-et-al.pdf},
 keywords = {conceptual model of systems engineering, knowledge-centric explication of engineering, relationship between knowledge and systems, theory of software systems engineering, viewpoint mappings}
}


@InProceedings{2018-icdcit-Prasad-et-al,
  keywords= {conf-paper, icdcit, web-accessibility, renarration},
  author =  {Gollapudi VRJ Sai Prasad and Venkatesh Choppella and Sridhar Chimalakonda},
  title =  {A Style Sheets Based Approach for Semantic Transformation of Web Pages},
  OPTcrossref =  {},
  OPTkey = 		 {},
  booktitle = {Distributed Computing and Internet Technology 14th International Conference, {ICDCIT 2018}, Proceedings},
  year = 	 {2018},
  OPTeditor = 	 {},
  volume = 	 {10722},
  OPTnumber = 	 {},
  series = 	 {LNCS},
  pages = 	 {240--255},
  OPTmonth = 	 {January},
  date = {2018-01},
  address = 	 {Bhubaneshwar, India},
  OPTorganization = {},
  publisher = {Springer},
  OPTnote = 	 {},
  OPTannote = 	 {}
}
\end{lstlisting}
\section{2017}
\label{sec:orgd3a862a}
\lstset{language=bibtex,label= ,caption= ,captionpos=b,numbers=none}
\begin{lstlisting}
@Inbook{2017-ICISS-Agrawall-et-al,
keywords={conf-paper, web-sec, corp},
author={Agrawall, Akash and Maheshwari, Shubh and Bandyopadhyay,
                  Projit and Choppella, Venkatesh},
OPTeditor={Shyamasundar, Rudrapatna K. and Singh, Virendra and Vaidya, Jaideep},

title={Modelling and Mitigation of Cross-Origin Request Attacks on
                  Federated Identity Management Using Cross Origin
                  Request Policy},
				  
bookTitle={Information Systems Security: 13th International
                  Conference, ICISS 2017, Mumbai, India, December
                  16-20, 2017, Proceedings},
				  
year={2017},
OPTmonth={December},
publisher={Springer},
pages={263--282},
OPTabstract={Cross origin request attacks (CORA) such as Cross site
                  request forgery (CSRF), cross site timing,
                  etc. continue to pose a threat on the modern day
                  web. Current browser security policies inadequately
                  mitigate these attacks. Additionally, third party
                  authentication services are now the preferred way to
                  carry out identity management between multiple
                  enterprises and web applications. This scenario,
                  called Federated Identity Management (FIM) separates
                  the problem of identity management from the core
                  functionality of an application.},
isbn={978-3-319-72598-7},
doi={10.1007/978-3-319-72598-7_16},
url={https://doi.org/10.1007/978-3-319-72598-7_16},
pdf={./papers/todo.pdf}
} 

@InProceedings{2017-iske-gollapudi-et-al,
  keywords={conf-paper, iske, web-accessibility, renarration},
  author = 		 {Sai VRJ Gollapudi and Soumya M. Saraswathi and Venkatesh
                  Choppella},
  title = 		 {Renarrating Web Pages for Improving Information Accessibility},
  OPTcrossref =  {},
  OPTkey = 		 {},
  booktitle = {Proceedings of the 12th {I}nternational
                  {C}onference on {I}ntelligent {S}ystems and
                  {K}nowledge {E}ngineering},
  year = 	 {2017},
  OPTeditor = 	 {},
  OPTvolume = 	 {},
  OPTnumber = 	 {},
  series = 	 {ISKE 2017},
  pages = 	 {1--7},
  month = 	 {November},
  date = {2017-11},
  OPTaddress = 	 {},
  OPTorganization = {},
  publisher = {IEEE},
  OPTnote = 	 {},
  pdf = {./papers/2017-iske.pdf},		  
  OPTannote = 	 {}
} 


@inproceedings{DBLP-conf-seke-SanagavarapuGCR17,
  keywords={conf-paper, semantic-web, ontologies, sw-engg, knowledge-engg},
  author    = {Lalit Sanagavarapu and
               Sai Gollapudi and
               Sridhar Chimalakonda and
               Y. Raghu Reddy and
               Venkatesh Choppella},
  title     = {A Lightweight Approach for Evaluating Sufficiency of Ontologies},
  booktitle = {The 29th International Conference on Software Engineering and Knowledge
               Engineering, Wyndham Pittsburgh University Center, Pittsburgh, PA,
               USA},
  pages     = {557--561},
  month     = {July},
  year      = {2017},
  OPTcrossref  = {DBLP:conf/seke/2017},
  url       = {https://doi.org/10.18293/SEKE2017-185},
  doi       = {10.18293/SEKE2017-185},
  timestamp = {Tue, 22 Aug 2017 14:02:44 +0200},
  biburl    = {http://dblp.uni-trier.de/rec/bib/conf/seke/SanagavarapuGCR17},
  bibsource = {dblp computer science bibliography, http://dblp.org},
  pdf = {./papers/2017-seke.pdf},
} 


@inproceedings{Prasad-et-al-isec-2017,
 keywords={conf-paper, aop, web-accessibility, renarration, isec},
 author = {Prasad, Gollapudi VRJ Sai and Chimalakonda, Sridhar and Choppella, Venkatesh and Reddy, Y. Raghu},
 title = {An Aspect Oriented Approach for Renarrating Web Content},
 booktitle = {Proceedings of the 10th Innovations in Software Engineering Conference},
 series = {ISEC '17},
 year = {2017},
 OPTmonth={February},
 isbn = {978-1-4503-4856-0},
 location = {Jaipur, India},
 pages = {56--65},
 numpages = {10},
 url = {http://doi.acm.org/10.1145/3021460.3021466},
 doi = {10.1145/3021460.3021466},
 acmid = {3021466},
 publisher = {ACM},
 address = {New York, NY, USA},
 keywords = {Annotation, Aspects for documents, Microservices for Annotation, Re-narration, Structured Web Documents},
}


@InProceedings{agrawall-et-al-isec-2017,
 keywords={conf-paper, web-sec, corp},
 author = {Agrawall, Akash and Telikicherla,  Krishna Chaitanya  and Agrawal, Arnav Kumar and Choppella, Venkatesh},
 title = {Mitigating Browser-based {DDoS} Attacks Using {CORP}},
 booktitle = {Proceedings of the 10th Innovations in Software Engineering Conference},
 series = {ISEC '17},
 year = {2017},
 isbn = {978-1-4503-4856-0},
 location = {Jaipur, India},
 pages = {137--146},
 numpages = {10},
 url = {http://doi.acm.org/10.1145/3021460.3021477},
 doi = {10.1145/3021460.3021477},
 acmid = {3021477},
 publisher = {ACM},
 address = {New York, NY, USA},
 keywords = {Browser, Browser-based DDoS, Cross-origin requests, DDoS, Javascript, MITM (Man in the middle)},
}

@InProceedings{telikicherla-et-al-icissp-2017,
  keywords={conf-paper, web-sec, corp},
  author = 		 {Krishna Chaitanya Telikecherla and Akash Agrawall and Venkatesh Choppella},
  title = 		 {A Formal Model of web security showing malicious Cross Origin Requests and its mitigation using {CORP}},
  OPTcrossref =  {},
  OPTkey = 		 {},
  booktitle = {Proc. 3rd International Conference on Information on
                  Systems, Security and Privacy, {ICISSP} 2017. Porto,
                  Portugal},
  year = 	 {2017},
  OPTeditor = 	 {},
  OPTvolume = 	 {},
  OPTnumber = 	 {},
  OPTseries = 	 {},
  OPTpages = 	 {},
  month = 	 {February},
  OPTaddress = 	 {},
  OPTorganization = {},
  publisher = {Scitepress},
  OPTnote = 	 {},
  OPTannote = 	 {}
}


@InProceedings{sangavarapu-et-al-2017,
 keywords={conf-paper, t4e, edutech, vlabs, isec},
 author = {S., Lalit Mohan and Raman, Priya and Choppella, Venkatesh and Reddy, Y. R.},
 title = {A Crowdsourcing Approach for Quality Enhancement of eLearning Systems},
 booktitle = {Proceedings of the 10th Innovations in Software Engineering Conference},
 series = {ISEC '17},
 year = {2017},
 isbn = {978-1-4503-4856-0},
 location = {Jaipur, India},
 pages = {188--194},
 numpages = {7},
 url = {http://doi.acm.org/10.1145/3021460.3021483},
 doi = {10.1145/3021460.3021483},
 acmid = {3021483},
 publisher = {ACM},
 address = {New York, NY, USA},
 keywords = {Crowdsourcing, Quality, Software Development, eLearning Systems},
}

\end{lstlisting}
\section{2016}
\label{sec:orgfc6e4a1}
\lstset{language=bibtex,label= ,caption= ,captionpos=b,numbers=none}
\begin{lstlisting}
@InProceedings{bradford-choppella-uemcon-2016,
  keywords={conf-paper, theory, dyck-sets, uemcon},
  author = 		 {Phillip G. Bradford and Venkatesh  Choppella},
  title = 		 {Fast Dyck Constrained Shortest Paths},
  OPTcrossref =  {},
  OPTkey = 		 {},
  booktitle = {Proc. 7th IEEE Annual Ubiquitous Computing,
                  Electronics \& Mobile Communication Conference
                  (UEMCON 2016)},
  year = 	 {2016},
  OPTeditor = 	 {},
  OPTvolume = 	 {},
  OPTnumber = 	 {},
  OPTseries = 	 {},
  OPTpages = 	 {},
  month = 	 {October},
  OPTaddress = 	 {},
  OPTorganization = {},
  publisher = {IEEE Explore},
  OPTnote = 	 {},
  OPTannote = 	 {}
}

@INPROCEEDINGS{Singla-et-al-2016, 
keywords={conf-paper, model-checking, formal-methods,cloud computing;concurrency (computers);embedded systems;formal verification;large-scale systems;safety;cloud service providers;concurrent nature;distributed safety property verification algorithm;embedded systems;large scale distributed systems;vertex centric programming;Computational modeling;Java;Model checking;Object oriented modeling;Programming;Radiation detectors;Safety},
author={A. Singla and K. Desai and S. Purini and V. Choppella}, 
booktitle={2016 15th International Symposium on Parallel and Distributed Computing (ISPDC)}, 
title={Distributed Safety Verification Using Vertex Centric Programming Model}, 
year={2016}, 
volume={}, 
number={}, 
pages={114-120}, 
doi={10.1109/ISPDC.2016.23}, 
ISSN={}, 
month={July},
pdf={./papers/2016-ispdc-singla-et-al.pdf}
}

@INPROCEEDINGS{Choppella-et-al-2016,
keywords={conf-paper, cs-ed, vlabs, popl, computer science education;educational courses;programming;programming languages;abstract syntax trees annotation;functional program;imperative program;principles of programming languages course, program rewriting, visual model, Computer languages, Concrete, Education, Programming, Semantics, Syntactics, Visualization},
author={V. Choppella and G. Ahuja and A. Mavalankar}, 
booktitle={2016 International Conference on Learning and Teaching in Computing and Engineering (LaTICE)}, 
title={How Does a Program Run? A Visual Model Based on Annotating Abstract Syntax Trees}, 
year={2016}, 
volume={}, 
number={}, 
pages={38-42}, 
doi={10.1109/LaTiCE.2016.40}, 
ISSN={}, 
month={March},}
\end{lstlisting}
\section{2015}
\label{sec:org8e27bcc}
\lstset{language=bibtex,label= ,caption= ,captionpos=b,numbers=none}
\begin{lstlisting}
@InProceedings{mavalankar-et-al-t4e-2015,
  keywords={conf-paper, t4e, edutech, vlabs, semantic-web},
  author = 		 {Aditi Mavalankar and Tejaswinee Kelkar and Venkatesh
                  Choppella},
  title = 		 {Generation of Quizzes and Solutions Based on Ontologies -- A Case for a Music Problem Generator},
  OPTcrossref =  {},
  OPTkey = 		 {},
  booktitle = {2015 IEEE Seventh International Conference on Technology for Education (T4E)},
  year = 	 {2015},
  OPTeditor = 	 {},
  OPTvolume = 	 {},
  OPTnumber = 	 {},
  OPTseries = 	 {},
  pages = 	 {73--76},
  month = 	 {December},
  OPTaddress = 	 {},
  OPTorganization = {},
  publisher = {IEEE},
  OPTnote = 	 {},
  OPTannote = 	 {},
  DOI = {10.1109/T4E.2015.16},
  pdf = {./papers/2015-t4e.pdf}
}


@InProceedings{Kelkar-et-al-icalt-2015,
  keywords={conf-paper, edutech, vlabs, icalt},
  author = 	 {Tejaswinee Kelkar and Anon Ray and Venkatesh Choppella},
  title = 	 {SangeetKosh: An Open Web Platform for
                  Hindustani Music Education},
  OPTcrossref =  {},
  OPTkey = 	 {},
  booktitle = {Proceedings of the 15 IEEE Conference on
                  Advances in Learning Technologies},
  year = 	 {2015},
  OPTeditor = 	 {},
  OPTvolume = 	 {},
  OPTnumber = 	 {},
  series = 	 {ICALT 2015},
  pages = 	 {5--9},
  month = 	 {July},
  location = 	 {Hualien, Taiwan},
  OPTorganization = {},
  publisher = {IEEE},
  note = 	 {},
  OPTannote = 	 {}
}


@InProceedings{Ahuja-et-al-icalt-2015,
  keywords={conf-paper, edutech, icalt, vlabs},
  author = 	 {Garima Ahuja and Anubha Gupta and Harsh Wardhan and Venkatesh Choppella},
  title = 	 {Assessing the impact of Virtual Labs: a case study with the lab on Advanced VLSI},
  OPTcrossref =  {},
  OPTkey = 	 {},
  booktitle = {Proceedings of the 15 IEEE Conference on
                  Advances in Learning Technologies},
  year = 	 {2015},
  OPTeditor = 	 {},
  OPTvolume = 	 {},
  OPTnumber = 	 {},
  series = 	 {ICALT 2015},
  pages = 	 {290--292},
  month = 	 {July},
  OPTaddress = 	 {},
  OPTorganization = {},
  publisher = {IEEE},
  note = 	 {},
  OPTannote = 	 {},
  doi = {10.1109/ICALT.2015.41},
  pdf ={./papers/2015-icalt-garima.pdf},
} 

@InProceedings{Choppella-Pulijala-ModSym-2015,
  keywords = {conf-poster, visualization, modeling, javascript, programming-languages},
  author = 	 {Venkatesh Choppella and Amulya Pulijala},
  title = 	 {Visual Modeling of Javascript},
  OPTcrossref =  {},
  OPTkey = 	 {},
  booktitle = {Modeling Symposium},
  year = 	 {2015},
  OPTeditor = 	 {},
  OPTvolume = 	 {},
  OPTnumber = 	 {},
  series = 	 {8th India Software Engineering Conference},
  OPTpages = 	 {},
  month = 	 {February},
  location = 	 {Bengaluru, India},
  OPTorganization = {},
  OPTpublisher = {},
  note = 	 {Poster},
  OPTannote = 	 {}
}

\end{lstlisting}
\section{2014}
\label{sec:orga9f42ad}
\lstset{language=bibtex,label= ,caption= ,captionpos=b,numbers=none}
\begin{lstlisting}
@incollection{Telikicherla-Choppella-Bezawada-ICISS-2014,
keywords={conf-paper, web-sec, corp},
year={2014},
isbn={978-3-319-13840-4},
booktitle={Information Systems Security},
volume={8880},
series={Lecture Notes in Computer Science},
editor={Prakash, Atul and Shyamasundar, Rudrapatna},
doi={10.1007/978-3-319-13841-1_16},
title={CORP: A Browser Policy to Mitigate Web Infiltration Attacks},
url={http://dx.doi.org/10.1007/978-3-319-13841-1_16},
publisher={Springer International Publishing},
keywords={Web Browser; Security; World Wide Web; Cross-site request forgery; Access control policy},
author={Telikicherla, KrishnaChaitanya and Choppella, Venkatesh and Bezawada, Bruhadeshwar},
pages={277-297},
language={English}
}

@InProceedings{Telikicherla-2014-EDS-2593761-2593764,
 keywords = {conf-paper, web-sec, corp, Browsers, Mashups, Open APIs, Security, Web},
 author = {Telikicherla, Krishna Chaitanya and Choppella, Venkatesh},
 title = {Enabling the Development of Safer Mashups for Open Data},
 booktitle = {Proceedings of the 1st International Workshop on Inclusive Web Programming - Programming on the Web with Open Data for Societal Applications},
 series = {IWP 2014},
 year = {2014},
 isbn = {978-1-4503-2855-5},
 location = {Hyderabad, India},
 pages = {8--15},
 numpages = {8},
 url = {http://doi.acm.org/10.1145/2593761.2593764},
 doi = {10.1145/2593761.2593764},
 acmid = {2593764},
 publisher = {ACM},
 address = {New York, NY, USA},
}

@InProceedings{zade-et-al-chi-2014,
  keywords = {conf-paper, hci, formal-methods, chi},
  author = 	 {Himanshu Zade and Santosh Adimoolam and Sai Gollapudi and Anind Dey and Venkatesh Choppella},
  title = 	 {Edit Distance modulo Bisimulation: A Quantitative Measure to Study Evolution of User Models},
  OPTcrossref =  {},
  OPTkey = 	 {},
  booktitle = {Proceedings of the 32nd Annual ACM Conference on Human Factors in Computing Systems},
  year = 	 {2014},
  OPTeditor = 	 {},
  OPTvolume = 	 {},
  OPTnumber = 	 {},
  series = 	 {CHI'14},
  pages =        {1757--1766},
  url = {http://doi.acm.org/10.1145/2556288.2557191},
  month = 	 {April},
  location = 	 {Toronto, Canada},
  OPTorganization = {},
  publisher = {ACM},
  note = 	 {},
  OPTannote = 	 {}
}


@inproceedings{Prasad-2014-ONA-2596695-2596711,
 keywords = {conf-paper, semantic-web, web-accessibility, renarration, annotation, collaborative narration, metadata, social semantic web, web accessibility, web inclusion},
 author = {Prasad, Gollapudi {VRJ} Sai and Dinesh, T. B. and Choppella, Venkatesh},
 title = {Overcoming the New Accessibility Challenges Using the Sweet Framework},
 booktitle = {Proceedings of the 11th Web for All Conference},
 series = {W4A '14},
 year = {2014},
 isbn = {978-1-4503-2651-3},
 location = {Seoul, Korea},
 pages = {22:1--22:4},
 articleno = {22},
 numpages = {4},
 url = {http://doi.acm.org/10.1145/2596695.2596711},
 doi = {10.1145/2596695.2596711},
 acmid = {2596711},
 publisher = {ACM},
 address = {New York, NY, USA},
}


@InProceedings{2014-t4e-agarwal-et-al,
  keywords={conf-paper, t4e, edutech, vlabs},
  author = 	 {Jatin Agarwal and  Utkarsh Rastogi and Prateek Pandey and Nurendra  Choudhary and Venkatesh Choppella and Raghu Reddy},
  title = 	 {Large Scale Web Page Optimization of Virtual Labs},
  OPTcrossref =  {},
  OPTkey = 	 {},
  booktitle = {Proceedings of the IEEE International Conference on Technology for Education (T4E2014)},
  year = 	 {2014},
  OPTeditor = 	 {},
  OPTvolume = 	 {},
  OPTnumber = 	 {},
  OPTseries = 	 {},
  pages = 	 {146--147},
  OPTmonth = 	 dec,
  date = {2014-12},
  location = 	 {Kollam, India},
  OPTorganization = {},
  publisher = {IEEE},
  OPTnote = 	 {},
  OPTannote = 	 {}
}

@InProceedings{2014-t4e-choudhary-et-al,
  keywords={conf-paper, t4e, edutech, vlabs},
  author = 	 {Nurendra Choudhary and Venkatesh Choppella and Raghu Reddy and Thirumal Ravula},
  title = 	 {Large Scale Web Page Optimization of Virtual Labs},
  OPTcrossref =  {},
  OPTkey = 	 {},
  booktitle = {Proceedings of the IEEE International Conference on Technology for Education (T4E2014)},
  year = 	 {2014},
  OPTeditor = 	 {},
  OPTvolume = 	 {},
  OPTnumber = 	 {},
  OPTseries = 	 {},
  pages = 	 {29--31},
  month = 	 dec,
  date = {2014-12},
  location = 	 {Kollam, India},
  OPTorganization = {},
  publisher = {IEEE},
  note = 	 {},
  OPTannote = 	 {}
}
\end{lstlisting}
\section{2013}
\label{sec:org24be916}
\lstset{language=bibtex,label= ,caption= ,captionpos=b,numbers=none}
\begin{lstlisting}
@InProceedings{Manjula-et-al-t4e-2013,
  keywords = {conf-paper, cs-ed, algorithms, systems, education, t4e},
  author = 	 {Venkatesh Choppella and K Viswanath and P Manjula},
  title = 	 {Viewing algorithms as iterative systems and plotting their dynamic behaviour},
  OPTcrossref =  {},
  OPTkey = 	 {},
  booktitle = {Proceedings of the IEEE International Conference on Technology for Education (T4E2013)},
  year = 	 {2013},
  OPTeditor = 	 {},
  OPTvolume = 	 {},
  OPTnumber = 	 {},
  series = 	 {T4E 2013},
  pages = 	 {206--209},
  OPTmonth = 	 {},
  OPTaddress = 	 {},
  OPTorganization = {},
  publisher = {IEEE Press},
  OPTnote = 	 {Accepted for Publication},
  OPTannote = 	 {},
  pdf = {./papers/2013-t4e-choppell-et-al.pdf},
}

@InProceedings{Gollapudi-Choppella-t4e-2013,
  keywords = {conf-paper, edutech, research-methods, t4e},
  author = 	 {Sai Gollapudi  and Venkatesh Choppella},
  title = 	 {Descriptive Study of College Bound Rural Youth of AP, India},
  OPTcrossref =  {},
  OPTkey = 	 {},
  booktitle = {Proceedings of the IEEE International Conference on Technology for Education (T4E2013)},
  year = 	 {2013},
  OPTeditor = 	 {},
  OPTvolume = 	 {},
  OPTnumber = 	 {},
  OPTseries = 	 {},
  pages = 	 {76--79},
  month = 	 {December},
  OPTaddress = 	 {},
  OPTorganization = {},
  publisher = {IEEE},
  note = 	 {},
  OPTannote = 	 {},
  pdf = {./papers/2013-t4e-sai.pdf}
}


@inproceedings{Goel-Choppella-icegov-2013,
 keywords = {conf-paper, e-gov, formal-methods, icegov, access control, open government, workflows},
 author = {Goel, Ankur and Choppella, Venkatesh},
 title = {State Based Access Control for Open e-Governance},
 booktitle = {Proceedings of the 7th International Conference on Theory and Practice of Electronic Governance},
 series = {ICEGOV '13},
 year = {2013},
 isbn = {978-1-4503-2456-4},
 location = {Seoul, Republic of Korea},
 pages = {19--27},
 numpages = {9},
 url = {http://doi.acm.org/10.1145/2591888.2591892},
 doi = {10.1145/2591888.2591892},
 acmid = {2591892},
 publisher = {ACM},
 address = {New York, NY, USA},
} 



@TechReport{Adimoolam-Choppella-2013,
  keywords =   {tech-report, formal-methods, automata},
  author = 	 {Santosh Arvind Adimoolam and Venkatesh Choppella and PVR Murthy},
  title = 	 {Verifying Timed {CTL} contracts for continuous pure signal {I/O} automata by encoding as virtual environments},
  institution =  {International Institute of Information Technology Hyderabad},
  year = 	 {2013},
  OPTkey = 	 {},
  OPTtype = 	 {},
  number = 	 {IIIT/TR/2013/26},
  OPTaddress = 	 {},
  OPTmonth = 	 {},
  OPTnote = 	 {},
  OPTannote = 	 {}
}

@InProceedings{Chatterjee-et-al-t4e-2013,
  keywords = {conf-paper, edutech, vlabs, t4e},
  author = 	 {Sourav Chatterjee and Pranitha Reddy and Venkatesh Choppella},
  title = 	 {Automated Restructuring of Contents for Virtual Labs},
  OPTcrossref =  {},
  OPTkey = 	 {},
  booktitle = {Proceedings of the IEEE International Conference on Technology for Education (T4E2013)},
  year = 	 {2013},
  OPTeditor = 	 {},
  OPTvolume = 	 {},
  OPTnumber = 	 {},
  OPTseries = 	 {},
  OPTpages = 	 {},
  OPTmonth = 	 {},
  OPTaddress = 	 {},
  OPTorganization = {},
  publisher = {IEEE Press},
  OPTnote = 	 {Accepted for Publication},
  OPTannote = 	 {}
}


@TechReport{Alloy-Corp-Tech-2013,
keywords = {tech-report, web-sec, corp, Browsers, Mashups, Open APIs, Security, Web},
author = {Krishna Chaitanya Telikicherla and Venkatesh Choppella},
title = {{A}lloy model for {C}ross {O}rigin {R}equest {P}olicy ({CORP})},
number =  {IIIT/TR/2013/31},
institution = {IIIT-Hyderabad},
year = {2013},
month = {August},
note = {\url{http://web2py.iiit.ac.in/research_centres/publications/view_publication/techreport/112}},
}



\end{lstlisting}

\section{2012}
\label{sec:orga5d6eb4}
\lstset{language=bibtex,label= ,caption= ,captionpos=b,numbers=none}
\begin{lstlisting}
@InProceedings{malini-et-al-t4e-2012,
  keywords = {conf-paper, edutech, vlabs, t4e},
  author = 	 {Malani, S. and Prasanna, G.N.S. and del Alamo, J.A. and  Hardison, J.L. and Moudgalya, K. and Chopella, V.},
  title = 	 {Issues Faced in a Remote Instrumentation Laboratory},
  OPTcrossref =  {},
  OPTkey = 	 {},
  booktitle = {IEEE 4th International Conference on Technology for Education},
  pages = 	 {67-74},
  year = 	 {2012},
  OPTeditor = 	 {},
  OPTvolume = 	 {},
  OPTnumber = 	 {},
  OPTseries = 	 {},
  OPTaddress = 	 {},
  OPTmonth = 	 {},
  OPTorganization = {},
  publisher = {IEEE Press},
  OPTnote = 	 {},
  OPTannote = 	 {}
}



@InProceedings{khare-et-al-t4e-2012,
  keywords = {conf-paper, edutech, literate-programming, software-technology, t4e},
  author = 	 {Sankalp Khare and Ishan Misra and Venkatesh Choppella},
  title = 	 {Using Org-mode and Subversion for Managing and Publishing Content in Computer Science Courses},
  OPTcrossref =  {},
  OPTkey = 	 {},
  booktitle = {IEEE 4th International Conference on Technology for Education},
  pages = 	 {220-223},
  year = 	 {2012},
  OPTeditor = 	 {},
  OPTvolume = 	 {},
  OPTnumber = 	 {},
  OPTseries = 	 {},
  OPTaddress = 	 {},
  OPTmonth = 	 {},
  OPTorganization = {},
  publisher = {IEEE Press},
  OPTnote = 	 {},
  OPTannote = 	 {}
}


@InProceedings{goel-choppella-t4e-2012,
  keywords = {conf-paper, edutech, formal-methods,  modelling, education, workflows, t4e},
  author = 	 {Ankur Goel and Venkatesh Choppella},
  title = 	 {Algebraic Modelling of Educational Workflows},
  OPTcrossref =  {},
  OPTkey = 	 {},
  booktitle = {IEEE 4th International Conference on Technology for Education},
  pages = 	 {153-156},
  year = 	 {2012},
  OPTeditor = 	 {},
  OPTvolume = 	 {},
  OPTnumber = 	 {},
  OPTseries = 	 {},
  OPTaddress = 	 {},
  OPTmonth = 	 {},
  OPTorganization = {},
  publisher = {IEEE},
  OPTnote = 	 {},
  OPTannote = 	 {},
  pdf = {./papers/2012-t4e-ankur.pdf}
}

@InProceedings{choppella-et-al-t4e-2012,
  keywords = {conf-paper, cs-ed, algorithms, systems, programming, education, t4e},
  author = 	 {Venkatesh Choppella and  Hitesh Kumar and  P. Manjula and  K. Viswanath},
  title = 	 {From High-School Algebra to Computing through Functional Programming},
  OPTcrossref =  {},
  OPTkey = 	 {},
  booktitle = {IEEE 4th International Conference on Technology for Education},
  pages = 	 {180-183},
  year = 	 {2012},
  OPTeditor = 	 {},
  OPTvolume = 	 {},
  OPTnumber = 	 {},
  OPTseries = 	 {},
  OPTaddress = 	 {},
  OPTmonth = 	 {},
  OPTorganization = {},
  publisher = {IEEE Press},
  OPTnote = 	 {},
  OPTannote = 	 {}
}

@inproceedings{Dinesh-2012-AFR-2207016-2207030,
 keywords = {conf-paper, semantic-web, web-accessibility, renarration, annotation, collaborative narration, metadata, social semantic web, web accessibility, web inclusion},
 author = {Dinesh, T. B. and Uskudarli, S. and Sastry, Subramanya and Aggarwal, Deepti and Choppella, Venkatesh},
 title = {Alipi: A Framework for Re-narrating Web Pages},
 booktitle = {Proceedings of the International Cross-Disciplinary Conference on Web Accessibility},
 series = {W4A '12},
 year = {2012},
 isbn = {978-1-4503-1019-2},
 location = {Lyon, France},
 pages = {22:1--22:4},
 articleno = {22},
 numpages = {4},
 url = {http://doi.acm.org/10.1145/2207016.2207030},
 doi = {10.1145/2207016.2207030},
 acmid = {2207030},
 publisher = {ACM},
 address = {New York, NY, USA},
}


@inproceedings{Dinesh-2012-ATR-2207016-2207038,
 keywords = {conf-paper, semantic-web, web-accessibility, renarration, annotation, collaborative narration, metadata, social semantic web, web accessibility, web inclusion},
 author = {Dinesh, T. B. and Choppella, Venkatesh},
 title = {Alipi: Tools for a Re-narration Web},
 booktitle = {Proceedings of the International Cross-Disciplinary Conference on Web Accessibility},
 series = {W4A '12},
 year = {2012},
 isbn = {978-1-4503-1019-2},
 location = {Lyon, France},
 pages = {29:1--29:2},
 articleno = {29},
 numpages = {2},
 url = {http://doi.acm.org/10.1145/2207016.2207038},
 doi = {10.1145/2207016.2207038},
 acmid = {2207038},
 publisher = {ACM},
 address = {New York, NY, USA},
 note={\textbf{Microsoft Accessibility Challenge: Delegates award}}
}

@InProceedings{zade-choppella-ihci-2012,
  keywords = {conf-paper, hci, formal-methods, human computer interfaces, ihci},
  author = 	 {Himanshu Zade and Venkatesh Choppella},
  title = 	 {Functionality or User Interface: which is easier to learn when changed?},
  OPTcrossref =  {},
  OPTkey = 	 {},
  booktitle = {IEEE 4th International Conference on Intelligent Human
                  Computer Interaction (IHCI)},
  pages = 	 {1--6},
  year = 	 {2012},
  OPTeditor = 	 {},
  OPTvolume = 	 {},
  OPTnumber = 	 {},
  OPTseries = 	 {},
  OPTaddress = 	 {},
  month = 	 {December},
  OPTorganization = {},
  publisher = {IEEE},
  OPTnote = 	 {},
  OPTannote = 	 {},
  pdf = {./papers/2012-ihci.pdf},
}

@inproceedings{Aggarwal-2012,
keywords = {conf-paper, crowd-sourcing, hci, semantic-web, ihci},
 author = {Deepti Aggarwal and Rohit Ashok Khot  and  Vasudeva Varma  and
Venkatesh Choppella},
 title = {uPick: Crowdsourcing Based Approach to Extract Relations
among Named Entites},
 booktitle = {Proceedings of the 2012 international conference on
Human Computer Interaction},
 series = {IndiaHCI'12},
 year = {2012},
 location = {Pune, India},
 pages = {1-8},
}

\end{lstlisting}
\section{2011}
\label{sec:org7a691bf}
\lstset{language=bibtex,label= ,caption= ,captionpos=b,numbers=none}
\begin{lstlisting}
@InProceedings{choppella-et-al-t4e-2011,
  keywords = {conf-paper, edutech, vlabs, t4e},
  author = 	 {Venkatesh Choppella. and  V K Brahmajosyula  and  M. Vutpala. and S. Kole},
  title = 	 {Process Models for Virtual Lab Development, Deployment and Distribution},
  OPTcrossref =  {},
  OPTkey = 	 {},
  booktitle = {IEEE 3rd International Conference on Technology for Education},
  pages = 	 {293-294},
  year = 	 {2011},
  OPTeditor = 	 {},
  OPTvolume = 	 {},
  OPTnumber = 	 {},
  OPTseries = 	 {},
  OPTaddress = 	 {},
  OPTmonth = 	 {},
  OPTorganization = {},
  publisher = {IEEE Press},
  OPTnote = 	 {},
  OPTannote = 	 {}
}

@InProceedings{khot-choppella-t4e-2011,
  keywords = {conf-paper, edutech, vlabs, t4e},
  author = 	 {Rohit Khot and Venkatesh Choppella},
  title = 	 {DISCOVIR: A Framework for Designing Interfaces and Structuring Content for Virtual Labs},
  OPTcrossref =  {},
  OPTkey = 	 {},
  booktitle = {IEEE 3rd International Conference on Technology for Education},
  pages = 	 {121-127},
  year = 	 {2011},
  OPTeditor = 	 {},
  OPTvolume = 	 {},
  OPTnumber = 	 {},
  OPTseries = 	 {},
  OPTaddress = 	 {},
  OPTmonth = 	 {},
  OPTorganization = {},
  publisher = {IEEE Press},
  OPTnote = 	 {},
  OPTannote = 	 {}
}

@InProceedings{brahmajosyula-choppella-wambse-2011,
  keywords = {conf-paper, modeling, formal-methods, software-engineering},
  author = 	 {Vamsikrishna Brahmajosyula and Venkatesh Choppella},
  title = 	 {Modeling and Programming with State Variables},
  OPTcrossref =  {},
  OPTkey = 		 {},
  booktitle = {2nd Workshop on Advances in  Model-based Software Engineering},
  year = 	 {2011},
  OPTeditor = 	 {},
  OPTvolume = 	 {},
  OPTnumber = 	 {},
  OPTseries = 	 {},
  OPTpages = 	 {},
  OPTmonth = 	 {},
  OPTaddress = 	 {},
  OPTorganization = {},
  OPTpublisher = {},
  note = 	 {Colocated with 4th ISEC 2011, Trivandrum India},
  OPTannote = 	 {}
}

@Article{brahmajosyula-choppella-2011,
  keywords = {journal-article, modeling, formal-methods, software-engineering},
  author = 	 {Vamsikrishna Brahmajosyula and Venkatesh Choppella},
  title = 	 {Modeling and Programming with State Variables},
  journal = 	 {{SETLAB} Briefings},
  year = 	 {2011},
  OPTkey = 	 {},
  volume = 	 {9},
  number = 	 {4},
  pages = 	 {3-10},
  OPTmonth = {},
  note     = {Expanded version of \cite{brahmajosyula-choppella-wambse-2011}},
  OPTannote = 	 {}
}

@InProceedings{bandi-et-al-t4e-2011,
  keywords={conf-paper, vlabs, edutech, icalt},
  author = 	 {Bandi, K.C. and  Nori, A.K. and  Choppella, V. and  Kode, S.},
  title = 	 {A Virtual Laboratory for Teaching Linux on the Web},
  OPTcrossref =  {},
  OPTkey = 	 {},
  booktitle = {IEEE 3rd International Conference on Technology for Education},
  pages = 	 {212-215},
  year = 	 {2011},
  OPTeditor = 	 {},
  OPTvolume = 	 {},
  OPTnumber = 	 {},
  OPTseries = 	 {},
  OPTaddress = 	 {},
  OPTmonth = 	 {},
  OPTorganization = {},
  publisher = {IEEE Press},
  OPTnote = 	 {},
  OPTannote = 	 {},
  pdf = {./papers/2011-t4e-linux.pdf}
}


@InProceedings{Choppella-icdcit-2011,
  keywords = {conf-paper, modeling, formal-methods, software-engineering},
  author = 	 {Venkatesh Choppella and  Vamsikrishna Brahmajosyula and T B Dinesh and Nadin Kokciyan},
  title = 	 {Towards a declarative workflow model for customizing group processes},
  OPTcrossref =  {},
  OPTkey = 	 {},
  booktitle = {International Conference on Distributed Computing and Internet Technologies (ICDCIT 2011)},
  OPTpages = 	 {},
  year = 	 {2011},
  OPTeditor = 	 {},
  OPTvolume = 	 {},
  OPTnumber = 	 {},
  OPTseries = 	 {},
  OPTaddress = 	 {},
  month = 	 {February},
  OPTorganization = {},
  OPTpublisher = {},
  note = 	 {Oral presentation},
  OPTannote = 	 {}
}
\end{lstlisting}
\section{2010}
\label{sec:org313a60c}
\lstset{language=bibtex,label= ,caption= ,captionpos=b,numbers=none}
\begin{lstlisting}
@InProceedings{naidu-et-al-t4e-2010,
  keywords={conf-paper, vlabs, edutech, t4e},
  author = 	 {Thulasiram Naidu P and Manisha Verma and Venkatesh Choppella and Gangadhar Chalapaka},
  title = 	 {Synthesizing customizable learning environments},
  OPTcrossref =  {},
  OPTkey = 	 {},
  booktitle = {2nd IEEE International Conference on Technology for Education},
  OPTpages = 	 {},
  year = 	 {2010},
  OPTeditor = 	 {},
  OPTvolume = 	 {},
  OPTnumber = 	 {},
  OPTseries = 	 {},
  OPTaddress = 	 {},
  month = 	 {July},
  OPTorganization = {},
  OPTpublisher = {},
  OPTnote = 	 {},
  OPTannote = 	 {},
  pdf = {./papers/2010-t4e-naidu.pdf},
  abstract = {Making the experience of e-learning more
                  effective requires interactive and
                  collaborative systems to be adaptive and
                  customizable.  Specialized learning
                  systems tend to be monolithic and
                  difficult to extend.  We present an
                  alternative approach, where we synthesize
                  a customizable learning environment from
                  existing tools (Trac, SVN, reST, SQLite).
                  The system presents the student not just
                  with content, but an immersive experience
                  that allows both individual and group
                  annotations, versioning of the student's
                  work, custom querying, and a uniform
                  markup language to store content.  We
                  report the motivation and design of such
                  an environment.  We demonstrate the use of
                  this system and its ability to plug into
                  other environments by showcasing a custom
                  interactive workbook, built for teaching
                  and learning the principles of
                  programming.},

}


@InProceedings{Choppella-t4e-tutorial-2010,
  keywords={conf-paper, vlabs, edutech, foss, t4e},
  author = 	 {Venkatesh Choppella},
  title = 	 {FOSS, Web2.0 and Mashups as a Natural Learning Management Infrastructure},
  OPTcrossref =  {},
  OPTkey = 	 {},
  booktitle = {IEEE Conf. on Technology for Education},
  OPTpages = 	 {},
  year = 	 {2010},
  OPTeditor = 	 {},
  OPTvolume = 	 {},
  OPTnumber = 	 {},
  OPTseries = 	 {},
  OPTaddress = 	 {},
  month = 	 {July},
  OPTorganization = {},
  OPTpublisher = {},
  note = 	 {Conference tutorial},
  OPTannote = 	 {}
}
\end{lstlisting}

\section{2009}
\label{sec:org933702a}
\lstset{language=bibtex,label= ,caption= ,captionpos=b,numbers=none}
\begin{lstlisting}
@InProceedings{choppella-srivathsan-icegov-2009,
  keywords = {conf-paper, e-gov, web-technologies, ict4d, icegov},
  author = 	 {Venkatesh Choppella and K R Srivathsan},
  title = 	 {Fostering Community Interaction with the
                  Trivandrum City Police Portal},
  OPTcrossref =  {},
  OPTkey = 	 {},
  booktitle = {3rd ACM International Conference on the Theory
and Practice of E-Governance},
  pages = 	 {365--368},
  year = 	 {2009},
  OPTeditor = 	 {},
  OPTvolume = 	 {},
  OPTnumber = 	 {},
  OPTseries = 	 {},
  OPTaddress = 	 {},
  month = 	 {November},
  OPTorganization = {},
  OPTpublisher = {},
  OPTnote = 	 {},
  OPTannote = 	 {},
  pdf = {./papers/2009-icegov.pdf},
  abstract = {    The Trivandrum City Police Portal is an
                  example of fostering government-community
                  interaction in the area of law
                  enforcement.  The portal has been in
                  operation since 2004 and has been widely
                  used by the city's police force and
                  citizens.  The paper discusses the origins
                  and motivation of the police portal
                  project, the design of the system in terms
                  of its functional interfaces, lessons
                  learned in implementing and running the
                  system for the last few years, some
                  limitations of the current implementation
                  and scope for further work.  }
}


\end{lstlisting}
\section{2007}
\label{sec:orgfee4056}
\lstset{language=bibtex,label= ,caption= ,captionpos=b,numbers=none}
\begin{lstlisting}
@InProceedings{dinesh-choppella-iceg-2010,
  keywords = {conf-paper, e-gov, formal-methods, software-architecture},
  author = 	 {T B Dinesh and Venkatesh Choppella},
  title = 	 {A case for process-driven models for e-governance architectures},
  OPTcrossref =  {},
  OPTkey = 	 {},
  booktitle = {7th International Conference on E-Government},
  OPTpages = 	 {},
  year = 	 {2010},
  OPTeditor = 	 {},
  OPTvolume = 	 {},
  OPTnumber = 	 {},
  OPTseries = 	 {},
  OPTaddress = 	 {},
  month = 	 {April},
  OPTorganization = {},
  OPTpublisher = {},
  OPTnote = 	 {},
  OPTannote = 	 {},
  pdf = {./papers/2010-iceg.pdf},
  abstract = {    Because of their potentially wide impact, e-
                  governance systems beg the question of
                  validation. How do we know an e-governance
                  implementation does what it is supposed to
                  do? How do we even know what it is
                  supposed to do? Such questions are routine
                  in the field of software engineering and
                  are referred to, respectively as
                  verification and specification.
                  Verification and specification are tied
                  together via a model.  Software engineers
                  call this model-driven design. The models
                  most suitable for e-governance are a
                  combination of data and processes. We
                  introduce such a process-driven meta-model
                  and show how it could usefully describe
                  systems with e-governance behaviour.}
}


@InProceedings{choppella-et-al-afm07,
  keywords = {conf-paper, formal-methods, modelling, data-modelling, afm},
  author = 	 {Venkatesh Choppella and Arijit Sengupta and Ed Robertson and Steven Johnson},
  title = 	 {Prelimary Explorations in Specifying and Verifying Entity-Relationship models in PVS},
  OPTcrossref =  {},
  OPTkey = 	 {},
  booktitle =    {Proceedings of AFM'07: Second ACM workshop on Automated Formal Methods},
  pages = 	 {1-10},
  year = 	 {2007},
  editor = 	 {Natarajan Shankar and John Rushby},
  OPTvolume = 	 {},
  OPTnumber = 	 {},
  OPTseries = 	 {},
  OPTaddress = 	 {},
  month = 	 {November},
  OPTorganization = {},
  publisher = {ACM Press},
  OPTnote = 	 {},
  OPTannote = 	 {supercedes~\cite{choppella-sengupta-robertson-johnson-tr-2006}},
  pdf = {./papers/2007-afm.pdf},
  OPTabstract = {    Entity-Relationship (ER) diagrams are an
                  established way of doing data modeling.
                  In this paper, we report our experience
                  with exploring the use of PVS to formally
                  specify and reason with ER data models.
                  Working with a text-book example, we rely
                  on PVS's theory interpretation mechanism
                  to verify the correctness of the mapping
                  across various levels of abstraction.
                  Entities and relationships are specified
                  as user defined types, while constraints
                  are expressed as axioms.  We demonstrate
                  how the correctness of the mapping from
                  the abstract to a conceptual ER model and
                  from the conceptual ER model to a schema
                  model is formally established by using
                  typechecking.  The verification involves
                  proving the type correctness conditions
                  automatically generated by the PVS type
                  checker.  The proofs of most of the type
                  correctness conditions are fairly small
                  (four steps or less).  This holds out
                  promise for complete automatic formal
                  verification of data models.}
}


\end{lstlisting}
\section{2006}
\label{sec:org62ba4b8}
\lstset{language=bibtex,label= ,caption= ,captionpos=b,numbers=none}
\begin{lstlisting}
@Article{krishnan-et-al-jpdc-2006,
  keywords = {journal-article, tce, compilers, hpc, algorithms, jpdc},
  author = 	 {Sandhya Krishnan and Sriram Krishnamoorthy and Gerald Baumgartner and Chi-Chung Lam and J. Ramanujam and P. Sadayappan and Venkatesh Choppella},
  title = 	 {Efficient Synthesis of out-of-core algorithms using a nonlinear optimization solver},
  journal = 	 {Journal of Parallel and Distributed Computing},
  year = 	 {2006},
  OPTkey = 	 {},
  volume = 	 {66},
  OPTnumber = 	 {},
  pages = 	 {659-673},
  OPTmonth = 	 {},
  note = {\textbf{Invited Submission}. Supercedes~\cite{Krishnan-et-al-IPDPS-2004}},
  OPTannote = 	 {},
  pdf = {./papers/2006-jpdc.pdf},
  abstract = {We address the problem of efficient
out-of-core code generation for a special class of
imperfectly nested loops encoding tensor contractions
arising in quantum chemistry computations. These loops
operate on arrays too large to fit in physical memory. The
problem involves determining optimal tiling of loops and
placement of disk I/O statements. This entails a search in
an explosively large parameter space. We formulate the
problem as a nonlinear optimization problem and use a
discrete constraint solver to generate optimized out-ofcore
code. The solution generated using the discrete constraint
solver consistently outperforms other approaches by up to a
factor of four.  Measurements on sequential and parallel
versions of the generated code demonstrate the effectiveness
of the approach.}
}

@TechReport{choppella-sengupta-robertson-johnson-tr-2006,
  keywords = {tech-report, data-modelling, formal-methods, modelling, iucs},
  author = 	 {Venkatesh Choppella and Arijit Sengupta and Edward Robertson and Steven D.~Johnson},
  title = 	 {{C}onstructing and {V}alidating {E}ntity-{R}elationship models in the {PVS} {S}pecification {L}anguage: A case study using a text-book example},
  institution =  {Indiana University Computer Science},
  year = 	 {2006},
  OPTkey = 	 {},
  OPTtype = 	 {},
  number = 	 {632},
  OPTaddress = 	 {},
  month = 	 {April},
  OPTnote = 	 {},
  OPTannote = 	 {},
  pdf =          {./papers/2006-iucs-tr632.pdf},
  abstract = {    Data Modeling frameworks like the
                  Entity-Relationship (ER) approach are
                  usually specified using graphical and
                  natural language representations.  This
                  limits the ability to formally express and
                  verify the consistency of constraints on
                  data models.  The use of mathematical
                  notation makes the specification precise,
                  but also complex and tedious to write,
                  and, in the absence of automated support
                  for validation, error prone.  We use the
                  PVS specification language and its theorem
                  proving environment to formally construct,
                  reason with, and mechanically validate an
                  example data model at various levels of
                  abstraction.  The methodology proposed
                  here makes modeling resemble programming
                  in a strongly typed language.  Models are
                  implemented as PVS theories consisting of
                  type declarations, function definitions,
                  axioms and theorems.  Entities and
                  relationships are expressed as types.
                  Constraints on the data model are
                  expressed as axioms relating entity and
                  relationship sets.  Additional correctness
                  conditions are generated by PVS's type
                  checker.  Using the theory interpretation
                  mechanism of PVS, we prove the correctness
                  of the example's logical model with
                  respect to its ER model.  The example
                  model we consider has about fifteen
                  attributes, entities and relationships,
                  and twelve constraints.  The complete
                  hand-coded specification of the model is
                  about 600 lines of PVS (including
                  libraries).  Verification of the
                  correctness of the model reduces to
                  interactively proving about thirty
                  correctness conditions.  The proofs of
                  almost all of these are quite small (4
                  steps or less).  With modest additional
                  effort, it should be possible to
                  automatically generate the specification
                  and proofs, paving the way for automatic
                  verification of data models.  We see our
                  work as the initial step towards this
                  goal.  }
}

@InProceedings{Hartono-et-al-iccs-2006,
  keywords = {conf-paper, tce, compilers, hpc, algorithms, iccs},
  author = 	 {A. Hartono and Qingda Lu and Xiaoyang Gao and Sriram Krishnamoorthy and Marcel Nooijen and Gerald Baumgartner and David E. Bernholdt and Venkatesh Choppella and Russel M. Pitzer and J Ramanujam and Atanas Rountev and P. Sadayappan},
  title = 	 {Identifying cost-effective common subexpressions to reduce operation count in tensor contraction evaluations},
  OPTcrossref =  {},
  OPTkey = 	 {},
  booktitle = {Proceedings of the International Conference on Computational Science (ICCS), Part 1},
  pages = 	 {267-275},
  year = 	 {2006},
  editor = 	 {V. N. Alexandrov et al.},
  volume = 	 {3991},
  OPTnumber = 	 {},
  series = 	 {Lecture Notes in Computer Science},
  OPTaddress = 	 {},
  OPTmonth = 	 {},
  OPTorganization = {},
  publisher = {Springer-Verlag},
  OPTnote = 	 {},
  OPTannote = 	 {},
  pdf = {./papers/2006-iccs.pdf},
  abstract = {Complex tensor contraction expressions arise
in accurate electronic structure models in quantum
chemistry, such as the coupled cluster method.
Transformations using algebraic properties of commutativity
and associativity can be used to significantly decrease the
number of arithmetic operations required for evaluation of
these expressions. Operation minimization is an important
optimization step for the Tensor Contraction Engine, a tool
being developed for the automatic transformation of
high-level tensor contraction expressions into efficient
programs. The identification of common subexpressions among
a set of tensor contraction expressions can result in a
reduction of the total number of operations required to
evaluate the tensor contractions. In this paper, we develop
an effective algorithm for common subexpression
identification and demonstrate its effectiveness on tensor
contraction expressions for coupled cluster equations.}  
}

@Article{Auer-MolPhysics-2006,
  keywords = {journal-article, tce, compilers, hpc, algorithms, mol-physics},
  author = 	 {Alexander Auer and Gerald Baumgartner and David
                  E. Bernholdt and Alina Bibireata and Venkatesh
                  Choppella and Daniel Cociorva and Xiaoyang Gao and
                  Robert Harrison and Sriram Krishanmoorthy and
                  Sandhya Krishnan and Chi-Chung Lam and Marcel
                  Nooijen and Russell Pitzer and J. Ramanujam and
                  P. Sadayappan and Alexander Sibiryakov}, 
  title = 	 {Automatic Code Generation for Many-Body Electronic
                  Structure Methods: The {T}ensor {C}ontraction
                  {E}ngine},
  volume = 	 {104},
  number = 	 {2},
  pages = 	 {211-228},
  month = 	 {January},
  year =	 {2006},
  journal = {Molecular Physics},
  note = {\textbf{Invited paper}.  R. J. Bartlett Festschrift},
  pdf = {./papers/2006-molphysics.pdf},
  OPTabstract = {As both electronic structure methods and the
computers on which they are run become increasingly complex,
the task of producing robust, reliable, high-performance
implementations of methods at a rapid pace becomes
increasingly daunting. In this paper we present an overview
of the Tensor Contraction Engine (TCE), a unique effort to
address issues of both productivity and performance through
automatic code generation. The TCE is designed to take
equations for many-body methods in a convenient high-level
form and acts like an optimizing compiler, producing an
implementation tuned to the target computer system and even
to the specific chemical problem of interest. We provide
examples to illustrate the TCE approach, including the
ability to target different parallel programming models, and
the effects of particular optimizations.}

}

\end{lstlisting}
\section{2005}
\label{sec:orgc5eced0}
\lstset{language=bibtex,label= ,caption= ,captionpos=b,numbers=none}
\begin{lstlisting}
@Article{Choppella-Haynes-IAC-2005,
  keywords = {journal-article, unification, theory, rewrite-systems, proof-theory, logic, iac},
  author = 	 {V. Choppella and C. T. Haynes},
  title = 	 {Sourcetracking Unification (Revised and Extended Version)},
  journal = 	 {Information and Computation},
  year = 	 {2005},
  OPTkey = 	 {},
  volume = 	 {201},
  number = 	 {2},
  pages = 	 {121-159},
  month = 	 {September},
  note = 	 {\textbf{Invited Submission}.  Supercedes~\cite{Choppella-Haynes-CADE-2003}},
  annote = 	 {\href{papers/2005-information-and-computation-source-tracking-unification}{[pdf]} supercedes~\cite{Choppella-Haynes-CADE-2003}},
  pdf = {./papers/2005-iac.pdf},
abstract = {
We propose a path-based frameworkfor deriving and
simplifying source-tracking information for firstorder term
unification in the empty theory. Such a frameworkis useful
for diagnosing unification-based systems, including
debugging of type errors in programs and the generation of
success and failure proofs in logic programming. The objects
of source-tracking are deductions in the logic of term
unification. The semantics of deductions are paths over a
unification graph whose labels form the suffix language of a
semi- Dyckset. Based on this idea of unification paths, two
algorithms for generating proofs are presented: the first
uses context-free labeled shortest-path algorithms to
generate optimal (shortest) proofs in time O(n3) for a fixed
signature, where n is the number of vertices of the
unification graph. The second algorithm integrates easily
with standard unification algorithms, entailing an overhead
of only a constant factor, but generates non-optimal
proofs. These non-optimal proofs may be further simplified
by group rewrite rules.
}
} 

@Article{Baumgartner-ieee-2005,
  keywords = {journal-article, tce, compilers, hpc, algorithms, ieee},
  author = 	 {Gerald Baumgartner and Alexander Auer and David
                  E. Bernholdt and Alina Bibireata and Venkatesh
                  Choppella and Daniel Cociorva and Xiaoyang Gao and
                  Robert Harrison and So Hirata and Sriram
                  Krishanmoorthy and Sandhya Krishnan and Chi-Chung
                  Lam and Marcel Nooijen and Russell Pitzer and
                  J. Ramanujam and P. Sadayappan and Alexander
                  Sibiryakov}, 
  title = 	 {Synthesis of High-Performance Parallel Programs for
                  a Class of Ab Initio Quantum Chemistry Models},
  year =	 {2005},
  journal = {Proc. of the IEEE},
  volume = {93},
  number = {2},
  pages = {276--292},
  month = {February},
  note = {\textbf{Invited Paper}},
  copyright= {IEEE},
  pdf = {./papers/2005-ieee.pdf},
  OPTabstract = {This paper provides an overview of a program
synthesis system for a class of quantum chemistry
computations.  These computations are expressible as a set
of tensor contractions and arise in electronic structure
modeling.  The input to the system is a high-level
specification of the computation, from which the system can
synthesize high performance parallel code tailored to the
characteristics of the target architecture.  Several
components of the synthesis system are described, focusing
on performance optimization issues that they address.}  
}


\end{lstlisting}

\section{2004}
\label{sec:org7733463}
\lstset{language=bibtex,label= ,caption= ,captionpos=b,numbers=none}
\begin{lstlisting}

@InProceedings{Krishnan-et-al-IPDPS-2004,
  keywords = {conf-paper, tce, compilers, hpc, algorithms, ieee},
  author = 	 {S.~Krishnan and  S.~Krishnamoorthy and  G.~Baumgartner and  C-C.~Lam and J.~Ramanujam and  P.~Sadayappan and V.~Choppella},
  title = 	 {Efficient Synthesis of Out-of-core Algorithms Using a Nonlinear Optimization Solver},
  OPTcrossref =  {},
  OPTkey = 	 {},
  booktitle = {Proc. International Parallel and
Distributed Processing Symposium (IPDPS 2004), Albuquerque, New Mexico, USA},
  OPTpages = 	 {},
  year = 	 {2004},
  OPTeditor = 	 {},
  OPTvolume = 	 {},
  OPTnumber = 	 {},
  OPTseries = 	 {},
  OPTaddress = 	 {},
  month = 	 {April},
  OPTorganization = {},
  publisher = {IEEE Computer Society},
  note = 	 {ISBN: 0-7695-2132-0},
  OPTannote = 	 {},
  pdf = {./papers/2004-ipdps.pdf},
  abstract = {
We address the problem of efficient out-of-core code generation
for a special class of imperfectly nested loops encoding tensor
contractions.  These loops operate on arrays too large to fit in
physical memory.  The problem involves determining optimal
tiling and placement of disk I/O statements.  This entails a
search in an explosively large parameter space.  We formulate
the problem as a non-linear optimization problem and use a
discrete constraint solver to generate optimized out-of-core code.
Measurements on sequential and parallel versions of the
generated code demonstrate the effectiveness of the proposed approach.
}
}


@InProceedings{Bibireata-et-al-LCPC-2004,
  keywords = {conf-paper, tce, compilers, hpc, algorithms, ieee},
  author = 	 {A.~Bibireata and  S.~Krishnan and  G.~Baumgartner and  D.~Cociorva and C-C.~Lam and  P.~Sadayappan and  J.~Ramanujam and  D.~E.~Bernholdt and
V.~Choppella},
  title = 	 {Memory-constrained Data Locality
Optimizations for Tensor Contractions},
  OPTcrossref =  {},
  OPTkey = 	 {},
  booktitle = {Proc. 16th
International Workshop on Languages and Compilers for Parallel
Computing (LCPC '03)},
  pages = 	 {93-108},
  year = 	 {2004},
  editor = 	 {},
  OPTvolume = 	 {},
  number = 	 {2958},
  series = 	 {Lecture Notes in Computer Science},
  address = 	 {College Station, Texas},
  OPTmonth = 	 {},
  OPTorganization = {},
  publisher = {Springer},
  OPTnote = 	 {},
  OPTannote = 	 {},
  pdf = {./papers/2004-lcpc.pdf},
  abstract = {The accurate modeling of the electronic structure of atoms
and molecules involves computationally intensive tensor contractions
over large multi-dimensional arrays. Efficient computation of these contractions
usually requires the generation of temporary intermediate arrays.
These intermediates could be extremely large, requiring their storage
on disk. However, the intermediates can often be generated and used
in batches through appropriate loop fusion transformations. To optimize
the performance of such computations a combination of loop fusion and
loop tiling is required, so that the cost of disk I/O is minimized. In
this paper, we address the memory-constrained data-locality optimization
problem in the context of this class of computations. We develop
an optimization framework to search among a space of fusion and tiling
choices to minimize the data movement overhead. The effectiveness of
the developed optimization approach is demonstrated on a computation
representative of a component used in quantum chemistry suites.}
}

@Unpublished{Choppella-2004-compositionality,
  keywords = {unpublished, unification, theory, rewrite-systems, unif},
  author = 	 {Venkatesh Choppella},
  title = 	 {A compositionality principle for unification},
  note = 	 {Accepted for publication at UNIF 2004},
  OPTkey = 	 {},
  OPTmonth = 	 {},
  year = 	 {2004},
  pdf  =         {./papers/2004-unif.pdf},
  OPTabstract  =    {Many unification-based type reconstruction algorithms operate by
constructing and composing unifying substitutions in a
sequential, rather than compositional manner.  This
sequentiality makes the debugging of substitution-based
algorithms difficult.  We show two different
ways in which the compositionality of algorithms for
Hindley-Milner type reconstruction can be improved using
elementary results from the algebra of unifiers.
The first method employs a compositionality principle for
syntactic unification, while the second is based on defining
type reconstruction algorithms to generate term equations
instead of substitutions.  We discuss how improving
compositionality improves the quality of error diagnosis in
Hindley-Milner type reconstruction.}
}


\end{lstlisting}

\section{2003}
\label{sec:orgd37f517}
\lstset{language=bibtex,label= ,caption= ,captionpos=b,numbers=none}
\begin{lstlisting}

@InProceedings{Krishnan-et-al-HiPC-2003,
  keywords = {conf-paper, tce, compilers, hpc, algorithms, ieee},
  author = 	 {S. Krishnan and  S. Krishnamoorthy and  G. Baumgartner and  D. Cociorva and C. Lam and  P.  Sadayappan and  J. Ramanujam and  D. E. Bernholdt and
V. Choppella},
  title = 	 {Data Locality Optimization for Synthesis of
Efficient Out-of-Core Algorithms},
  OPTcrossref =  {},
  OPTkey = 	 {},
  booktitle = {Proc. of the Intl. Conf. on High Performance Computing (HiPC 2003)},
  pages = 	 {406--417},
  year = 	 {2003},
  OPTeditor = 	 {},
  OPTvolume = 	 {},
  number = 	 {2913},
  series = 	 {Lecture Notes in Computer Science},
  address = 	 {Hyderabad, India},
  month = 	 {December},
  OPTorganization = {},
  publisher = {Springer},
  OPTnote = 	 {},
  OPTannote = 	 {},
  pdf  = {./papers/2003-hipc.pdf},
  abstract  = {
This paper describes an approach to synthesis of efficient out-of-core
code for a class of imperfectly nested loops that represent
tensor contraction computations. Tensor contraction expressions arise in
many accurate computational models of electronic structure. The developed
approach combines loop fusion with loop tiling and uses a performance-model
driven approach to loop tiling for the generation of out-of-core code.
Experimental measurements are provided that show a good match with 
model-based predictions and demonstrate the effectiveness
of the proposed algorithm.
}
}


@InProceedings{Choppella-IFL-2003,
  keywords = {conf-paper, types, programming-languages, compilers, logic, ifl},
  author = 	 {Venkatesh Choppella},
  title = 	 {Polymorphic Type Reconstruction using Type Equations},
  OPTcrossref =  {},
  OPTkey = 	 {},
  booktitle = {Implementation of Functional Languages: 15th International Workshop, IFL 2003, Edinburgh, UK},
  pages = 	 {53--68},
  year = 	 {2004},
  editor = 	 {Phil Trinder, Greg Michaelson and Recardo Pe\~{n}a},
  volume = 	 {3145},
  OPTnumber = 	 {},
  series = 	 {Lecture Notes in Computer Science},
  OPTaddress = 	 {},
  month = 	 {December},
  OPTorganization = {},
  publisher = {Springer Verlag},
  note = 	 {ISBN: 3-540-23727-5},
  OPTannote = 	 {},
  pdf = {./papers/2003-ifl.pdf},
  abstract = { The $W$ algorithm of Milner and its numerous
variants implement type reconstruction by building type
substitutions.  We define an algorithm $W^{E}$ centered around
building type equations rather than substitutions.  The design
of $W^{E}$ is motivated by the belief that reasoning with
substitutions is awkward.  More seriously, substitutions fail to
preserve the exact syntactic form of the type equations they
solve.  This makes analysing the source of type errors more
difficult.  By replacing substitution composition with unions of
sets of type equations and eliminating the application of
substitution to environments, we obtain an algorithm for type
reconstruction that is simple and also useful for type error
reconstruction.  We employ a sequentiality principle for unifier
composition and a constructive account of mgu-induced variable
occurrence relation to design $W^{E}$ and prove its correctness.
We introduce syntax equations as a formal syntax for progam
slices.  We use a simple constraint generation relation to
relate syntax equations with type equations to trace program
slices responsible for a type error.}  } 

@InProceedings{Choppella-Haynes-CADE-2003,
  keywords = {conf-paper, unification, programming-languages, rewrite-systems, logic, cade},
  author = 	 {Venkatesh Choppella and Chistopher T. Haynes},
  title = 	 {{Source-tracking Unification}},
  OPTcrossref =  {},
  OPTkey = 	 {},
  booktitle = {Proceedings of 19th International Conference on Automated Deduction, CADE-19, Miami Beach, USA},
  pages = 	 {458-472},
  year = 	 {2003},
  editor = 	 {Franz Baader},
  OPTvolume = 	 {},
  number = 	 {2741},
  series = 	 {Lecture Notes in Artificial Intelligence},
  OPTaddress = 	 {},
  OPTmonth = 	 {},
  OPTorganization = {},
  publisher = {Springer},
  note = 	 {Superceded by ~\cite{Choppella-Haynes-IAC-2005}},
  OPTannote = 	 {\href{papers/2003-cade-source-tracking-unification.pdf}{[pdf]}},
  pdf = {./papers/2003-cade.pdf},
OPTabstract = {
We propose a practical path-based framework for deriving and
simplifying source-tracking information for term unification in
the empty theory.  Such a framework is  useful for
debugging unification-based systems, including the diagnosis of
ill-typed programs and the generation of success and failure
proofs in logic programming.

The objects of source-tracking are deductions in the logic of
unification.  The semantics of deductions are paths over a
unification graph whose labels form the language of suffixes of
a semi-Dyck set.  Based on this framework, two algorithms for
generating proofs are presented: the first uses context-free
shortest-path algorithms to generate optimal (shortest) proofs
in time $O(n^3)$, where $n$ is the number of vertices of the
unification graph.  The second algorithm integrates easily with
standard unification algorithms, entailing an overhead of only a
constant factor, but generates non-optimal proofs.  These
non-optimal proofs may be further simplified by
group rewrite rules.
}
}


\end{lstlisting}
\section{2002}
\label{sec:orgb63ffbf}
\lstset{language=bibtex,label= ,caption= ,captionpos=b,numbers=none}
\begin{lstlisting}
@TechReport{WSCL-2002,
  keywords = {tech-report, workflows, modelling, web-standards, w3c},
  author = 	 {A. Banerji and  C. Bartolino and D. Beringer and V. Choppella,
K. Govindarajan and  A. Karp and H. Kuno and M. Lemon and  G. Pogossiants and
S. Sharma and  S. Williams},
  title = 	 {Web Services Conversation Language (WSCL) 1.0},
  institution =  {Hewlett-Packard Company},
  year = 	 {2002},
  OPTkey = 	 {},
  OPTtype = 	 {},
  OPTnumber = 	 {},
  OPTaddress = 	 {},
  month = 	 {March},
  note = 	 {World Wide Web Consortium Note \url{http://www.w3.org/TR/wscl10}},
  OPTannote = 	 {},
  url       =    {http://www.w3.org/TR/wscl10},
  abstract = {
This document specifies the Web Services Conversation
Language. WSCL allows the abstract interfaces of Web services,
i.e. the business level conversations or public processes
supported by a Web service, to be defined. WSCL specifies the
XML documents being exchanged, and the allowed sequencing of
these document exchanges. WSCL conversation definitions are
themselves XML documents and can therefore be interpreted by Web
services infrastructures and development tools. WSCL may be used
in conjunction with other service description languages like
WSDL; for example, to provide protocol binding information for
abstract interfaces, or to specify the abstract interfaces
supported by a concrete service.  }
}



@PhdThesis{Choppella-thesis-02,
keywords = {thesis, unification, theory, rewrite-systems, proof-theory, programming-languages, types, logic, iu},
  author = 	 "Venkatesh Choppella",
  title = 	 "Unification Source-tracking with Application to Diagnosis of Type Inference",
  school = 	 "Indiana University",
  year = 	 "2002",
  OPTcrossref =  "",
  OPTkey = 	 "",
  OPTaddress = 	 "",
  month = 	 "August",
  OPTtype = 	 "",
  note = 	 "IUCS Tech Report TR566",
  url = {http://www.cs.indiana.edu/cgi-bin/techreports/TRNNN.cgi?trnum=TR566},
  pdf = {./papers/2002-tr566.pdf},
  OPTabstract = {
Prior diagnoses in unification-based type reconstruction systems
have either missed information that is relevant, presented
irrelevant details, or both.  

We use a framework based on the Unification Logics of
Le~Chenadec to define, derive and simplify proof-based
source-tracking for term unification.  The objects of
source-tracking are proofs in this deduction system, and
correspond to path expressions over a unification graph whose
labels form a semi-Dyck language of balanced parentheses.
Simplification of source-tracking information is implemented as
proof normalization in the rewrite system for free groups.
Subject-reduction properties guarantee that normalization
preserves the semantics of deductions.  The presentation of the
logic facilitates proof construction by a simple extension to
standard unification algorithms.

We apply unification source-tracking to type inference in the
Curry-Hindley type system.  Programs are represented as systems
of syntax equations.  Program slices correspond to weakenings of
syntax and type equations.  A constraint generation function
maps weakenings of type equations to weakenings of syntax
equations.  Source-tracking information is defined in terms of
the inverse of this generating function.

Unification is central to many applications of symbolic
computation and artificial intelligence, including computer
algebra, automated theorem proving, expert systems, and
programming language type systems.  Source-tracking is a
debugging technique based on tracing the execution of a program
to identify those subparts that contribute to the result of the
execution.
}
}
\end{lstlisting}
\section{2000}
\label{sec:orgd655f94}
\lstset{language=bibtex,label= ,caption= ,captionpos=b,numbers=none}
\begin{lstlisting}
@InProceedings{Govindaraju-et-al-Supercomputing-2000,
  keywords = {conf-paper, hipc, protocols, xml, sc},
  author = 	 {M. Govindaraju and A. Slomenski and V. Choppella and R. Bramley and D. Gannon},
  title = 	 {Requirements for and Evaluation of RMI Protocols for
On the Performance of Remote Method Invocation for Scientific Computing},
  OPTcrossref =  {},
  OPTkey = 	 {},
  booktitle = {Proc. of the IEEE/ACM International Conference on Supercomputing (SC 2000)},
  OPTpages = 	 {},
  year = 	 {2000},
  OPTeditor = 	 {},
  OPTvolume = 	 {},
  OPTnumber = 	 {},
  OPTseries = 	 {},
  OPTaddress = 	 {},
  month = 	 {November},
  OPTorganization = {},
  OPTpublisher = {},
  OPTnote = 	 {},
  OPTannote = 	 {},
  pdf = {./papers/2000-sc.pdf},
  OPTabstract = {Distributed software component architectures
provide a promising approach to the problem of building large
scale, scientific Grid applications. Communication in these
component architectures is based on Remote Method Invocation
(RMI) protocols that allow one software component to invoke the
functionality of another. Examples include Java remote method
invocation (Java RMI) and the new Simple Object Access
Protocol (SOAP). SOAP has the advantage that many
programming languages and component frameworks can support it.
This paper describes experiments showing that SOAP by itself is
not e cient enough for large scale scientific
applications. However, when it is embedded in a multi-protocol
RMI framework, SOAP can be effectively used as a universal
control protocol that can be swapped out by faster, more special
purpose protocols when large data transfer speeds are needed.
}
}
\end{lstlisting}
\section{1999}
\label{sec:org7c404b0}
\lstset{language=bibtex,label= ,caption= ,captionpos=b,numbers=none}
\begin{lstlisting}
@Patent{1999-aop-patent,
 keywords={aop, patent},
 author = 	 {Cristina Lopes and Gregor Kiczales and John Lamping and
                  Erik Hilsdale and Venkatesh Choppella and Taher
                  Haveliwala},
  title = {{A}spect-{O}riented {S}ystem {M}onitoring and {T}racing},
  number = 		 {09/357,508},
  year = 	 {1999},
  ALTdate = 	 {},
  OPTkey = 		 {},
  OPTholder = 	 {},
  OPTsubtitle =  {},
  OPTtitleaddon = {},
  OPTtype = 	 {},
  OPTversion = 	 {},
  location =  {United States},
  note ={Awarded April 2002},  
  OPTmonth = 	 {},
  OPTyear = 	 {},
  OPTaddendum =  {},
  OPTpubstate =  {},
  OPTdoi = 		 {},
  OPTeprint = 	 {},
  OPTeprintclass = {},
  OPTeprinttype = {},
  url = 		 {http://www.patentstorm.us/patents/6473895.html},
  OPTurldate = 	 {},
  OPTannote = 	 {}
}
\end{lstlisting}
\section{1996}
\label{sec:orga4b87d7}
\lstset{language=bibtex,label= ,caption= ,captionpos=b,numbers=none}
\begin{lstlisting}
@InProceedings{Bose-Tuna-Choppella-FMCAD-1996,
  keywords = {conf-paper, formal-methods, vlsi, compilers, fmcad},
  author = 	 {B. Bose and M. E. Tuna and V. Choppella},
  title = 	 {{Tutorial on Digital Design Derivation with DRS}},
  OPTcrossref =  {},
  OPTkey = 	 {},
  booktitle = {Proc. 1st International Conf. on Formal Methods in Computer Aided Design, (FMCAD '96), Palo Alto, CA, USA},
  OPTpages = 	 {},
  year = 	 {1996},
  OPTeditor = 	 {},
  OPTvolume = 	 {},
  number = 	 {1166},
  series = 	 {Lecture Notes in Computer Science},
  OPTaddress = 	 {},
  month = 	 {November},
  OPTorganization = {},
  publisher = {Springer},
  OPTnote = 	 {},
  OPTannote = 	 {},
  pdf =          {./papers/1996-fmcad.pdf},

  abstract = {This paper presents a tutorial on digital
  design derivation using DRS.  The DRS system is an
  integrated formal system for the design of verified
  hardware.  The underlying approach employs a derivation
  methodology in which a series of correctness preserving
  transformations are applied to high-level specifications
  in order to synthesize hardware descriptions.  In this
  paper, we sketch the key steps in the derivation of an
  example circuit.  The example illustrates several aspects
  of DRS and serves as an introduction to the derivational
  paradigm of synthesis. }
} 
\end{lstlisting}

\section{1995}
\label{sec:org87d7da7}
\lstset{language=bibtex,label= ,caption= ,captionpos=b,numbers=none}
\begin{lstlisting}
@Article{Rath-Choppella-Johnson-1995,
  keywords = {journal-article, formal-methods, vlsi},
  author = 	 {K.~Rath and  V.~Choppella and S.~D.~Johnson},
  title = 	 {Decomposition of Sequential Behavior using Interface Specification and
Complementation},
  journal = 	 {VLSI Design, Special Issue on Decomposition},
  year = 	 {1995},
  OPTkey = 	 {},
  volume = 	 {3},
  number = 	 {3--4},
  pages = 	 {347--358},
  OPTmonth = 	 {},
  OPTnote = 	 {},
  OPTannote = 	 {},
  pdf = {./papers/1995-vlsi-design.pdf},
  abstract = {
Decomposition of functional behavior along system boundaries
into interacting sequential components is a key step in top-down
system design.  In this paper, we present {\em sequential
decomposition}, a method for factoring sequential components
from a system specification based on interface specification of
components.  The resulting components can be independently
synthesized, or realized using off-the-shelf components.  We
introdue {\em interface specification language (ISL)}, based on
finite machine semantics, to specify input/output behavior of
synchronous subsystems.  A component is factored from a system
by embedding an {\em implementation\/} of the {\em complement\/}
of its interface into the system description.  The {\em
composition\/} of a machine with its complement is shown to be
isomorphic to the machine, and the composition of a machine with
an implementation of its complement is shown to be a safe
interaction.  We apply sequential decomposition to a non-trivial
example, a special-purpose computer with Scheme programming
language primitives as its instructions.
}
}

@TechReport{Choppella-Haynes-1995,
  keywords = {tech-report, programming-languages, types, compilers, iu},
  author = 	 {Venkatesh Choppella and Chistopher T. Haynes},
  title = 	 {Diagnosis of Ill-typed Programs},
  institution =  {Indiana University},
  year = 	 {1995},
  OPTkey = 	 {},
  OPTtype = 	 {},
  number = 	 {426},
  OPTaddress = 	 {},
  month = 	 {February},
  OPTnote = 	 {},
  OPTannote = 	 {},
  OPTurl       =    {ftp://ftp.cs.indiana.edu/pub/techreports/TR426.pdf},
  pdf       =    {./papers/1995-iucs-tr426.pdf},
  OPTabstract = {A framework, based on syntactic and type constraints,
                  is provided for defining program slices that
                  contribute to a given type error or similar
                  syntactic property.  We specify soundness,
                  minimality and completeness criteria for these
                  slices and outline an algorithm for their lazy
                  evaluation.}}

@InProceedings{Bradford-Choppella-Rawlins-1995,
  keywords = {conf-paper, algorithms, latin},
  author = 	 {P. Bradford and  V. Choppella and G. J. E. Rawlins},
  title = 	 {Lower Bounds on the Matrix Chain Order problem},
  OPTcrossref =  {},
  OPTkey = 	 {},
  booktitle = {Proc. 2nd Latin American Symposium on Theoretical Informatics, (LATIN'95), Valparaiso, Chile},
  pages = 	 {112-130},
  year = 	 {1995},
  editor = 	 {Ricardo Baeza-Yates and Eric Goles and Patricio V. Poblete},
  OPTvolume = 	 {},
  OPTnumber = 	 {911},
  series = 	 {Lecture Notes in Computer Science},
  OPTaddress = 	 {},
  OPTmonth = 	 {},
  OPTorganization = {},
  publisher = {Springer},
  OPTnote = 	 {},
  annote = 	 {supercedes~\cite{bradford-choppella-rawlins-1993}},
}
\end{lstlisting}
\section{1993}
\label{sec:orga7eef74}
\lstset{language=bibtex,label= ,caption= ,captionpos=b,numbers=none}
\begin{lstlisting}
@TechReport{Bradford-Choppella-Rawlins-1993,
  keywords = {tech-report, algorithms, iucs},
  author = 	 {Philip Bradford and Venkatesh Choppella and Gregory J. E. Rawlins},
  title = 	 {Lower Bounds for the Matrix Chain Order Problem},
  institution =  {Indiana University},
  year = 	 {1993},
  OPTkey = 	 {},
  OPTtype = 	 {},
  number = 	 {TR 391},
  OPTaddress = 	 {},
  month = 	 {October},
  OPTnote = 	 {},
  annote = 	 {superceded by~\cite{bradford-choppella-rawlins-1995}},
  pdf = {./papers/1993-iucs-tr391.pdf},
}
\end{lstlisting}

\section{1987}
\label{sec:org1e3ba6a}
\lstset{language=bibtex,label= ,caption= ,captionpos=b,numbers=none}
\begin{lstlisting}
@MastersThesis{1987-choppella-mtech-thesis,
  keywords = {thesis, programming-languages, compilers, iitm},
  author = 		 {Venkatesh Choppella},
  title = 		 {Implementation of ML on the SECD machine},
  school = 		 {IIT Madras},
  year = 		 {1987},
  OPTkey = 		 {},
  OPTtype = 	 {},
  OPTaddress = 	 {},
  OPTmonth = 	 {},
  OPTnote = 	 {},
  OPTannote = 	 {}
}
\end{lstlisting}
\end{document}
