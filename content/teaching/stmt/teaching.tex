\documentclass[
%titlepage, 
%separate page for title
9pt, 
%twoside % page numbers on the outside
]{article}
%{report}
\usepackage[pdftex,dvipsnames,x11names,svgnames,table,fixpdftex,hyperref]{xcolor}
\usepackage{palatino}
\usepackage[square]{natbib}
\usepackage{array}
\usepackage{colortbl}
\usepackage{longtable}
\usepackage{dcolumn}
\usepackage{epigraph}
\usepackage{multirow}
\usepackage{graphicx}
%%% Comment the line below if you want a regular portrait
%%% style page layout 
% \include{pdfscreen}
\definecolor{lightgrey}{gray}{0.75}
%%% Comment the line below if you want a white background.
% \pagecolor{lightgrey}
\usepackage{palatino}
\usepackage[pdftitle={Proposal of Venkatesh Choppella's activities under Eureca Programme},%
colorlinks=true,% true
citecolor=blue,
linkcolor=Brown,
urlcolor=Brown,%navy%
]{hyperref}

\hypersetup{
pdfauthor= {Venkatesh Choppella <choppell@iiitmk.ac.in>},
pdfkeywords = {Research, Teaching, IIITM-K, Venkatesh Choppella, EURECA}
}

%                                  %%% pdfcreator, pdfproducer, 
%                                      and CreatioDate are automatically set
%                                      by pdflatex !!!
\pdfadjustspacing=1                %%% force LaTeX-like character spacing
%

\begin{document}
\title{Teaching Statement}
\author{Venkatesh Choppella}
\date{October 15th 2008}
\maketitle

% \subsection*{Introduction}


% In the rest of this document, I briefly outline my
% understanding of the role of computing science in today's
% science and engineering education, my efforts at modernizing
% the teaching of programming, and my plans for continuing and
% expanding the scope of these activities.

%  why it increasingly relevant in science
% and engineering.  I see software and programming are at the
% core of computing.  I discuss my efforts in the last few
% years towards redesigning programming education so that it
% connect principles, technologies, practices and domains.  I
% present my plans for expanding the sphere of my teaching
% activities in this area.
% \subsection*{Computing in science and engineering}

% \epigraph{Simulation and mathematical modeling will drive
%   the 21st century the way steam drove the 19th.}{{\sc
%     William Press}, author of {\em Numerical
%     Recipes\/}~\cite{numerical-recipes-book-2007}}
% % \footnote{Press's quote is from the blurb on the jacket of
% %   {\em The Nature of Mathematical
% %     Modeling}~\cite{nature-of-mathematical-modeling-book-1994}.}

% Computing today is increasingly indispensible to anyone
% managing information.
% % of sciences and engineering and technology.  %
% % For scientists to run their 21st century
% % simulations and mathematical models, they will need to have
% % more than a casual knowledge of computing skills.
% Ironically, the centrality of computing poses a challenge to
% its identity as a discipline.  

\paragraph{The main challenge in teaching computing}
Computing science is continuously reinventing itself to meet
the demands of other disciplines that depend on it.  This
also impacts the way computing is taught, because it
requires the conscientious teacher of computing to
continuously adapt himself and learn from the new
technologies, platforms and application domains where
computing is done.

% \subsection*{Teaching the Principles of Programing} 


\paragraph{Teaching the principles of programming}
My current teaching interests are focused on teaching
introductory programming and software design.  In the
Principles of Programming (PoP) course I teach, students
learn the fundamental ideas of how programs are built and
run, then consciously use these principles to build browser
based applications using current software engineering
practices.  At the end of the course, the students acquire
an understanding of concepts and an ability to program a
prototype web application.  They also learn the skills and
good practices of the profession and enjoy the confidence
and the excitement of building something that resembles a
popular application.

\paragraph{The complex structure of programs}
Most application programs today are simple in terms of their
formal algorithmic complexity.  Instead, the complexity
arises from the structure of their implementations.  There
is a need to balance correctness, efficiency and
adaptibility in the face of an ever-changing environment in
which these programs run.  This makes programming difficult.

\paragraph{Fashion oriented approach to learning programming and its limitations}
Students pick up the principles of computing from the (one)
course they do on programming.  Most programming courses
across the country today have reduced programming education
to learning the currently fashionable ``programming
paradigm'' in the currently fashionable programming language.
(These days it is object oriented programming in Java.)
Thus students focus on learning programming language syntax
and just enough library API's to get the job at hand done
(implement an algorithm, write a servlet, or paste together
a shell script for systems administration).  These
traditional programming tasks leave the student equipped
with hardly anything of substantial or lasting value.  They
miss out on the concepts needed to understand and adapt to
newer computing models, programming languages and
technologies.  The homework programs they do, like matrix
multiplication, and graph traversals and simple text
processing, reflect the algorithmic bias and the obsolete
view of programs as input-output artefacts.  These exercises
fail to bring to fore the complexity of programming, which
arises when trying to assemble program structures in the
presence of real world concerns like distribution,
concurrency, interaction, communication and large data sets.
Small, input-output exercises encourage the students to work
alone, focusing on the algorithm and not on the interfaces.
Students miss the opportunity to work with each other, which
involves, for example, code reviews, documentation, and
version control.  Thus they develop little of the social skills
needed to participate in a software project.

\paragraph{Definitional interpretation approach to learning programming}
I learnt most of what I know about the principles of
programming from landmark texts written by outstanding
computing scientists and educators~\cite{sicp,htdp,eopl}.
These texts introduce programming concepts by defining them
via interpreters.  Interpreters offer an analytical, as
opposed to a descriptive account of how programs run.  The
interpreter is a computational model -- a virtual computer
-- that runs other programs.  The best way to understand the
mechanics of programs is to build the machinery of
programming into an interpreter.  The concept of an
interpreter goes back to foundational ideas in the theory of
computability: the Universal Turing machine and the
$\lambda$-calculus.  Concepts like closures, lexical scoping
and continuations remain elusive to any amount of textual
description.  Their meaning in programs is effectively and
precisely conveyed when they are expressed in terms of their
implementations in an interpreter that runs those programs.

\paragraph{Computing needs a platform context}
Today, however, the challenge in programming education is to
teach the principles of computing not in isolation, but
within the demands of today's technologies, application
domains and engineering practices.  Programs no longer run
as stand alone entities.  The platform on which a program
runs (a web application, an embedded or mobile system, or a
parallel supercomputer) greatly influences the way it is
designed and implemented.

\paragraph{PoP:  bringing principles, platforms, and practices together}
In the PoP course I teach, the effort is to introduce
programming as an engineering science, and show the coming
together of concepts, platforms and practices to build real
world applications.  In PoP, students learn fundamental
concepts like abstraction, recursion, and interpretation.
They apply these concepts to build web applications that
they can see and click, all the while employing modern
software engineering practices.  Students use the
programming languages Scheme and Javascript and combine them
with HTML and DOM programming on a web browser.  Through a
series of home work assignments, they build a browser based
spreadsheet application ({\em \`a~la\/} Google
Spreadsheets).  The spreadsheet turns out to be an excellent
exercise.  Its real world nature and utility is immediate.
While building an online spreadsheet, the programming model
needs to adapt to the notion that a browser is a computing
platform that interprets Javascript.  The application needs
to handle interaction, data flow, concurrency, asynchronous
communication, persistence and many other concerns.

\paragraph{Encouraging industry attention}
PoP is garnering the attention of the IT industry.  The
semester course I teach at IIITM-K has drawn enrollment from
IT professionals.  Over the last year, I have taught this
course at two large IT firms.  Local companies in Trivandrum
have expressed interest in attending the course as well.  I
have talked about the course at several venues, most
recently at Google, Bangalore~\cite{venk-google-talk-2008}.

\paragraph{Teaching experience}
Other than PoP, I have taught or co-taught the following
courses: data structures, discrete mathematics, data bases,
compilers, web technologies, scientific computing,
computational biology, programming languages, denotational
semantics, and program verification.  Some of these courses
and student feedback are available
online~\cite{choppella-courses}.

\paragraph{Teaching plans}
For the next three years, I will work to consolidate and
popularize the Principles of Programming course.  I plan to
write a text book on this topic.  My coauthors in this
venture are Guillaume~Marceau (PhD student at Brown
University) and Dr.~T~B~Dinesh (computing scientist and
social entrepeneur based in Bangalore).  Simultaneously, I
plan to build a portable version of this course along the
lines of the NPTEL model of video instruction.  The course
will be supported by a lively online learning community
around an open wiki.  PoP and other courses could be part of
a professional development programme for the industry,
established along the lines of the Stanford Centre for
Professional Development~\cite{scpd}.  At a later stage, I
plan to design variants of the PoP course for mobile and
embedded computing platforms.  There will also be a version
of PoP developed for computational scientists.
Concurrently, I plan to teach seminar courses in type
systems, formal verification, software architectures and
computational biology.

% \paragraph{Formal notation}
% I am interested in developing pedagogy that introduces
% software engineers to formal notation at an early stage in
% their education.  Set theory, higher-order logic and
% axiomatic methods are eminently suited for providing precise
% formulations of many kinds of software requirements and
% design.  High level programming languages like Scheme and
% Haskell, and specification languages like Z and the
% Prototype Verification System (PVS) allow the programmer to
% use such formal notation as stylized equational logics.
% While the average mechanical or aerospace engineer will
% easily identify a rate equation, most programmers rarely
% even know, let alone use formal notations like sets,
% relations, and rewrite rules.  I plan to introduce formal
% notation early and often so programmers can think more like
% engineers.

% \paragraph{ICASE}
% Today's software and computer systems exhibit a convergence
% of concepts, methodologies and technologies and concepts not
% only from the subdisciplines of computer science, but from
% other sciences and engineering disciplines as well.
% Education in computer science should address how to harness
% this convergence.  

% My longer term agenda is to reformulate computer science
% education along the ideas mentioned in the preceding
% paragraphs, building in the process a programme for
% Integrated Computing and Software engineering Education
% (ICASE).  The emphasis on ICASE is to show how concepts
% across various subdisciplines of computing are connected.
% Theoretical foundations will be connected to models,
% methodologies, technologies and applications.  ICASE will
% systematically bring in the element of rigour in the
% teaching of designing of software systems so that it matures
% into an engineering discipline.

% ICASE will involve six courses: Principles of Programming,
% Logic and Discrete Mathematics, Data Structures, Database
% Design, Software Engineering and project management, and Web
% Technologies.  These courses will be closely linked with
% each other through assignments, homework problems and
% projects that take a concept, and elaborate its various
% aspects across different courses (e.g., first order logic
% and its exposition in a discrete maths course, its
% implementation in an interpreter in a programming course,
% and its use as a query language in a database course).
% ICASE could be offered as an undergraduate or master's level
% stream.

% \paragraph{Professional Development Programme}
% The ICASE initiative should be part of a larger professional
% development programme in computing, science and engineering.
% I would like to actively help IITH establish such a
% programme, which could use the the Stanford Centre for
% Professional Development~\cite{scpd} as its model.

% \paragraph{Instructional Technologies and FOSS in Education}
% I have been using free and open software (FOSS) since 1988,
% and I will continue to actively promote its use in science
% and engineering education, and general computing.  At
% IIITM-K, my current employer, I introduced several
% instructional technologies to make learning more effective:
% wikis, online student course feedback, google calendar and
% groups, version control, \LaTeX, and Emacs.  FOSS is also
% making huge inroads into the world of scientific computing.
% Pre-eminent institutes like the IIT's are specially
% positioned well to promote free and open source software
% (FOSS) through education.  

\paragraph{Conclusion: The romance of teaching}
I teach because I like to learn and share.  Today, we have
even greater opportunities than before to make learning and
education enjoyable, purposeful and share it with a world
wide classroom of lifelong learners.  At the same time,
there are so many more opportunities to make education
holistic and socially useful.  All this needs new ideas in
cross-disciplinary curriculum development, and innovations
in instructional technologies and methodologies for course
delivery.  These are indeed exciting times for the teaching
profession.

% But the challenges remain.  Our education should show the student
% how to combine the abstract with the concrete, and lead them
% to understand the invariants driving a constantly changing
% world.  %
% Can what we teach measure up to the famous
% observation of Swami Vivekananda, ``Education is the
% manifestation of the perfection already in Man.''?  



% To drive this innovation in education, I
% propose that IITH establish an advanced centre for
% instructional technologies and innovative learning, perhaps
% jointly with IGNOU or other national institutes.  Ideally, a
% joint IIT-IGNOU centre could be established at the Hyderabad
% campus.


% We must innovate constantly to
% remain relevant, because the student today has many choices?

\bibliographystyle{plain}

\bibliography{../../biblio/venk,../../biblio/rest}
\end{document}

