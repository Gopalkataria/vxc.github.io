%<---------------------------64 col--------------------------->|
%<---------------------------64 col--------------------------->|
%<---------------------------64 col--------------------------->|

\documentclass[titlepage, %separate page for title
11pt, 
%twoside % page numbers on the outside
]{article}
\usepackage[pdftex,dvipsnames,x11names,svgnames,table,fixpdftex,hyperref]{xcolor}
\usepackage{palatino}
\usepackage[square]{natbib}
\usepackage{array}
\usepackage{colortbl}
\usepackage{longtable}
\usepackage{dcolumn}
\usepackage{epigraph}
\usepackage{multirow}
\usepackage{graphicx}
%%% Comment the line below if you want a regular portrait
%%% style page layout 
% \include{pdfscreen}
\definecolor{lightgrey}{gray}{0.75}
%%% Comment the line below if you want a white background.
% \pagecolor{lightgrey}
\usepackage{palatino}
\usepackage[pdftitle={Venkatesh Choppella's Professional Activities at IIITM-K: Oct 2003 to Feb 2009},%
colorlinks=true,% true
citecolor=blue,
linkcolor=Brown,
urlcolor=Brown,%navy%
]{hyperref}

\hypersetup{
pdfauthor= {Venkatesh Choppella <choppell@iiitmk.ac.in>},
pdfkeywords = {Research, Teaching, Conferences, contributions, IIITM-K, Venkatesh Choppella}
}

%                                  %%% pdfcreator, pdfproducer, 
%                                      and CreatioDate are automatically set
%                                      by pdflatex !!!
\pdfadjustspacing=1                %%% force LaTeX-like character spacing
%

\newcommand{\achievement}[1]{\textcolor{OliveGreen}{\bf #1}}

\begin{document}

\title{Venkatesh Choppella's Professional Activities at
  IIITM-K: Oct~2003 - Feb~2009}
\author{Venkatesh Choppella}
\date{\today}

\maketitle

\begin{abstract}
  This document is a summary of my professional activities
  at IIITM-K from October~2003, when I joined the institute,
  till Feb~2009.
\end{abstract}

\tableofcontents
\newpage
\listoftables
\newpage
\section{Introduction}
\label{sec:introduction}


\epigraph {Intelligence and capability are not enough. There
  must be the joy of doing something beautiful.}{{\sc
    Govindappa Venkataswamy} Founder, Aravind Eye Hospital}

This is a report of my professional activities at IIITM-K
for the period Oct~2003 to Feb~2009 highlighting milestones
and achievements.  The report covers research work
(Section~\ref{sec:research-work}), innovations in teaching
(Section~\ref{sec:teaching}), projects
(Section~\ref{sec:projects}), administrative service
(Section~\ref{sec:admin}), conference attendance and
professional service (Section~\ref{sec:conf}), collaboration
with foreign institutes (Section~\ref{sec:foreign}), and
contributions towards IT innovations
(Section~\ref{sec:it-innovations}).  It concludes with
acknowledgements and observations about the institute
(Section~\ref{sec:concl}).
\section{Research work}
\label{sec:research-work}
At IIITM-K, I have continued research in three related areas
of computer science: programming language type systems,
compilers for scientific computing, and formal methods in
software engineering.  In addition, I have also developed
interests in bioinformatics, systems biology and computer
security.  I have also invested significant effort into IT
development projects.
Tables~\ref{tbl:publications-1}~and~\ref{tbl:publications-2}
summarise published and submitted research work that I have
done since 2003. 

\subsection{Programming language type systems}
\label{subsec:types}
After my arrival at IIITM-K, I have building on my earlier
work in diagnostic first order term unification and its
application to polymorphic type reconstruction.  An extended
version of my earlier work in automatic proof generation for
term unification~\cite{Choppella-Haynes-CADE-2003} was
published as an \achievement{invited paper\/} in {\em
  Information and
  Computation}~\cite{Choppella-Haynes-IAC-2005}.  The main
result of this paper was the development of a practical, but
formal framework for an algorithm for automatic generation
of proofs of term unification.  An application of this
algorithm to Hindley-Milner polymorphic type reconstruction
was presented at the International Workshop on
Implementation of Functional Languages at Edinburgh in 2003
and subsequently published by Springer Lecture Notes in
Computer Science~\cite{Choppella-IFL-2003}.  The paper's
algorithm relied on an interesting compositionality
principle for unification which I elaborated in a separate
report~\cite{Choppella-2004-compositionality}, but addressed
only the non-generic case of Hindley-Milner.  Diagnosing
type errors for polymorphic, higher-order languages is still
an active resarch area and falls under the broader category
of static and dynamic program debugging.  There are several
directions for further work.
%
%
% I am working on
% applying graph-theoretic formalism developed for unification
% to generalized Dyck and Semi-Dyck path
% problems~\cite{bradford-choppella-2008}.  Second, %
%
I working on a framework of path embeddings, which relates
unification graphs of Hindley-Milner polymorphism with the
unification graph of the ``let-expanded'' monomorphic
simply-typed lambda calculus.  Preliminary work in this area
is mostly complete.  Later this year, I plan to extend type
debugging to debugging with contracts.  PLT Scheme has an
extensive contracts framework, and I plan to use it to
develop an implementation.  To collaborate on this work, I
invited Guillaume~Marceau, a PhD scholar from Brown
University and a contributor to the development of PLT
Scheme.  Marceau spent the first semester of Academic~Year
2008-2009 at IIITM-K.  In the long term, I plan to develop
specific frameworks of proof generation for term unification
in equational theories, specially those involving equational
axioms for commutativity and associativity (C, A and AC
unification).

\subsection{Compilers for scientific computing: the Tensor Contraction Engine}
\label{subsec:compilers}

From 2003 to 2006 I continued to collaborate with scientists
at the Oak Ridge National Laboratory (my ex-employer) and
the Ohio State University on the Tensor Contraction Engine
compiler (TCE) for high performance computing.  The Tensor
Contraction Engine is an optimizing compiler that implements
ab initio quantum mechanical computations expressed as
tensor equations into high performance Fortran code on
parallel computers.  The work resulted in the publication of
three invited journal
papers~\cite{Auer-MolPhysics-2006,Baumgartner-ieee-2005,krishnan-et-al-jpdc-2006}
in the Journals {\em Molecular Physics}, {\em IEEE
  Journal\/} and the {\em Journal of Parallel and
  Distributed Computing\/} respectively, and four
international conference and workshop
papers~\cite{Bibireata-et-al-LCPC-2004,Hartono-et-al-iccs-2006,Krishnan-et-al-HiPC-2003,Krishnan-et-al-IPDPS-2004},
the first three published Springer LNCS and the last by IEEE
Computer Society.  The main issues addressed in the research
are operation minimization, loop fusion, data locality
optimization, integration with constraint solvers,
space-time tradeoffs between memory and recomputations, and
optimum memory layout of intermediate arrays.  The paper on
data locality optimizations~\cite{Krishnan-et-al-HiPC-2003}
won the \achievement{best paper\/} award at the
International Conference on High Performance Computing in
2003.  The main data structures used in the Tensor
Contraction Engine compiler are tensor expressions.  These
expressions are represented as first order structures with
variable names embedded in them, which makes their internal
manipulation quite cumbersome.  In work to be soon submitted
for publication, I have proposed a novel variable-free
representation for tensor expressions along with a calculus
of selection and projection
operators~\cite{Choppella-et-al-var-free-tce-in-preparation}.


%%%%%%%%%%%%%%%%%%%  Table Parameters end %%%%%%%%%%%%%%%%%%%%%
% \newcolumntype{r}{>{\columncolor{red!10}}c}
% \newcolumntype{b}{>{\columncolor{blue!10}}l}
% \newcolumntype{g}{>{\columncolor{green!10}}l}
% \newcolumntype{w}{>{\columncolor{white}}l}

\renewcommand{\multirowsetup}{\centering}
\renewcommand{\arrayrulewidth}{1.0pt}  % default 0.4pt
\renewcommand{\arraystretch}{1.5}

%%%%%%%%%%%%%%%%%%%  Table Parameters end %%%%%%%%%%%%%%%%%%%%%

\begin{table}
\begin{minipage}{1.0\linewidth}
\rowcolors%[\hline]
{2}{White}{blue!10}
\setlength\extrarowheight{4pt}
\begin{tabular}%
{|p{0.10\linewidth}|p{0.40\linewidth}|p{0.20\linewidth}|p{0.20\linewidth}|p{0.10\linewidth}|}
\hline
\multicolumn{1}{|m{0.10\linewidth}|}{\centering{\bf Year}}&
\multicolumn{1}{m{0.40\linewidth}|}{\centering {\bf Paper}}&
\multicolumn{1}{m{0.20\linewidth}|}{\centering {\bf Conf. or Journal}}&
\multicolumn{1}{m{0.20\linewidth}|}{\centering {\bf Publisher}}&
\multicolumn{1}{m{0.10\linewidth}|}{\centering {\bf Biblio entry}}\\
\hline
2003 & Source-tracking Unification & Proc. CADE 2003 & Springer LNAI & \cite{Choppella-Haynes-CADE-2003}\\
2003 & Data Locality Optimization for Synthesis of Efficient Out-of-Core Algorithms & Proc. HiPC 2003 & Springer LNCS & \cite{Krishnan-et-al-HiPC-2003}\\
2004 & Polymorphic Type Reconstruction using Type Equations & Proc. IFL 2003 & Springer LNCS & \cite{Choppella-IFL-2003}\\
2004 & A compositionality principle for unification & Unpublished manuscript\footnote{This paper was accepted at the 2004 International Workshop on Unification at Cork Ireland. I was unable to attend the workshop and as a result,  chose to withdraw my paper from the proceedings.} &  & \cite{Choppella-2004-compositionality}\\
2004 & Memory-constrained Data Locality Optimizations for Tensor Contractions & Proc. LCPC 2003 & Springer LNCS & \cite{Bibireata-et-al-LCPC-2004}\\
2004 & Efficient Synthesis of Out-of-core Algorithms Using a Nonlinear Optimization Solver& Proc. IPDPS 2004 & IEEE Computer Press & \cite{Krishnan-et-al-IPDPS-2004}\\
2005 & Source-tracking Unification (Expanded and Revised Version) & Information \& Computation & Elsevier & \cite{Choppella-Haynes-IAC-2005}\\
2005 & Synthesis of high performance
parallel programs for a class of ab initio quantum chemistry
models. & Proc. of the IEEE & IEEE & \cite{Baumgartner-ieee-2005}\\
\hline
\end{tabular}
\end{minipage}
\caption{Papers published  2003 - 2005\label{tbl:publications-1}}
\end{table}

\begin{table}
\begin{minipage}{1.0\linewidth}
\rowcolors%[\hline]
{2}{White}{blue!10}
\setlength\extrarowheight{4pt}
\begin{tabular}%
{|p{0.10\linewidth}|p{0.40\linewidth}|p{0.20\linewidth}|p{0.20\linewidth}|p{0.10\linewidth}|}
\hline
\multicolumn{1}{|m{0.10\linewidth}|}{\centering {\bf Year}}&
\multicolumn{1}{m{0.40\linewidth}|}{\centering {\bf Paper}}&
\multicolumn{1}{m{0.20\linewidth}|}{\centering {\bf Conf. or Journal}}&
\multicolumn{1}{m{0.20\linewidth}|}{\centering {\bf Publisher}}&
\multicolumn{1}{m{0.10\linewidth}|}{\centering {\bf Biblio entry}}\\
\hline
2006 & Automatic code generation for many-body electronic structure methods: the Tensor Contraction Engine & Molecular Physics & Taylor and Francis & \cite{Auer-MolPhysics-2006}\\
2006 &  Efficient Synthesis of out-of-core algorithms using a nonlinear optimization solver & JPDC & Elsevier & \cite{krishnan-et-al-jpdc-2006}\\
2006 & Constructing and Validating Entity-Relationship models in the PVS specification language.& IU Tech. Report  & Indiana University Computer Science Dept. & \cite{choppella-sengupta-robertson-johnson-tr-2006}\\
2006 & Identifying cost-effective common subexpressions to reduce operation count in tensor contraction evaluations& Proc. ICCS 2006 & Springer LNCS & \cite{Hartono-et-al-iccs-2006}\\
2007 & Prelimary Explorations in Specifying and Verifying Entity-Relationship models in PVS& AFM 2007  & ACM & \cite{choppella-et-al-afm07}\\
% 2008 & Fast Dyck and semi-Dyck constrained shortest paths on DAGS & Journal of Computing (submitted) & INFORMS & \cite{bradford-choppella-2008}\\
\hline
\end{tabular}
\end{minipage}
\caption{Papers published Jan~2006 - May~2008\label{tbl:publications-2}}
\end{table}

\subsection{Formal verification of data models in software engineering}
\label{subsec:formal-verification}
In 2006, I started investigating the possibility of
verifying properties of data models using the PVS theorem
prover.  The goal is to apply the idea of theory
interpretations from universal algebra to data refinement.
The work, jointly done with my colleagues at Indiana and
Wright State Universities, resulted in a technical report
and an international workshop paper published by
ACM~\cite{choppella-et-al-afm07,choppella-sengupta-robertson-johnson-tr-2006}.
The work falls into the broader category of Model Driven
Architecture (MDA)-based Software
Engineering~\cite{kleppe-et-al-mda-explained-2003}, which
encourages a model-based, transformation approach to
generating software.  The additional aspect emphasized in
our work is verifiability of models.  The
entity-relationship models in our work were hand transformed
to relational tables.  My work with Alice~Joseph, M.Tech.\@
student at Vellore Institute of Technology, addresses the
problem of automatically generating theories in the PVS
specification language from E-R diagrams~\cite{joseph-2008}.
I am also looking into how the correctness of
transformations across models may be proved with graph
simulation and bisimulation techniques.  Such techniques
are, for example, commonly used in the theory of labeled
transition systems and concurrency.

\subsection{Interest in $\pi$-calculus models for systems biology}
\label{subsec:pi-calculus}
In 2007, I began to cultivate an interest in bio-informatics
and systems biology.  My interests are in applying
programming language calculi, specially the $\pi$-calculus,
and its stochastic variants to study cellular
pathways~\cite{Regev-Shapiro-pi-calulus-modeling-chapter-2004}.



\subsection{Pantoto: Generative Programming for Web Applications}
\label{subsec:pantoto}

Over the last one year, I have been working on refining the
architectural design of Pantoto~\cite{pantoto-google}, an
open source project hosted on Google's open source code
portal. Pantoto is web framework for building information
management solutions for small medium enterprises.  From a
computer science perspective, Pantoto has a remarkably
interesting design: it is a generic framework which allows
creation of Information management applications, not from
scratch, but by defining a set of users and users, and
templates and pagelets.  The goal of Pantoto is to be able
to allow non-IT people to build their own information
systems.  Pantoto is built on top of Python's Django
toolkit.  It's generative approach allows highly flexible
websites to be created in very a matter of a few hours.
Over the last several months, I have started formalizing
data type architecture for Pantoto which is expected to be
the basis for future extensions of Pantoto.  I plan to use
Pantoto as an experimental testbed for exploring verifiable
architectures for E-Governance. 


\section{Course development  and innovations in teaching}
\label{sec:teaching}

The last four years at IIITM-K have given me many excellent
opportunities to design new courses and apply innovative
methods and technologies towards teaching them.  I have
taught or co-taught the following core courses at IIITM-K:
discrete mathematics, web technology, database systems,
principles of programming, data structures and algorithms.
In addition, I have been involved in the teaching of the
following three electives: compilers and interpreters
(principal instructor), computational biology,
(co-instructor), and scientific computing (associate
instructor).  Table~\ref{tbl:courses} summarizes the courses
I taught and assisted at IIITM-K.  Many of the courses saw
active student participation and enthusiasm.  As a result,
student feedback of my courses and teaching style have been
largely positive and quite
insightful~\cite{iiitmk-course-reviews}. (See also
Table~\ref{tbl:feedback} for sample feedback from students.)
Specific contributions in some of these courses is
summarized below.

\subsection{Principles of Programming}
\label{subsec:pop}

In 2006, in collaboration with my colleage Dr.~T.~B.~Dinesh
of Janastu, Bangalore, I proposed a new core course on the
Principles of Programming.  Most of IIITM-K's students enter
the IT industry as programmers or software engineers and
apply their knowledge about programming abstractions into
design, implementation, deployment and maintenance of
software.  Programming thus forms the centre piece of
today's software engineering education.  Until the
introduction of the Principles course, the institute's
introductory programming course employed the object-oriented
programming language Java.  The limitations of this approach
have been well documented by software engineering educators.
Instead, the Principles of Programming course teaches
concepts from foundations.  It uses Scheme and Javascript,
two much simpler languages.  Scheme has a proven track
record as an excellent introductory programming language,
and Javascript, although syntactically more like C and C++,
is an object oriented language semantically closer to Scheme
and other functional languages.  Besides, the use of
Javascript acknowledges and addresses the need to align
modern programming education with the pervasive programming
paradigm of today, namely, interactive programming over the
world wide web.

The approach taken in the course was presented and evoked
interest at several seminars and invited lectures, including
the National Conference of Software Engineering at Cochin in
April~2007, Bar~Camp~Bangalore,~April~2008, NIT~Calicut,
IIT~Guwahati, engineering colleges in Karnataka and Andhra
Pradesh, and in the industry (Satyam Computers, TCS, and
Cognizant).  This course became the first course offered in
open mode at the institute and drew external enrollment from
professionals at Technopark (Trivandrum's IT Park).

\begin{table}
\begin{minipage}{1.0\linewidth}
\rowcolors%[\hline]
{2}{White}{blue!10}
\setlength\extrarowheight{4pt}
\begin{tabular}%
{|c|p{0.40\linewidth}|p{0.30\linewidth}|}
\hline
\multicolumn{1}{|m{0.30\linewidth}|}{\centering {\bf Year}}&
\multicolumn{1}{m{0.40\linewidth}|}{\centering {\bf Course}}&
\multicolumn{1}{m{0.30\linewidth}|}{\centering {\bf Role}}\\
\hline
2009 & Data Structures and Algorithms  &  Principal Instructor\\
2008 & Web Technology  &  Principal Instructor\\
2008 & Computer Networks  &  Associate Instructor\\
2007 & Principles of Programming  &  Principal Instructor\\
2007 & Principles of Programming  &  Principal Instructor\\
2007 & Web Technology  &  Principal Instructor\\
2007 & Computational Biology  &  Co-instructor\\
2006 & Web Technology & Principal Instructor\\
2006 & Data Structures & Guest Lecturer\\
2006 & Scientific Computing  & Associate Instructor\\
2005 & Databases &  Co-Instructor\\
2004 & Discrete Maths & Principal Instructor\\
2004 & Web Technology & Principal Instructor\\
2004 & Interpreters and Compilers & Principal Instructor\\
2003 & Web Technology & Principal Instructor\\
\hline
\end{tabular}
\end{minipage}
\caption{Courses taught at IIITM-K \label{tbl:courses}}
\end{table}


\begin{table}
\begin{minipage}{1.0\linewidth}
\rowcolors%[\hline]
{2}{White}{blue!10}
\setlength\extrarowheight{4pt}
\begin{tabular}%
{|p{0.90\linewidth}|}
\hline
\multicolumn{1}{|m{0.90\linewidth}|}{\centering {\bf Comments}}\\
\hline
Sir is having great knowledge of the domain.\\
Clear, sincere and approachable.\\
Enthusiasm to teach and work.\\
He was successful in explaining the basic concepts clearly in the class.  Also, he was approachable any time.\\
The lectures were exceptionally good and have helped us get a very good understanding ...\\
The emphasis laid upon punctuality and verbal communication ... helped us a lot.\\
Enthusiasm of teaching.  Clear communication.  Detailed reply to students.  Encouraging students.\\
He always applies mathematical logic which I love very much.\\
I got a good grasp of the subject.  Moreover, I find the quizzes and exam questions interesting to solve. \\
Innovative and effective usage of computing resources.\\
I enjoyed every session to the bone.\\
\hline
\end{tabular}
\end{minipage}
\caption{Sample anonymous feedback from students\label{tbl:feedback}}
\end{table}


\subsection{Web Technology}
\label{subsec:webtech}
I have designed and taught the institute's Web Technology
course almost every year since 2003.  It has been informally
acknowledged by students as the most technology-intensive
and fast-paced course in the institute.  The technologies
taught in this course -- HTTP, Java and J2EE, XML, AOP,
Hibernate, PKI, to name a few -- are also essential for
students when they graduate to industry jobs.  Over the
years, the Web Technology course has evolved into a course
where students do team projects or technology surveys.

An important aspect in this course is the emphasis on
conceptual principles behind the plethora of web
technologies.  These principles began to crystallize after
the first few renditions of the course.  Web programming is
inherently trickier than ordinary, sequential programming,
but why?  To understand and appreciate the difference
between web programming and ordinary stand-alone programs, I
adopted an innovative pedagogical approach based on
continuation models of web programming.  This new approach,
which I used in the 2007 offering of the web technology
course, is derived from recent research by the PLT Scheme
group~\cite{Krishnamurthi-et-al-plt-web-server-hosc-2007}.
Since the students had learnt Scheme programming in the
previous semester, they were able to not only understand the
Scheme web server model, but also compare it with the
traditional Java servlet model, and appreciate the various
pitfalls in Web program design.

\subsection{Scientific Computing}
\label{subsec:scicomp}

In 2006, I assisted Prof.~M.~S.~Gopinathan in designing and
teaching the Scientific Computing elective.  An innovation
in the course was the use of the Scilab open source
scientific computing platform.  As part of the course,
students designed and ran computer simulation experiments
modeling many physical, chemical and biological processes.
The course also led to two students doing their final
project in the area of simulating cell
mutation~\cite{sam-soman-2007}.

\subsection{Computational Biology}
\label{subsec:computational-biology}

In 2007, I designed and taught the computational biology
elective course along with Profs.~Gopinathan and
Sundarapandian.  The course attracted faculty from Kerala
University's plant biology department and also a member from
Trivandrum's fledgling biotech industry.  The
interdisciplinary nature of the subject allowed us to
introduce concepts from biology, chemistry, computer
science, statistics and information theory to the study of
bio-informatics and also systems biology.  

\section{Project Funding}
\label{sec:projects}

At IIITM-K, I have been involved in projects in the area of
E-Governance, Software Engineering Education, Information
Security, and Grid computing.  Table~\ref{tbl:projects}
summarizes my initiatives in projects.  Several student
projects related to these projects were completed under my
guidance.  These are listed in
Tables~\ref{tbl:ms-projects-1} and~\ref{tbl:ms-projects-2}.

\subsection{E-Governance}
\label{subsec:egov}
I coordinated the development and deployment of the
Trivandrum City Police project from 2004 to
2005~\cite{tvm-city-portal-2005}.  This project was executed
with the help of the project staff of the institute, and
with the active encouragement of Prof.~K.~R.~Srivathsan and
the then Trivandrum Police Commissioner, Mr.~Rajan~Singh who
jointly conceived and proposed the project in 2004.  This
project won the \achievement{National Manthan Award Silver Prize}
for the Best IT project in the E-Governance category in
2005.

In the last year, I have been working with the University of
Zurich's Centre for Direct Democracy in the area of
Citizen-directed e-democracy platforms.  Two proposals,
including an institutional partnership programme and a
research proposal were submitted to the Indo-Swiss Research
Council in
March~2008~\cite{indo-swiss-jrp-2008,indo-swiss-ipp-2008}.


\subsection{Grid Computing}
\label{subsec:grid}

I have initiated collaboration with the Aeronautical
Development Agency (ADA), Bangalore for the development of a
Grid computing solution for the lab's structural
aerodynamics division~\cite{ada-proposal-2007}.  After
extensive discussions between ADA and IIITM-K, we have
submitted a proposal for Rs.~30 Lakhs, which is currently
under consideration for approval by ADA.

\subsection{Integrated Computer Science and Software Engineering Education}
\label{subsec:icase}
I am the coordinator of the {\em Integrated Computer Science
  and Software Engineering Education\/} (ICASE) project
funded by the institute.  The goal of this project is to
realign software engineering education so that it is based
on sound computer science principles.  The introduction of
the Principles of Programming course is a preliminary step
in this project.  Several student projects have already
contributed to software and courseware development towards
ICASE.  I have been working closely with Dr.~T.~B.~Dinesh,
Bangalore on designing the course material and pedagogical
approach to teaching modern programming.  Several others
within and outside India are interested in the project.  I
have been in touch with Profs.~Shriram~Krishnamurthi and
Kathi~Fisler of Brown University and Worcester Polytechnic
Institute, Dr.~Philipp~Kutter of Montages Software, Zurich,
Switzerland, and Mr.  R.~Narayanan, former Vice-President of
Training at TCS.


\begin{table}
\rowcolors%[\hline]
{2}{White}{blue!10}
\setlength\extrarowheight{4pt}
\begin{tabular}%
{|p{0.15\linewidth}|p{0.40\linewidth}|p{0.25\linewidth}|p{0.20\linewidth}|}
\hline
\multicolumn{1}{|m{0.15\linewidth}|}{\centering {\bf Duration}}&
\multicolumn{1}{m{0.40\linewidth}|}{\centering {\bf Project Title}}&
\multicolumn{1}{m{0.25\linewidth}|}{\centering {\bf Funding Agency}}&
\multicolumn{1}{m{0.20\linewidth}|}{\centering {\bf Amount}}\\
\hline
2005-2006 & Trivandrum City Police Portal for Community Interaction, Phase - II & Kerala State Police Department & Rs.~29 Lakhs\\
2006-2011 & Information Security Education and Awareness & Ministry of IT, Govt. of India & Rs.~6 Lakhs per annum\\
2007-2008 & Data and Computational Grid with Portal Front-end for Aerodynamic Analysis & Aeronautical Development Agency, Minstry of Defence, Bangalore & Rs.~30 Lakhs (pending approval)\\
2008-2009 & Integrated Computer Science and Software Engineering Education & Govt.~of~Kerala& Rs.~12.5 Lakhs\\
2008-2009 & Principles of Programming Course Development Grant & Govt.~of~Kerala& Rs.~5 Lakhs\\
2008-2011 & Direct Democracy at the micro-level in India:  The design and implementation of a civic participation platform & Indo Swiss Joint Research Programme & Rs.~8.64 Lakhs per annum (pending approval)\\
2008-2011 & Citizen Empowerment through ICT's: A new approach academic collaborations.  & Indo Swiss Joint Research Programme & Rs.~9.2 Lakhs per annum (pending approval)\\
\hline
\end{tabular}
\caption{Ongoing projects and proposals submitted for funding \label{tbl:projects}}
\end{table}

\subsection{Information Security Education and Awareness}
\label{subsec:isea}

IIITM-K is one of the participating institutes in the
Ministry of IT's Information Security Education and
Awareness project.  I am the coordinator of this five year
project at IIITM-K.  As part of this project, the institute
has run several workshops in systems administration.  The
grants from this project have resulted in the institute
acquiring a large collection of books in computer and
information security, several desktop machines, a printer
and a dual core HP Proliant server class machine which is
being used for designing security related experiments.  I am
currently working on designing an introductory course on
information and computer security and have initiated
research in information security.  I have interests in
studying two aspects of security: the first is to apply
principles from bio-informatics to study the problem of
intrusion detection.  I have guided one M.Tech. \@ thesis in
this area~\cite{pratap-2007}.  The second area of my
interest is to detect and expose security threats in web
application programs (Javascript) using static analysis and
formal methods techniques.

\begin{table}
\rowcolors%[\hline]
{2}{White}{blue!10}
\setlength\extrarowheight{4pt}
\begin{tabular}%
{|p{0.10\linewidth}|p{0.45\linewidth}|p{0.25\linewidth}|p{0.10\linewidth}|p{0.10\linewidth}|}
\hline
\multicolumn{1}{|m{0.10\linewidth}|}{\centering {\bf Year}}&
\multicolumn{1}{m{0.45\linewidth}|}{\centering {\bf Project Title}}&
\multicolumn{1}{m{0.25\linewidth}|}{\centering {\bf Student}}&
\multicolumn{1}{m{0.10\linewidth}|}{\centering {\bf Degree}}&
\multicolumn{1}{m{0.10\linewidth}|}{\centering {\bf Univ.}}\\
\hline
2004 & Web Services for Agriculture Marketing System & K.~M.~Manu Prathab and, P.~Roopiga & MSc. CS & Kerala Univ.\\
2004 & Implementation of Efficient Algorithms for Unification Debugging &  M.~T.~Chitra  and V.~G.~Suman& MSc. CS & Kerala Univ.\\
2005 & Police Portal for Community Interaction: General Category
Manipulation and Alert System module & Jeena Premkumar  & PGDIT & IIITM-K\\
2005 & Formal Specification Of Police Portal's Complaint Module Using Z And OCL & Sangram Jena and Manish Garg & PGDIT & IIITM-K\\
2006 & WAP push and MMS Applications & Manish Kumar Singh and Ramesh Nakka  & PGDIT & IIITM-K\\
\hline
\end{tabular}
\caption{Student projects guided: 2004 - 2006 \label{tbl:ms-projects-1}}
\end{table}


\begin{table}
\rowcolors%[\hline]
{2}{White}{blue!10}
\setlength\extrarowheight{4pt}
\begin{tabular}%
{|p{0.10\linewidth}|p{0.40\linewidth}|p{0.25\linewidth}|p{0.10\linewidth}|p{0.15\linewidth}|}
\hline
\multicolumn{1}{|m{0.10\linewidth}|}{\centering {\bf Year}}&
\multicolumn{1}{m{0.40\linewidth}|}{\centering {\bf Project Title}}&
\multicolumn{1}{m{0.25\linewidth}|}{\centering {\bf Student}}&
\multicolumn{1}{m{0.10\linewidth}|}{\centering {\bf Degree}}&
\multicolumn{1}{m{0.15\linewidth}|}{\centering {\bf Univ.}}\\
\hline
2007 & Modelling Point Mutations in E-Coli Gene and their effect on
fitness over successive generations. & Nithya Sam and Vimi Soman & MSIT & IIITM-K\\
2007 & IT Support for Palliative care services with an
e-learning platform and an electronic health record. & Cibi Chacko,  K.~Rejoy Bhaskar,  Lekshmi V.~R. and I.~Subha & MSIT & IIITM-K\\
2007 & Course Evaluation System & Prashob M. Das & MCA & CUSAT\\
2007 & Personnel Management System using LDAP & Priyamvada & MCA & CUSAT\\
2007 & Detecting system-breakins by comparing correlated frequency matrices & Anju Pratap & M.Tech & Alagappa University\\
May 2008 (Expected) & Automatic Generation of Formal Specifications of  Entity-Relationship Diagrams & Alice Joseph & M.Tech & Vellore Inst. of Tech.\\
\hline
\end{tabular}
\caption{Student projects guided and currently guiding: Jan~2007 -- May~2008 \label{tbl:ms-projects-2}}
\end{table}


\section{Administrative service}
\label{sec:admin}

\subsection{Placement}
\label{subsec:placement}
I was the institute's placement faculty coordinator during
the academic year 2004-2005.  That year, for the first time,
IIITM-K managed to place students at General Electric and
IBM.  In November 2004, with the help of David Mathews,
design engineer at IIITM-K, I organized the institute's
pavilion at IT.com in Bangalore.

\subsection{Academic initiatives and service}
\label{subsec:computational-science}

In 2004, I drafted the institute's norms on Sponsored
Projects and Industrial Consultancy.  In early 2006, I was
part of the Working Group on Education in Computational
Science jointly convened by IIITM-K and Kerala University
in.  I wrote and maintained the early drafts of this
proposal.  The proposal was reviewed by the Kerala
University syndicate, which decided against approving the
programme as yet.  Meanwhile, however, the IIITM-K is
considering offering on its own a similar programme in Post
Graduate Diploma in Computational Science.

I served as the Convener of the Institute Academic Affairs
Committee from July~2006 to July~2007.  During my tenure as
Convener, the institute saw several innovations: for the
first time, the institute formally announced and opened its
courses to the industry .  The institute's courses, class
and exam schedules were made accessible via the web.  The
concept of research credits was introduced.  Web based
evaluation of courses and instructors was implemented.  The
practice of end of the semester course reports by
instructors was initiated.  I also took charge of completing
the Institute Academic Bulletin before the end of my tenure
as Convener.  The document was completed with the help from
Prof.~Sundarapandian and Mr.~Rajendra~Kumar, the Registrar,
and other faculty.

\section{Conferences and Professional Service}
\label{sec:conf}

\subsection{Conference organization and peer review of
  papers}
\label{subsec:conf-org}

A summary of my conference organization efforts is
summarised in Table~\ref{tbl:conf-organized}.  I was on the
organizing committee of the four nation (India, Brazil,
Venezuela, Italy) conference on Free Software, Free Society
held in Trivandrum in May 2005.  In December 2005, I
organized an IEEE Workshop on XML and Databases at IIITM-K.
I was programme committee member of International Conference
on Advanced Computing and Communication (ADCOMM) 2005.  In
2006, I organized a 2-day workshop on Community Wireless
Mesh Networks.  In March~2007, I organized an informal
workshop on Software Engineering Education.  In December
2007, I was involved in the organization of two
international workshops: Connecting Computer Science to
Domains, held at IIITM-K, and Technology, Governance and
Citizenship, held at IIM Bangalore.  In 2008, I organized a
2 day, hands-on workshop on FreeMap, an international open
source project and movement to compile a free map of the
entire world.  Currently, I am on the programme committee of
the International Conference on Software Engineering
Education and Training (CSEET), to be held in 2009.  A
summary of my previous and current service to the
professional community as a peer reviewer of papers is
summarized in Table~\ref{tbl:reviewer}.

\begin{table}
\begin{minipage}{1.0\linewidth}
\rowcolors%[\hline]
{2}{White}{blue!10}
\setlength\extrarowheight{4pt}
\begin{tabular}%
{|p{0.20\linewidth}|p{0.40\linewidth}|p{0.20\linewidth}|p{0.20\linewidth}|}
\hline
\multicolumn{1}{|m{0.20\linewidth}|}{\centering {\bf Year}}&
\multicolumn{1}{m{0.40\linewidth}|}{\centering {\bf Conference or Workshop}}&
\multicolumn{1}{m{0.20\linewidth}|}{\centering {\bf Location or Publisher}}&
\multicolumn{1}{m{0.20\linewidth}|}{\centering {\bf Role}}\\
\hline
2005 & Free Software, Free Society & Trivandrum & Organizing Committee member\\
2005 & IEEE Workshop on XML and Databases & Trivandrum & Organizer\\
2005 & International Conf. of Advanced Computing and Communication (ADCOMM) & Coimbatore & Programme Committee member and reviewer\\
2006 & 2-day workshop on Community Wireless Mesh Networks & Trivandrum & Organizer\\
2007 & Workshop on Software Science Education & Trivandrum & Organizer\\
2007 & IEEE Workshop on Ruby on Rails & Trivandrum & Organizer\\
2007 & Technology, Governance and Citizenship & Bangalore & Organizer\\
2007 & Connecting Computer Science to Domains & Trivandrum & Organizer\\
2008 & Two-day workshop on FreeMap & Trivandrum & Organizer\\
2009 & International Conference on Software Engineering Education and Training& Hyderabad & Programme Committee member and reviewer\\
2009 & International Conference on Web Intelligent Systems& Chennai & Programme Committee member\\
\hline
\end{tabular}
\end{minipage}
\caption{Conference organization and programme committee participation\label{tbl:conf-organized}}
\end{table}

\begin{table}
\begin{minipage}{1.0\linewidth}
\rowcolors%[\hline]
{2}{White}{blue!10}
\setlength\extrarowheight{4pt}
\begin{tabular}%
{|p{0.20\linewidth}|p{0.40\linewidth}|p{0.20\linewidth}|p{0.20\linewidth}|}
\hline
\multicolumn{1}{|m{0.20\linewidth}|}{\centering {\bf Year}}&
\multicolumn{1}{m{0.40\linewidth}|}{\centering {\bf Conference or Journal}}&
\multicolumn{1}{m{0.20\linewidth}|}{\centering {\bf Location or Publisher}}&
\multicolumn{1}{m{0.20\linewidth}|}{\centering {\bf Role}}\\
\hline
2004 & International Conf. on Parallel and Distributed Computing, Applications and Technologies (PDCAT-04) & Singapore & Reviewer\\
2005 & International Conf. of Advanced Computing and Communication (ADCOMM) & Coimbatore & PC member and reviewer\\
2008 & Journal of Parallel and Distributed Computing & Elsevier & Reviewer\\
2008 & International Journal of Formal Computing & Springer & Reviewer\\
2009 & International Conference on Software Engineering Education and Training& Hyderabad & PC member and reviewer\\
\hline
\end{tabular}
\end{minipage}
\caption{Peer reviewing of conference and journal papers\label{tbl:reviewer}}
\end{table}

\subsection{Conference attendence}
\label{subsec:conf-attendence}
During the last few years, I attended several conferences
and workshops and gave talks at various institutions and
events.  Details are summarized in Tables~\ref{tbl:conf}
through \ref{tbl:talks-2}.  Numerous professional contacts
were made during these trips.  Two people I met during these
trips, Dr.~T.~B.~Dinesh and Prof.S.~Parthasarathy, were
later invited to join the institute faculty as adjunct
associate professor and full professor, respectively.

\begin{table}
\rowcolors%[\hline]
{2}{White}{blue!10}
\setlength\extrarowheight{4pt}
\begin{tabular}%{|l|l|}
{|p{0.25\linewidth}|p{0.75\linewidth}|}
\hline
\multicolumn{1}{|m{0.25\linewidth}|}{\centering {\bf Year}}&
\multicolumn{1}{m{0.75\linewidth}|}{\centering {\bf Meeting}}\\
\hline
Dec. 2003 & Int. Conf. on Logic Programming, Mumbai\\
Dec. 2003 & Eighth Asian Computing Science Conference, Mumbai\\
Dec. 2003 & {\bf Int. Conf. on High Performance Computing (HiPC), Hyderabad}\\
\hline
Dec. 2004 & Int. Conf. on High Performance Computing (HiPC), Bangalore\\
\hline
May 2005 & Workshop on Design Patterns in Parallel Programming, Urbana Champaign, IL USA\\
Dec. 2005 & {\bf Int. Conf. on Adv. Computing and Communication (ADCOMM), Coimbatore}\\
\hline
Aug. 2006 & International GPLv3 Conference, Bangalore\\
Oct. 2006 & Air Jaldi Summit on Community Wireless Mesh Networks , Dharamsala\\
\hline
Mar. 2007 & {\bf Indo-French Workshop on Scilab, IIT Mumbai}\\
Jul. 2007 & Gnu C Compiler workshop, IIT Mumbai\\
Apr. 2007 & {\bf CSI National Conference on S/W Engg, Cochin}\\
\hline
Feb. 2008 & India Software Engg. Conference (ISEC), Hyderabad\\
\hline
Dec. 2008 & Asian Programming Languages Conference (APLAS), IISc Bangalore\\
\hline
Feb. 2009 & International Conference on Software Engineering Education and Training (CSEET), Hyderabad\\
\hline
\end{tabular}
\caption{Workshops and Conferences attended representing IIITM-K.  Boldface entries indicate events where my work was presented.\label{tbl:conf}}
\end{table}

\begin{table}
\rowcolors%[\hline]
{2}{White}{blue!10}
\setlength\extrarowheight{4pt}
\begin{tabular}%{|l|l|}
{|p{0.25\linewidth}|p{0.75\linewidth}|}
\hline
\multicolumn{1}{|m{0.25\linewidth}|}{\centering {\bf Year}}&
\multicolumn{1}{m{0.75\linewidth}|}{\centering {\bf Event, Location}}\\
\hline
Nov  2003 & TCS Corporate Training Centre, Trivandrum\\
Dec  2003 & TCS, Mumbai\\
\hline
July 2004 & Open University of Catalunya, Catalonia, Spain\\
Aug. 2004 & IEEE Workshop on Frontiers in Computing, Trivandrum\\
\hline
Oct  2005 & CS 888 Seminar Course on Advanced Programming Languages, Computer Science, Ohio State University, USA.  Guest Lectures.\\
Dec  2005 & ADCOMM, Coimbatore\\
Dec  2005 & IEEE Kerala workshop on XML and Databases, Trivandrum\\
Jun  2006 & Computational Chemistry Workshop for Teachers, Trivandrum\\
Jul  2006 & IEEE Tutorial on Aspect-Oriented Programming, Trivandrum\\
Aug  2006 & International GPLv3 conference, IIM Bangalore (Panelist)\\
Sep  2006 & National Productivity Council Workshop, Poovar\\
Dec  2006 & Aeronautical Development Authority, Bangalore\\
\hline
\end{tabular}
\caption{Talks and presentations at institutes, industry and conferences: Oct 2003 --  Dec~2006\label{tbl:talks-1}}.
\end{table}

\begin{table}
\rowcolors%[\hline]
{2}{White}{blue!10}
\setlength\extrarowheight{4pt}
\begin{tabular}%{|l|l|}
{|p{0.25\linewidth}|p{0.75\linewidth}|}
\hline
\multicolumn{1}{|m{0.25\linewidth}|}{\centering {\bf Year}}&
\multicolumn{1}{m{0.75\linewidth}|}{\centering {\bf Event, Location}}\\
\hline
Jan  2007 & NPTEL Workshop, Trivandrum (Moderator)\\
Jan  2007 & MACFAST College, Thiruvalla\\
Feb  2007 & National Inst. of Technology, Calicut\\
Mar  2007 & Indo-French Workshop on Scilab\\
Apr  2007 & National S/W Engg. Conference, CUSAT, Cochin\\
July 2007 & Workshop on Computational Chemistry, IIT Madras\\
Aug  2007 & Cognizant Chennai\\
Oct  2007 & Get to Know Seminar, IIITM-K\\
Dec  2007 & Connecting Computer Science to Domains, IIITM-K, Trivandrum\\
\hline
Feb 2008 & Centre for Direct-Democracy, University of Zurich, Switzerland\\
Feb 2008 & Satyam Computers, Hyderabad\\
Feb 2008 & TCS Excellence in Computer Science Workshop on Practical Program Verification, Hyderabad\\
May 2008 & Supercomputing Education and Research Centre , IISc, Bangalore (project review presentation)\\
Apr 2008 & BarCamp, Indian Inst. of Management, Bangalore\\
Jul 2008 & Google R\&D, Bangalore\\
\hline
Jan 2009 & Faith Infotech, Trivandrum\\
Feb 2009 & Centre for Advanced Computing (CDAC), Hyderabad\\
\hline
\end{tabular}
\caption{Talks and presentations at institutes, industry and conferences:  Jan~2007 -- May~2008\label{tbl:talks-2}}.
\end{table}

\section{Building relations capital for IIITM-K}

\subsection{Collaboration with foreign institutes and  universities}
\label{sec:foreign}
Since October 2003, I have made four trips abroad: one to
the Open University of Catalonia, Spain, two to Ohio State
University, and one to the University of Zurich.  The trips
to Ohio State University in May 2005 and March 2006 were
research assignments where I worked with researchers from
Ohio State University and Oak Ridge National Labs on
compilers for high performance computing.

I was invited to the Open University Catalonia in July 2004
to present the institute's Education Grid initiative.  In
February 2008, I was invited by the University of Zurich's
Centre for Direct Democracy to discuss the submission of a
joint proposal between the University of Zurich and IIITM-K
on establishing and institutional partnership and building
an e-platform for the social sciences.  Two proposals have
been submitted to the Indo-Swiss Research Council in
March~2008~\cite{indo-swiss-jrp-2008,indo-swiss-ipp-2008}.

My meeting with Prof.~Sundeep Sahay of the University of
Oslo, Norway in mid 2006 led to IIITM-K signing an MoU and
submitting a joint proposal with the University of Oslo in
the area of health informatics and web technology.

In April 2008, I initiated an exploration of joint research
with the School of Innovation, Design and Engineering at
M\"alardalen University in Sweden.  To take this
collaboration at a formal level, IIITM-K and the
M\"alardalen University have jointly submitted two proposals
for funding by the European Union.  The first proposal
(EVEREST) involves six other universities from EU member
countries, and two other Indian University (Amritha
Vishwavidyapeetham, and XLRI Jamshedpur) propose joint
research and academic initiatives in the areas of software
engineering to fit with M\"alardalen's GSEEM (Global
Software Engineering Masters) programme.  

The second proposal (EURECA) involves nine universities from
EU countries, with M\"alardalen University as lead
applicant, and seven institutes from South Asia, including
IIITM-K and IIT Kanpur from India\cite{eureca-2008}.  The
proposal seeks funding to establish a Eurasian Academic
Mobility Network in research and academics, with emphasis on
Information Technology, Engineering and Management.  The
proposal was successfully funded and IIITM-K is now part of
a consortium of 16 European and South Asian University
participating under the mobility programme.  As IIITM-K
coordinator of the Eureca project, I held the first meeting
of the Eureca project in August 2008.  Under the Eureca
programme six PGDIT students and 3 staff members for the
mobility programme.

\subsection{Invitation and Hosting of Academics and Researchers from India and Abroad}
\label{subsec:invitations}

Over the last five years, I have invited and hosted over 35
academics and researchers from India and abroad.  Some of
these visitors later accepted visiting faculty positions at
the institute, some conducted workshops, many presented the
institute Get to Know (GTK) seminars, and others visited the
institute for brief periods of time.  The details are given
in Tables~\ref{tbl:visitors-2008} to
~\ref{tbl:visitors-2006}.  Many of these visited spawned
research collaborations and other related visits,
benefitting the faculty and students of IIITM-K.  A few
examples of these benefits are cited here.  Brown
University's Dr. Shriram Krishnamurthi visited IIITM-K in
2007. His visit spawned the visit of his student Guillaume
Marceau in the first semester of AY 2008-2009.  Guillaume
taught the Principles of Programming and received
outstanding feedback from students for his efforts.
Dr. Arijith Sengupta's (Wright State University, USA) visit
in 2005 led to joint research work which was published
later~\cite{choppella-et-al-afm07}.  Dr.~Rene Ejury's visit
in February 2008 helped our students learn about server
virtualization and network maintenance.


\begin{table}
\rowcolors%[\hline]
{2}{White}{blue!10}
\setlength\extrarowheight{4pt}
\begin{tabular}%{|l|l|}
{|p{0.6\linewidth}|p{0.6\linewidth}|}
\hline
\multicolumn{1}{|m{0.6\linewidth}|}{\centering {\bf Person and Affiliation}}&
\multicolumn{1}{m{0.6\linewidth}|}{\centering {\bf Event and Date}}\\
\hline
Dr. T.~B.~Dinesh, Servelots Inc. & Adjunct Faculty, 2008\\
%
Guillaume Marceau, Brown University & Visiting Lecturer, 2008\\
%
Prof. S. Parthasarathy & Professor, 2008\\
%
Prof.~Sasi Punnekkat, Malardalen University, Sweden & Eureca Project meeting 2008\\
%
Dr. Rajiv Thottappillil, Uppsala University, Sweden & Eureca Project meeting, 2008\\
%
Dr. Uday Damodaran, XLRI Jamshedpur  & Eureca Project meeting, 2008\\
%
Dr. Krishnashree Achuthan, Amritha University  & Eureca Project meeting, 2008\\
%
Dr. V. Mohan, Honeywell Corporation, USA & Institute Seminar 2008\\
%
Dr. Rene Ejury, Rostock Germany & Visiting Faculty, 2008\\
%
Dr. Fernando Mendez, Univ. of Zurich, Switzerland & Insitute Visitor, 2008\\
%
Mr. Schuyler Erle, USA & FreeMap Workshop, 2008\\
%
Mr. Mikel Maron Erle, USA & FreeMap Workshop, 2008\\
\hline
\end{tabular}
\caption{Visitors  hosted in 2008 at IIITM-K through my initiative.  \label{tbl:visitors-2008}}
\end{table}

\begin{table}
\rowcolors%[\hline]
{2}{White}{blue!10}
\setlength\extrarowheight{4pt}
\begin{tabular}%{|l|l|}
{|p{0.6\linewidth}|p{0.6\linewidth}|}
\hline
\multicolumn{1}{|m{0.6\linewidth}|}{\centering {\bf Person and Affiliation}}&
\multicolumn{1}{m{0.6\linewidth}|}{\centering {\bf Event and Date}}\\
\hline
%
Prof. P. Krishnan, Bond Univ. Australia & Workshop on
Connecting Computer Science to Domains, 2007\\
%
Prof. Shriram Krishnamurthy, Brown Univ. USA & Workshop on
Connecting Computer Science to Domains, 2007\\
%
Prof. Kathi Fisler, Worcestor Polytechnic Inst. USA & Workshop on
Connecting Computer Science to Domains, 2007\\
%
Prof. Bharat Jayaraman, University of Buffalo USA & Workshop on
Connecting Computer Science to Domains, 2007\\
%
Prof. Sanjiva Prasad, IIT Delhi & Workshop on
Connecting Computer Science to Domains, 2007\\
%
Mr. R. Narayanan, Vice President TCS & Workshop on
Connecting Computer Science to Domains, 2007\\
%
Prof. C.~R.~Muthukrishnan, (Retd.) IIT Madras & Workshop on
Connecting Computer Science to Domains, 2007\\
%
Dr. Mathai Joseph, Head, TRDDC Pune & Workshop on
Connecting Computer Science to Domains, 2007\\
%
Dr. Krishna Jayakar, Pennsylvania State Univ., USA & Visiting Faculty 2007\\
%
Dr. Silvano de Gennaro, CERN, Geneva  & Insitute Seminar 2007\\
%
Prof. T.~V.~Prabhakar, IIT Kanpur 2007 & Workshop on Software Education 2007\\
\hline
\end{tabular}
\caption{Visitors hosted in 2007 at IIITM-K through my initiative.
\label{tbl:visitors-2007}}
\end{table}

\begin{table}
\rowcolors%[\hline]
{2}{White}{blue!10}
\setlength\extrarowheight{4pt}
\begin{tabular}%{|l|l|}
{|p{0.6\linewidth}|p{0.6\linewidth}|}
\hline
\multicolumn{1}{|m{0.6\linewidth}|}{\centering {\bf Person and Affiliation}}&
\multicolumn{1}{m{0.6\linewidth}|}{\centering {\bf Event and Date}}\\
\hline
%
Dr.~Kavita Philip, Univ. of California Irvine, USA & Institute Seminar 2006\\
%
Mr.~C.~V.~Radhakrishnan, Focal Image Inc., Trivandrum & Institute  Seminar 2006\\
%
Dr.~R.~Chandrasenan, European Space Agency  Holland  & Insitute Seminar 2006\\
%
Dr.~Madan Thangavelu, Cambridge University & Institute Seminar 2006\\
%
Dr.~Sundeep Sahay, Univ. of Oslo  & Insitute Seminar 2005\\
%
Dr.~Knut Staring, Univ. of Oslo  & Insitute Seminar 2005\\
%
Dr.~R.~K. Bera, IBM  & Insitute Seminars 2005\\
%
Dr.~Arijit Sengupta, Wright State Univ. USA & Workshop on XML and Databases 2005\\
%
Dr.~Srinivas Murthy, IIT Madras & Workshop on XML and Databases 2005\\
%
Dr.~K.~Gopinath, IISc Bangalore  & Insitute Seminars 2004\\
%
Prof.~Christopher~T.~Haynes, Indiana University USA  & Insitute Visitor 2004\\
%
Prof.~K.~Mani Chandy, Caltech USA  & Insitute Seminar 2004\\
%
Prof.~Sukhamaya Kundu, Lousiana State Univ. USA  & Insitute Seminar 2004\\
%
\hline
\end{tabular}
\caption{Visitors hosted in 2006 and earlier at IIITM-K through my initiative.
\label{tbl:visitors-2006}}
\end{table}

\section{IT Engineering Service and  contributions towards IT innovations}
\label{sec:it-innovations}

At the institute, I have been an active, and, sometimes
demanding user of the institute's network and computing
infrastructure.  Since May 2008, I have also been the
institute's part time networks and systems administrator.

My interest and involvement with the running of the
institute's systems and networks is driven by (a) the need
to increase my own productivity at work, (b) share and
increase the productivity of the community of people I work
within the institute and outside, and (c) assume some
responsibility in the functioning of an important
infrastructural component of the institute.  My
contributions towards bringing in innovation in specific
areas of the institute's IT infrastructure and management
are summarized below.

\subsection{Revamping the institute IT network and systems infrastructure}
\label{subsec:network}

In Mar 2008, I was appointed as convener of the Institute
Infrastructure Committee, a committee constituted by the
Director.  This committee has the responsibility to oversee
all aspects of the institute's IT infrastructure.  Even
before this committee was established I invited Dr.~Rene
again to work on fixing some deep seated problems with the
institute's network infrastructure and its management.
Working with Dr.~Rene, I helped bring two important changes
in the management of IT infrastructure at the institute:
first, the direct involvement of students in maintaining
institute network systems, and two, put into place platforms
(wiki and electronic bulletin board) where all discussions
about networks are being documented and archived.  In the
last 3 months, close to a hundred messages have been
archived on the mailing
list~\cite{networks-iiitmk-mailing-list}.  Since May 2008,
when three of our systems staff quite abruptly, I took over
as the main systems administrator of the institute.  As part
of this initiative, I launched a massive effort to redesign
the institute's network from scratch, document the networks
and all the systems and services of the institute and ensure
regular maintenance of network scripts and configuration
files using version control.


\subsection{Bringing Free and Open Source Software to
  IIITM-K}
\label{subsec:foss}

At IIITM-K, I have continuously promoted the use of Free and
Open Source Software (FOSS).  A summary of the tools I
helped introduce into the institute is given in
Table~\ref{tbl:foss}.  My support and enthusiasm in
promoting FOSS is based on my own wonderful experience with
free software, particularly Linux, which I have been using
for over 10 years now, and GNU Emacs, which I first used in
1988.  When I arrived at IIITM-K, I resolved to make the
promotion of FOSS an important side activity at the
institute.  Furthermore, it is my committed belief that as
an academic institution, IIITM-K has a duty to promote the
values of intellectual freedom and openness that are
embodied in free software.  It also has a responsibility to
eschew proprietary software and data formats and the absence
of freedom implicit in them.  When I arrived at IIITM-K,
there was only one user of Linux (the systems administrator,
Mr.~P.~Siddhartha).  With the encouragement of the Director
and Mr.~Siddhartha, I helped launch a Windows to Linux
migration plan in mid 2004.  This plan involved weekly
lectures on FOSS tools (emacs, shell scripting, CVS, Open
Office, etc.).  I also delivered several lectures for
students on open source tools: Subversion version control
system SVN, Python, Javascript and Scheme programming
languages, the Emacs Editor, open ssh, the open source tool
for public key cryptography, and \LaTeX, the open source
document writing tool.  Four years later, {\em all\/} our
current first year PGDIT students, and a few faculty have
adopted Linux as their main operating system.  Also, \LaTeX
is now the standard format in which students at the
institute submit their final project report.  This
remarkable transformation not only had the benefit of
reducing the institute's dependence on expensive,
proprietary Windows licenses, but it also gave students many
more opportunities to experiement with the institute's
computer and network systems.  

I have also been involved in the promotion of free software
in enterprises at the NGO and Government level in Kerala.  I
am member of the board of SPACE, {\em Society for Promotion
  of Alternative Computing and Employment}, an NGO devoted
to the promotion and use of free and open source software
within Kerala.  I am a member of the special committee
appointed by the Secretary Kerala State IT Mission, to
advise the state government on the implementation of {\em
  Public Key Infrastructure for Secure E-governance\/} in
all its offices.

\begin{table}
\rowcolors%[\hline]
{2}{White}{blue!10}
\setlength\extrarowheight{4pt}
\begin{tabular}%{|l|l|}
{|p{0.25\linewidth}|p{0.75\linewidth}|}
\hline
\multicolumn{1}{|m{0.25\linewidth}|}{\centering {\bf Tool Name}}&
\multicolumn{1}{m{0.75\linewidth}|}{\centering {\bf Description}}\\
\hline
CVS    & Concurrent Version system.  A tool for collaborative document development\\
SVN    & Subversion.  A successor to CVS\\
Emacs  & Text-based extensible editor\\
\LaTeX & Tool for typesetting documents\\
ssh    & Secure Shell.  A tool encrypted login access to remote machines\\
gpg    & Gnu Privacy Guard.  A tool for  encryption and decryption and key management\\
Scilab & Programming language and environment for interactive scientific computing\\
Scheme & Language for teaching programming and manipulating XML\\
Python & Interactive language for general-purpose and systems programming\\
Javascript & Interactive language running in the browser\\
Mediawiki & A web-based tool for collaborative development of webpages\\
Google Calendar &  Mashups of Google calendar and RSS feeds with course wiki pages\\
\hline
\end{tabular}
\caption{FOSS tools introduced and promoted in the institute through my initiative.\label{tbl:foss}}.
\end{table}


\subsection{Bringing Wiki-based instructional methodology}
\label{subsec:wiki}

In December 2006, following the October Airjaldi Summit in
Dharamsala, I invited Dr.~Rene~Ejury to the institute to run
a 2-day, hands-on workshop on community wireless mesh
networks.  As part of the workshop, we installed Mediawiki,
the open source wiki software that is famously used by
Wikipedia.  The following month, in January 2007, the
Principles of Programming course that I offered became
\achievement{institute's first wiki-based course}.  Soon
after that, several courses and projects adopted the wiki
model.  The wiki has been a spectacularly useful tool for
sharing information within the institute.  Today, this wiki
not only functions as an important repository of the
institute's various documents, it is being routinely
employed for courses and is being accessed by people outside
the institute as well as outside the country.


\section{Acknowledgements and Observations}
\label{sec:concl}

The last five years have given me the opportunity to
experiment and innovate: in research, in teaching, and in
meeting and collaborating with students, new institutions
and researchers across the world.  For my work at the
institute, I was fortunate to enjoy the support of people at
all levels.  I wish to place on record my gratitude to the
past Director, Prof.~K.~R.~Srivathsan, and to all colleagues
in the faculty, students (past and present), and staff for
their help and support during my time here.

%% This is an exciting time to be at IIITM-K.  There are huge
%% challenges, to be sure.  The instiute's physical coordinates
%% are yet to be fixed, and it needs sustained growth in all
%% directions to attain criticality.  Yet, there is reason to
%% be optimistic and rededicate oneself to the task of building
%% IIITM-K into an institute of excellence.


\vfill

\epigraph {Always do right.  This will gratify some people
and astonish the rest.}  {{\sc Mark Twain} (1835 - 1910)}

\newpage

\bibliographystyle{../biblio/venk}
% \bibliographystyle{plainnat}
\bibliography{../biblio/venk,../biblio/rest}
\end{document}

