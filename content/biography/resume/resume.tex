\documentclass[11pt,margin,line]{res}
\usepackage{palatino}
\usepackage[pdftex]{graphicx}
\usepackage[usenames,dvipsnames]{color}
%% \input{colordefs}
\usepackage[pdftitle={Venkatesh Choppella's CV},%
pdfauthor={Venkatesh Choppella},%
linkcolor=black,%
colorlinks=true,% true
citecolor=black,%blue,%
urlcolor=Brown,%navy%
]{hyperref}


%\pagecolor{lightgoldenrodyellow}
%% \color{Gray}
% see http://www.sci.usq.edu.au/staff/robertsa/LaTeX/usecolour.html
% on how to use the following predefined colors

%% GreenYellow, Yellow, Goldenrod, Dandelion, Apricot, Peach,
%% Melon, YellowOrange, Orange, BurntOrange, Bittersweet,
%% RedOrange, Mahogany, Maroon, BrickRed, Red, OrangeRed,
%% RubineRed, WildStrawberry, Salmon, CarnationPink, Magenta,
%% VioletRed, Rhodamine, Mulberry, RedViolet, Fuchsia, Lavender,
%% Thistle, Orchid, DarkOrchid, Purple, Plum, Violet, RoyalPurple,
%% BlueViolet, Periwinkle, CadetBlue, CornflowerBlue, MidnightBlue,
%% NavyBlue, RoyalBlue, Blue, Cerulean, Cyan, ProcessBlue, SkyBlue,
%% Turquoise, TealBlue, Aquamarine, BlueGreen, Emerald,
%% JungleGreen, SeaGreen, Green, ForestGreen, PineGreen, LimeGreen,
%% YellowGreen, SpringGreen, OliveGreen, RawSienna, Sepia, Brown,
%% Tan, Gray


%% \color{MidnightBlue}

%\pagecolor{snow}
\oddsidemargin -.5in
\evensidemargin -.5in
\textwidth=6.0in
\itemsep=0in
\parsep=0in
\newenvironment{list1}{
  \begin{list}{\ding{113}}{%
      \setlength{\itemsep}{0in}
      \setlength{\parsep}{0in} \setlength{\parskip}{0in}
      \setlength{\topsep}{0in} \setlength{\partopsep}{0in} 
      \setlength{\leftmargin}{0.17in}}}{\end{list}}
\newenvironment{list2}{
  \begin{list}{$\bullet$}{%
      \setlength{\itemsep}{0in}
      \setlength{\parsep}{0in} \setlength{\parskip}{0in}
      \setlength{\topsep}{0in} \setlength{\partopsep}{0in} 
      \setlength{\leftmargin}{0.2in}}}{\end{list}}



% From
% http://notfaq.wordpress.com/2006/10/04/latex-customize-page-numbering/

% Requires total page count to be known in advance.

%% \usepackage{fancyhdr}
%% \pagestyle{fancy}
%% \fancyhf{} % clear all header and footer fields
%% \fancyfoot[R]{\footnotesize Page \thepage\ of 8}
%% \renewcommand{\headrulewidth}{0pt}
%% \renewcommand{\footrulewidth}{0pt}


% page n of m works with the res class only if you make the
% following fix in res.cls:

% comment the \nofiles line.  By commenting it, you ensure
% that the .aux file is generated.

\usepackage{fancyhdr}
\pagestyle{fancy}
\usepackage{lastpage}
\fancyhf{} % clear all header and footer fields
%\fancyhead[RO,LE]{\footnotesize Venkatesh Choppella}
%% \fancyfoot[R]{\footnotesize Venkatesh Choppella}
%% \fancyfoot[R]{\footnotesize Page \thepage\ of \pageref{LastPage}}
\renewcommand{\headrulewidth}{0pt}
\renewcommand{\footrulewidth}{0pt}
\fancyfoot[RO,LE]{\footnotesize  \href{http://www.iiitmk.ac.in/\~choppell}{Venkatesh Choppella}\\ Page \thepage\ of \pageref{LastPage}}
\fancyfoot[LO,RE]{\footnotesize \today}


\begin{document}

\name{Venkatesh Choppella \vspace*{.1in}}

\begin{resume}
\section{\sc Contact Information}
\vspace{.05in}
\begin{tabular}{@{}p{3.5in}p{3in}}
International Institute of Information  Technology  & {\it Voice:}+91 965-274-0281\\
Hyderabad                     &{\it Fax:}    +91-40-6653 1413\\
GachiBowli, Hyderabad        & {\it E-mail:}  choppell@gmail.com\\ 
Andhra Pradesh 500 032, INDIA            & %\hspace{-4em}\url{http://www.iiitmk.ac.in/~choppell}%
\\     
%  {www.iiitmk.ac.in/{\textasciitilde}choppell}
\end{tabular}

%% \begin{tabular}{@{}p{3.5in}p{3in}}
%% Computer Science and Engineering & 
%% {\it Voice:}  +1-614-915-4090 (m) \\
%% Ohio State University                         & {\it Fax:}    +1-614-292-2911 \\
%% 395 Dreese Labs                               & {\it E-mail:}  choppell@gmail.com\\ 
%% Columbus, OH 43210 USA
%%            & {\it WWW:}
%% \href{http://www.iiitmk.ac.in/~choppell}{www.iiitmk.ac.in/{\textasciitilde}choppell}\\
%% \end{tabular}

%% \section{\sc Objective}
%% Teaching, Research and Development in Software Engineering,
%% Information Technology and Computer Science.

\section{\sc Interests}
Software Architectures and technologies, Programming
Languages, Formal specification and verification of systems,
Computer Science Education

\section{\sc Education}
{\bf Indiana University}, Bloomington, Indiana, USA\\
%{\em Department of Computer Science} 
\vspace*{-.1in}
\begin{list1}
\item[] Ph.D. in Computer Science, Aug 2002
\begin{list2}
\vspace*{.05in}
\item Dissertation:  ``Unification Source-tracking with
  Application to Diagnosis of Type Inference'' 
\item Advisor:  Christopher T. Haynes
\end{list2}
\vspace*{.05in}
\item[] M.S., Computer Science,  May 1995,
\end{list1}

{\bf Indian Institute of Technology, Madras}, Chennai, India\\
%{\em Department of Computer Science and Engineering} 
\vspace*{-.1in}
\begin{list1}
\item[] M.Tech., Computer Science and Engineering,  May, 1987
\begin{list2}
\vspace*{.05in}
\item Dissertation:  ``Implementation of ML using the SECD
  machine'' 
\item Advisor: C.~R.~Muthukrishnan
\end{list2}
\end{list1}

{\bf Indian Institute of Technology, Kanpur}, Kanpur, India\\
%{\em Department of Computer Science and Engineering} 
\vspace*{-.1in}
\begin{list1}
\item[] B.Tech., Computer Science and Engineering,  May, 1985. %CPI 7.8/10.0
\end{list1}

\section{\sc Employment}

% To add extra space between roww
% http://www.cs.toronto.edu/~scalosub/latextips.html

%% % Adds a space between the text and the [T]op \hline
\newcommand\T{\rule{0pt}{2.8ex}}

%% % Adds a space between the text and the [B]ottom \hline
%% \newcommand\B{\rule[-1.7ex]{0pt}{0pt}}

\begin{tabular}{lllcrr}
{\em Associate Professor}, IIIT Hyderabad & {\bf
  Jan} & {\bf 2010\/} & {\bf -} &  & \T\\
{\em Visiting Associate Professor}, IIIT Hyderabad & {\bf
  Jul} & {\bf 2009\/} & {\bf -} & {\bf Dec } & {\bf 2009}\T\\
{\em Visiting Associate Professor}, IIIT Bangalore & {\bf
  Jul} & {\bf 2009\/} & {\bf -} & {\bf Apr } & {\bf 2010}\T\\
{\em Visiting Research Associate}, Ohio State University & {\bf
  May} & {\bf 2005\/} & {\bf -} & {\bf May } & {\bf 2006}\T\\
{\em Associate Professor,} Indian Inst. of Info. Tech.\&
  Mgmt Kerala  & {\bf Oct\/} & {\bf 2003\/} & {\bf -} &\T\\
{\em Adjunct Faculty,} Tata
  Consultancy Services  & {\bf Nov\/} &  {\bf 2003\/} & {\bf
  -\/}  & {\bf Nov\/} &   {\bf 2005\/}\T\\ %6mo
{\em Post-Doctoral Researcher}, Oak Ridge National
  Laboratory & {\bf Dec\/} & {\bf 2002\/} & {\bf -\/}  & {\bf Sep\/} &
  {\bf 2003\/}\T\\ %10mo
{\em PhD. Candidate}, Indiana University & {\bf
  Sep\/} & {\bf 2001\/} & {\bf -\/}  & {\bf Jul\/} & {\bf 2002\/}\T\\
{\em  Senior Software Engineer}, Hewlett-Packard Company & 
{\bf Jan\/} &  {\bf 2001\/} & {\bf -\/}  & {\bf Aug\/} & {\bf
  2001\/}\T\\ %8mo
 {\em Research Staff}, Indiana University &
{\bf Mar\/} & {\bf 2000\/} & {\bf -}  & {\bf Dec\/} & {\bf
  2000\/}\T\\%9mo
{\em Senior S/W Engineer}, Xerox Corporation & {\bf
  Sep} & {\bf 1999\/} &  {\bf -}  & {\bf Feb\/} & {\bf 2000\/}\T\\%4mo
{\em PhD. Candidate}, Indiana University & {\bf
  Jan\/} & {\bf 1999\/} &  {\bf -\/}  & {\bf Sep\/} & {\bf 1999\/}\T\\
\end{tabular}
\newpage

\section{\sc Employment Contd.}
\begin{tabular}{lllcrr}
{\em Research Staff}, Derivation Systems & {\bf
  Jun\/} & {\bf 1995\/} &  {\bf -\/}  & {\bf May\/} & {\bf
  1997\/}\T\\ %24mo
{\em Research/Teaching Associate} Indiana University & {\bf Aug}
  & {\bf 1988\/} &  {\bf -\/}  & {\bf Jan\/} & {\bf 1995\/}\T\\ %36mo
{\em S/W Engineer} CMC Ltd & {\bf Mar} & {\bf 1987\/} &  {\bf
  -\/}  & {\bf Aug\/} & {\bf 1988\/}\T%17mo
\end{tabular}


\vspace{3em}

\section{\sc Teaching Experience}

{\bf International Institute of Information Technology, Hyderabad}

\vspace{-.5cm}

{\em Associate Professor} \hfill {\bf Jan 2010 - present}\\
Designing and teaching the following courses in computer
science:

\vspace*{1em}

\begin{list2}
\item Information Technology Workshop  2011, 2012
\item Principles of Programming Languages  2010, 2011, 2012
\item Topics in Programming Languages   2011
\end{list2}



{\bf International Institute of Information Technology, Bangalore}

\vspace{-.5cm}

{\em Adjunct Associate Professor} \hfill {\bf Jul 2009 - Jun 2010}\\
Designing and teaching the following courses in computer
science:

\vspace*{1em}

\begin{list2}
\item Principles of Programming  2009
\item Programming Languages and Models  2010
\end{list2}

{\bf International Institute of Information Technology, Hyderabad}

\vspace{-.5cm}

{\em Adjunct Associate Professor} \hfill {\bf Jul 2009 - Dec 2009}\\
Designing and teaching the following course


\vspace*{1em}


\begin{list2}
\item Principles of Programming  Languages
\end{list2}

\vspace*{1em}

{\bf Indian Institute of Information Technology and Management -- Kerala},

\vspace{-.5cm}

{\em Associate Professor} \hfill {\bf Oct 2003 - present}\\
Designing and teaching the following courses in computer
science and information technology.

\vspace*{1em}

\begin{list2}
\item Principles of Programming  2007, 2008
\item Web Technologies  2003, 2004, 2006, 2007, 2008
\item Interpreters and Compilers 2004.
\item Mathematical Foundations of Information Technology 2004.
\item Introduction to Databases 2005, 2006
\item Data Structures 2009
\item Computer Networks  2008 (Associate Instructor)
\item Scientific Computing 2006 (Associate Instructor)
\item Computational Biology 2007 (Co-Instructor)
\end{list2}

%\vspace{-.1cm}
%\section{\sc Teaching Experience Contd.}

{\bf Indiana University Computer Science},
Bloomington, Indiana, USA

\vspace{-.5cm}

{\em Associate Instructor} \hfill {\bf Jan 1990 - Jan 1995}\\
Graded assignments, conducted discussions, and
shared lecturing responsibilities for the following courses:

%% undergraduate courses in
%% programming languages, data structures, and algorithms, and
%% graduate courses in programming language semantics.  Courses
%% assisted. 

\vspace*{.05in}  
\begin{list2}
\item Introduction to Programming with C and Scheme
\item Introduction to Data Structures
\item Introduction to Algorithms
\item Introduction to Programming Languages 
\item Semantics of Programming Languages
\end{list2}

%\vspace{-.1cm}

%\vspace{-.1cm}

\newpage

\section{\sc Engineering Experience}

{\bf IIITM-K}, Trivandrum, India

\vspace{-.5cm}

{\em Part-time Networks and Systems Administrator} \hfill
{\bf May 2008 - Present}\\ 
Designed and continue to maintain institute network, systems
and services, including institute Firewall, DNS, SSH, SVN,
File server, Web Server and Pound reverse-proxy.  Supervised
deployment of other services: Moodle, DHCP, OTRS, Wiki.
Instituted processes for managing the networks group:
version control of network and system scripts using SVN,
using mailing lists and regular meetings of systems staff,
documentation of institute systems on the institute Wiki.
Wrote programs in the Python and BASH scipting languages to
automate change management in network, system and service
configurations.  Liaised with Internet Service Providers
(ISP)'s.  Negotiated with RailTel India and ERNET India on
installation of network equipment.  Oversaw the institute's
switch from Reliance to Asianet as the institute's ISP's.
Taught many classes introducing SSH, LaTeX, SVN, Emacs, and
other Free and Open Source Software (FOSS) tools at IIITM-K
and other colleges from Oct 2003 to present.

{\bf Hewlett-Packard Company}, Cupertino, CA USA

\vspace{-.5cm}

{\em Senior S/W Engineer} \hfill {\bf Jan 2001 - Aug 2001}\\
Worked in the the E-speak product development project.
Implemented and participated in Java XML standards bodies
(mainly JAXB).  Implemented EJB-based applications using the
HP-Bluestone application server.  Coauthored proposal to the
World Wide Web consortium on WSCL: Web Services Conversation
Language.

{\bf Xerox Corporation}, Palo Alto, CA USA

\vspace{-.5cm}

{\em Senior S/W Engineer} \hfill {\bf Sep 1999 - Feb 2000}\\
Worked in the Document Portals Business Group at Xerox on the
Flowport server.  Designed and implemented an XML-based Java API
for remote forms management.  Briefly participated in the Xforms
subgroup of the W3C HTML working group.

{\bf Derivation Systems}, Carlsbad, CA USA

\vspace{-.5cm}

{\em Research Engineer} \hfill {\bf Jun 1995 - May 1997}\\ 
Part of a three member team that designed, developed and
verified Derivational Reasoning System, the company's
flagship product and an interactive environment for formally
transforming high level functional specifications of digital
hardware to gate-level descriptions.
Designed and implemented an ML-style
polymorphic type inference system that also provided support for
type coercions.  Developed a mathematical formalism for the
semantics of DRS and the core program transformations that it
was based on.
Mechanically verified the specification and the
correctness of these transformations in the PVS theorem prover.
Other duties included writing government and NASA grant
proposals.  One of the three authors of the Phase-II NASA-SBIR
proposal that initially funded Derivation Systems.

%\section{\sc Engineering Experience contd.}

{\bf CMC Ltd.}, Secunderabad, INDIA

\vspace{-.5cm}

{\em Research and Development Engineer} \hfill {\bf Jan 1987 -
Aug 1988}\\ 
Implemented the memory bank allocation subsystem of
a Vectorized Fortran Compiler.


\newpage

\section{\sc Research Experience}

{\bf International Insititutes of Information Technology,
  Hyderabad and Bangalore}

\vspace{-.5cm}

Exploring model based architectures.  Working on process
approaches in E-governance, and exploring applications of
process models to other areas.  Researching innovative
Web Technologies for education.  

{\bf Indian Institutes of Information Technology and
  Management -- Kerala} 

\vspace{-.5cm}

Worked with and guided students in the area of web
technologies and portal development.  Continued research in
unification theory, and investigated the application of
formal methods to data bases.


{\bf Ohio State University}, Columbus, OH, USA

\vspace{-.5cm}

{\em Visiting Research Associate\/} \hfill {\bf May 2005 - May 2006}\\
Worked with Prof. P. Sadayappan in the Tensor Contraction
Engine (TCE) project.  TCE is a compiler that generates
optimized scientific computing algorithms from high-level
specifications.  Worked on operation minimization, tiling,
and higher-order notation for canonical forms.

{\bf Oak Ridge National Laboratory}, Oak Ridge, TN, USA

\vspace{-.5cm}

{\em Post-Doctoral Researcher\/} \hfill {\bf Dec 2002 - Sep
  2003}\\
Did research work supervised by Dr.~David~Bernholdt in the
Tensor Contraction Engine (TCE) project.  TCE is a compiler
that generates optimized scientific computing algorithms
from high-level specifications.  Implemented and extended
TCE compiler to interface with existing quantum chemistry
packages.

%% Also did joint work with Trey~White
%% towards building a monadic embedded meta-programming environment
%% for the generation of Fortran programs for high-performance
%% scientific computing.

{\bf Indiana University}, Bloomington, IN, USA

\vspace{-.5cm}

{\em Research Staff\/} \hfill {\bf Feb 2000 - Dec 2000}\\ Worked
in the CCAT Project in the Computer Science Department.  Under
supervision of Prof.~Dennis~Gannon, designed component-based
architectures for high-performance distributed systems.  With
other members in the lab, developed a Simple Object Access
Protocol (SOAP) based implementation of Java's Remote Method
Invocation  Application Programmer Interface. 

{\bf Xerox Palo Alto Research Center}, Palo Alto, CA, USA

\vspace{-.5cm}

{\em Research Staff\/} \hfill {\bf Aug 1997 - Dec 1998}\\ 
Worked with the Aspect-Oriented Programming group in the
development of AspectJ, an aspect-oriented extension to
Java.  Implemented the parser and AST module and parts of the
AspectJ weaver, and managed early releases of AspectJ to the
user community.

{\bf Indiana University}, Bloomington, IN, USA

\vspace{-.5cm}

{\em Research Associate\/} \hfill {\bf Aug 1988 - Dec 1994}\\
 Member of the Indiana programming language research group.
 Under supervision of Prof.~Chris~Haynes, designed and
 implemented Infer, an extension of Scheme with static type
 inference primarily used as an early test-bed for research in
 diagnostic unification and type reconstruction of
 variable-arity procedures.  Earlier work included implementing
 Conlog, a Scheme embedding of Prolog that supported first-class
 continuations.


\newpage

\section{\sc Refereed Workshop and Conference Publications}

Venkatesh Choppella, Hitesh Kumar, P Manjula and K
Viswanath.  {\bf From high-school algebra to computing
  through functional programming}.  4th IEEE International
Conference on Technology for Education, Hyderabad, Jul 2012.

Ankur Goel and Venkatesh Choppella.  {\bf Algebraic
  Modelling of Educational Workflows}.  4th IEEE
International Conference on Technology for Education,
Hyderabad, Jul 2012.

Sankalp Khare, Ishan Misra and Venkatesh Choppella.  {\bf
  Using org-mode and SVN for Managing nd Publishing Content
  in Computer Science courses}.
4th IEEE International Conference on Technology for
Education, Hyderabad, Jul 2012.

Shreya Malani, G N Srinivasa Prasanna Jesus A del Alamo,
James L Hardison,Kannan Moudgalya and Venkatesh Choppella
(IIITH). {\bf Issues faced in a Remote Instrumentation
  Laboratory.}  4th IEEE International Conference on
Technology for Education, Hyderabad, Jul 2012.

T B Dinesh, Venkatesh Choppella.  {\bf Alipi -- tools for a
  Re-narration Web.}  9th International Cross-Disciplinary
Conference on Web Accessibility, 16/17th April 2012, Lyon,
France.  Microsoft accessibility challenge: {\bf delegates
  award.}

T B Dinesh, Suzan Uskudarli, Subramanya Sastry, Deepti
Aggarwal, Venkatesh Choppella.  {\bf Alipi: A framework for
  re-narrating web pages.}  9th International
Cross-Disciplinary Conference on Web Accessibility, 16/17th
April 2012, Lyon, France.

Chaitanya Bandi, Aditya K Nori, Venkatesh Choppella and
Sandhya Kode. {\bf A Virtual Laboratory for teaching Linux
  on the web.}  3rd IEEE International Conference on
Technology for Education, Chennai, Jul 2011.

Rohit Khot and Venkatesh Choppella.  {\bf 'DISCOVIR: A
  Framework for Designing Interfaces and Structuring Content
  for Virtual Labs.}  3rd IEEE International Conference on
Technology for Education, Chennai, Jul 2011.

Venkatesh Choppella , Vamsi Krishna Brahmajosyula , Medhamsh
Vutpala and Sukant Kole.  {\bf Process Models for Virtual
  Lab Development, Deployment and Distribution.} 3rd IEEE
International Conference on Technology for Education,
Chennai, Jul 2011.

Vamsikrishna Brahmajosyula and Venkatesh Choppella.  {\bf
  Designing and Programming with State Variables}.  2nd
India Workshop on Advances in Model based Software
Engineering (WAMBSE) 2011.

Thulasi Ram Naidu P., Manisha Verma, Venkatesh Choppella and
Gangadhar Chalapaka.  {\bf Synthesizing Learning
  Environments}.  2nd IEEE Conference on Technology for
Education, Mumbai, Jul 2010.

T B Dinesh and Venkatesh Choppella.  {\bf A case for
  process-driven models for e-governance architectures}.
7th International Conference on E-Governance.  22 - 24
April, 2010, Bangalore, India.

Venkatesh Choppella and K~R~Srivathsan.  {\bf Fostering
  Community Interaction with the Trivandrum City Police
  Portal.}  3rd ACM International Conference on the Theory
and Practice of E-Governance, Bogota, Colombia, pp 365--368,
Nov 2009.

V.~Choppella, A.~Sengupta, E.~L. Robertson and
S.~D.~Johnson.  {\bf Preliminary Explorations in Specifying
  and Validating Entity-Relationship Models in PVS.}
Proceedings of AFM~07, ACM Workshop on Applied Formal
Methods, Nov.~6th,~2007, Atlanta, USA.  Preprint available
at \url{http://fm.csl.sri.com/AFM07/afm07-preprint.pdf}.

Albert Hartono, Qingda Lu, Xiaoyang Gao, Sriram
Krishnamoorthy, Marcel Nooijen, Gerald Baumgartner,
Venkatesh Choppella, David Bernholdt, Russell Pitzer,
J. Ramanujam, Atanas Rountev, and P. Sadayappan.  {\bf
Identifying Cost-Effective Common Subexpressions to Reduce
Operation Count in Tensor Contraction Evaluations.}
International Conference on Computational Science (ICCS'06),
LNCS 3991, pages 267-275, May 2006.  Copyright 2006
Springer-Verlag.

S.~Krishnan, S.~Krishnamoorthy, G.~Baumgartner, C-C.~Lam,
J.~Ramanujam, P.~Sadayappan and V.~Choppella.  {\bf
Efficient Synthesis of Out-of-core Algorithms Using a
Nonlinear Optimization Solver.}  In {\em
{Proc. International Parallel and Distributed Processing
Symposium, Albuquerque, New Mexico, USA, Apr~2004}}.
(Proceedings available on CD-ROM).  IEEE Computer Society.

A.~Bibireata, S.~Krishnan, G.~Baumgartner, D.~Cociorva,
C-C.~Lam, P.~Sadayappan, J.~Ramanujam, D.~E.~Bernholdt and
V.~Choppella.  {\bf Memory-constrained Data Locality
Optimizations for Tensor Contractions.}  In {\em {Proc. 16th
International Workshop on Languages and Compilers for
Parallel Computing (LCPC '03), College Station, Texas, Oct
2003}}.  Springer Lecture Notes in Computer Science
Vol. 2958, pp.~93--108, 2004.

V. Choppella.  {\bf Polymorphic Type Reconstruction Using
Type Equations.}  In {\em {Implementation of Functional
Languages 15th International Workshop, IFL 2003, Edinburgh,
UK, Sep~2003 Revised Papers}}.  Springer Lecture Notes in
Computer Science, Vol. 3145, pp.~53--68, 2004.

S. Krishnan, S. Krishnamoorthy, G. Baumgartner, D. Cociorva,
C. Lam, P.  Sadayappan, J. Ramanujam, D. E. Bernholdt, and
V. Choppella.  {\bf Data Locality Optimization for Synthesis
of Efficient Out-of-Core Algorithms}. In {\em {Proc. of the
Intl. Conf. on High Performance Computing, Hyderabad, India,
Dec~2003}}.  Springer Lecture Notes in Computer Science,
Vol.~2913 pp.~406--417, 2003.

\section{\sc Refereed Workshop and Conference Publications Contd.}

V.~Choppella and C.~T. Haynes. {\bf Source-tracking
  Unification}.  In {\em {Proceedings of 19th International
    Conference on Automated Deduction, (CADE-19), Miami
    Beach, USA, Aug~2003}} Springer Lecture Notes in
Artificial Intelligence, Vol. 2741, pp.~458--472, 2003.

M. Govindaraju, A. Slomenski, V. Choppella, R. Bramley and
  D. Gannon.  {\bf Requirements for and Evaluation of RMI
  Protocols for Scientific Computing}. In {\em {Proc. of the
  IEEE/ACM conference on SuperComputing (SC 2000), Nov
  2000}}.

B. Bose. M. Esen Tuna, and V. Choppella.  {\bf Tutorial on
Digital Design Derivation with DRS}.  In {\em {Proc. 1st
International Conf. on Formal Methods in Computer Aided
Design, (FMCAD '96), Palo Alto, CA, USA, Nov 1996}},
Springer Lecture Notes in Computer Science, Vol~1166, 1996.

P. Bradford, V. Choppella, and G. J. E. Rawlins.  %
%\url{http://dx.doi.org/10.1007/3-540-59175-3_85}
{\bf Lower
  Bounds on the Matrix Chain Order problem: Extended
  Abstract}.  In {\em {Proc. 2nd Latin American Symposium on
    Theoretical Informatics, (LATIN'95), Valparaiso, Apr
    1995}}.  Springer Verlag Lecture Notes in Computer
Science Vol.~911, 1995.

\section{\sc Refereed\\ Journal\\ Publications}


Vamsikrishna Brahmajosyula and Venkatesh Choppella.  {\bf
  Designing and Programming with State Variables}.  Setlab
Briefings.  Vol 9, No 4. 2011 pp 3-10.


S.~Krishnan, S.~Krishnamoorthy, G.~Baumgartner, C-C.~Lam,
J.~Ramanujam, P.~Sadayappan and V.~Choppella.  {\bf
  Efficient Synthesis of Out-of-core Algorithms Using a
  Nonlinear Optimization Solver (Revised and Extended
  Version).}  {\em {Journal of Parallel and Distributed
    Computing}} Vol. 66. pp.~659-673, May~2006.  {\bf
  Invited submission}.

Alexander A. Auer, Gerald Baumgartner, David E. Bernholdt,
Alina Bibireata, Venkatesh Choppella, Daniel Cociorva,
Xiaoyang Gao, Robert Harrison, Sriram Krishnamoorthy,
Sandhya Krishnan, Chi-Chung Lam, Qingda Lu, Marcel Nooijen,
Russell Pitzer, J. Ramanujam, P. Sadayappan and Alexander
Sibiryakov.  {\bf Automatic code generation for many-body
electronic structure methods: the tensor contraction
engine}.  {\em Molecular Physics.} Vol.~104,~No.~2,
pp.~211-228, January 2006.

V. Choppella and C.T. Haynes.  {\bf Source-tracking
Unification (Revised and Extended Version)}.  {\em
{Information and Computation}} Vol.~201 Issue~2,
pp.~121-159, Sep~2005.  {\bf Invited submission}.

G. Baumgartner, A. Auer, D. Bernholdt, A. Bibireata,
V. Choppella, D. Cociorva, X. Gao, R. Harrison, S. Hirata,
S. Krishnamoorthy, S. Krishnan, C. Lam, M. Noojien,
R. Pitzer, J. Ramanujam, P. Sadayappan, A. Sibiryakov.  {\bf
Synthesis of High Performance Parallel Programs for a Class
of ab initio Quantum Chemistry Models}.  {\em {Proceedings
of the IEEE: Special Issue on Domain-Specific Program
Generation and Optimization}}, Vol.~93, No.~2, pp.~276--292,
February 2005.  {\bf Invited submission}.

K.~Rath, V.~Choppella and S.~D.~Johnson.  {\bf Decomposition
of Sequential Behavior using Interface Specification and
Complementation}.  In {\em {VLSI Design, Special Issue on
Decomposition}}, Vol. 3, Nos.~3-4, pp.~347--358, 1995.

% \section{\sc Journal and Conference Submissions}

% Philip Bradford and Venkatesh Choppella.  Fast Semi-Dyck
% Constrained Shortest Paths.  Submitted for review. 

\vspace{2em}

%% \section{\sc Papers in preparation}
%% % J. B. White, V. Choppella, D. E. Bernholdt.  Monadic synthesis
%% % of the NAS Multi-grid Benchmark.

%% Venkatesh Choppella.  {\bf Variable-free canonical forms for
%% Tensor Expressions}.

%% Venkatesh Choppella and Guillaume Marceau.  {\bf Graph Embeddings
%% for diagnosing Hindley-Milner type reconstruction}.


\vspace{2em}

\section{\sc Books in preparation}

An Introduction to the Principles of Programming.  With
T~B~Dinesh.

\vspace{2em}

\section{\sc Technical Reports}

V. Choppella, A. Sengupta, E. L. Robertson, S. D. Johnson.
Constructing and Validating Entity-Relationship Data Models
in the PVS Specification Language: A case study using a
text-book example.  Indiana University Computer Science
Technical Report TR632.  April 2006.

V. Choppella.  {A Compositionality Principle for Syntactic
Unification.}  July 2004. {\em Unpublished manuscript}.  

Venkatesh Choppella. Unification Source-tracking with
Application to Diagnosis of Type Inference.
(PhD. dissertation) {\em {Indiana University Computer
Science Tech. Report. \#566, Aug 2002}}.

A. Banerji, C. Bartolino, D. Beringer, V. Choppella,
K. Govindarajan, A. Karp, H. Kuno, M. Lemon, G. Pogossiants,
S. Sharma, S. Williams.  Web Services Conversation Language
(WSCL) 1.0 Hewlett-Packard Company, {\em {World Wide Web
Consortium Note, Mar 2002}}. \url{http://www.w3.org/TR/wscl10}

B. Bose, M. Esen Tuna, V. Choppella.  Derivational Reasoning
System: A Digital Design Derivation System for Hardware
Synthesis.  {\em {Contract NAS1-20414 National Aeronautics and
 Space Administration (NASA) SBIR Phase II Final
Report, Mar 1997}}.

V. Choppella and C. T. Haynes.  Diagnosis of Ill-Typed Programs,
{\em {Indiana University Tech. Report. \#426, Dec 1994}}.

Venkatesh Choppella.  Implementation of ML using the SECD
machine.  M.Tech thesis IIT Madras. 1987. 

\section{\sc Thesis Supervision}

{\em Automatic generation and deployment of Network
  Configuration scripts}.
Gangadhar~Chalapaka. Post-Graduate Diploma in Information
Technology (PGDIT), IIITM-K 2009 (Expected).

{\em Rebuilding the Institute Network}.  V.~Asha~ Rose,
Girish~N~Gopal, Mithu~Mary~Kuruvilla.  MSIT, IIITM-K 2008.

{\em Building Instructional Technologies and Content for the
  Course Principles of Programming Course}.  Girish~N~Gopal,
Mithu~Mary~Kuruvilla.  MSIT, IIITM-K 2008. 

{\em Automatic Generation of Formal Specification of
  Entity-Relationship Diagrams}.  Alice~Joseph.  M.Tech
Vellore Institute of Technology 2008.  External Supervisor.

{\em Modeling Point Mutations in E-Coli Gene and their
  effect on fitness over successive generations}.
Nithya~Sam and Vimi~Soman.  Masters in Information
Technology (MSIT), IIITM-K 2007.  


\newpage

\section{\sc Thesis Supervision Contd.}

{\em IT support for Palliative Care services with an
  e-learning platform and an electronic health record.}
Cibi Chacko, K.~Rejoy Bhaskar, V.~R.~Lekshmi and I.~Subha.
MSIT, IIITM-K 2007.

{\em Detecting system-breakins by comparing correlated
  frequency matrices}.  Anju~Pratap.  M.Tech. Computer
Science, Alagappa University 2007.  External Supervisor.

{\em Police Portal for Community Interaction: General
  Category Manipulation and Alert System}.  Jeena Premkumar.
Post Graduate Diploma in Information Technology (PGDIT),
IIITM-K 2005.

{\em Formal Specification of Police Portal's Complaint
  Module using Z and OCL}.  Sangram~Jena and Manish~Garg.
PGDIT, IIITM-K 2005. 

{\emph{Web Services for Agricultural Marketing System}}.
K.~M.~Manu~Prathab and P.~Roopiga.  MSc Computer Science,
Kerala University 2004.

{\em Implementation of Efficient Algorithms for Unification
  Debugging}.  M.~T.~Chitra and V.~G.~Suman.  MSc. Computer
Science, Kerala University 2004.

\vspace{2em}

\section{\sc Talks}

IT Management Infrastructure in Colleges: The FOSS
alternative. {\em EnhanceEdu Open Day, IIIT Hyderabad.  April 20th 2010}.


Documentation with Restructured Text:. {\em IIT-Bombay IIIT
  Hyderabad workshop on Scientific Computing with Python
  (SciPy) February 9th 2010}.


FOSS Productivity tools: An introduction. {\em Workshop at IIIT
Hyderabad , December 23rd, 2009}.


Teaching Principles of Programming in the Web 2.0 Era. {\em
  Seminar at IIIT Bangalore, Jun 22nd, 2009}.

Teaching Principles of Programming in the Web 2.0 Era. {\em
  Seminar at IIIT Hyderabad, May 28th, 2009}.

Teaching Principles of Programming in the Web 2.0 Era. {\em
  Seminar at the Centre for Distance Engineering Education
  Programme, IIT Bombay, Jun 11th, 2009}.

{\em Competency Development - Sharing of International Best
  Practices}.  Technopark, Trivandrum, Mar~9th,~2009.
Moderator, Panel Discussion.

A Software Engineering Approach to Network Security.  {\em
  Centre for Devolopment of Advanced Computing}, Hyderabad,
Feb 19th, 2009.

\newpage

\section{\sc Talks Contd.}

Bringing Rigour in Teaching IT. {\em Workshop on IT
  education at Faith Infotech, Trivandrum, January 22,
  2009}.

Reiventing Programming Education in the Web 2.0 Era. {\em
  Seminar at Google R\&D, Bangalore, July 21st, 2008}.

Lexical Scope, Closures and Objects in Javascript. {\em
  Lecture at Bar Camp Bangalore, Apr 20th, 2008}.

Reiventing Programming Education in the Web 2.0 Era. {\em
  Seminar at the Computer Science and Engineering
  Department, National Institute of Technology, Calicut, Apr
  15, 2008}.

Teaching the Principles of Programming: A foundational and
applicative approach. {\em Presentation to Computer Science
  Faculty from Malaysia at Satyam Computers, Feb 19, 2008}.

Introduction to theorem proving with PVS. {\em TCS
  Excellence in Computer Science Workshop Feb 14th-18th,
  2008, Hyderabad, India.}  {\bf Invited Tutorial.}

Information and Communication Technologies for Community
Instrumentation and Direct Democracy: A Kerala perspective.
Informal presentation.  {\em Centre for Direct Democracy,
  University of Zurich, Aarau, Switzerland.  Feb 5th,~2008.}

Teaching the Principles of Programming: A foundational and
applicative Approach. {\em Computer Society of India
  National Conference on Software Engineering, Cochin
  University of Science and Engineering, Cochin, India.  Apr
  27th, 2007}.  Invited Talk.

Automatic Proof Generation for Syntactic Unification.  {\em
  Programming Languages and Verification research group,
  Computer Science and Engineering, IIT Bombay, India.
  Mar~9th,~2007}.

Scilab and Functional Programming.  {\em Indo-French
  Workshop on the Scilab Scientific Computing Language IIT
  Bombay, India.  Mar~8th,~2007}.  Invited Presentation.

IT Tools, Applications and Information Security.  {\em
National Productivity Council Workshop for Non-IT
Executives, Poovar Island, Kerala, India.  Sep~30th, 2006}.


One-day tutorial on Aspect-Oriented Programming and JML.
{\em IEEE Kerala Frontiers in Computing Practice Series,
Trivandrum, India.  July 26th, 2006}.  With Satish Babu.

Panelist, ``Free Software in Education'' track, {\em 3rd
International Conference on GPL v3.  Bangalore, India, Aug
23-24th, 2006}.

Introduction to Aspect-Oriented Design and Programming.
{\bf Invited Tutorial}.  {\em Thirteenth International
Conference on Advanced Computing and Communications,
Coimbatore, India, Dec~14th,~2005}.  With Anurag Mendhekar.

\newpage

\section{\sc Talks contd.}

Constructing and Validating Data Models using PVS {\em
IIITM-K/IEEE Kerala Workshop on XML and Databases,
Trivandrum, India, Dec~10th,~2005}.

Scheme and the Essence of Programming -- Part-I {\em IIITM-K
Get to Know Seminar, Trivandrum, India, Nov~17th,~2005}.

Introduction to Python Programming.  Invited Lecture.  {\em
Frontiers in Computing, IEEE Computer Society Kerala
Chapter, Trivandrum, India, Aug 2004}.

Virtual Learning Campus for the Kerala Education Grid.
Invited Lecture, {\em Open University of Catalunya,
Barcelona, Spain, July 2004}.

A taste of Formal Methods.  Guest Lecture, {\em {TCS Corporate
Training Centre,  Trivandrum,  India, Nov 2003}}. 


Polymorphic Type Reconstruction using Type
Equations.   {\em {15th  International
Workshop on the Implementation of Functional Languages (IFL
2003), Edinburgh, UK, Sep 2003}}.

Source-tracking Unification.  {\em {19th  International
Conf. on Automated Deduction, Miami, FL, USA, Aug 2003}}.

Aspect-Oriented Programming.  Colloquium at the {\em
{Computer Science Department of the State University of New
York at Buffalo, Apr 2002}}.

Diagnostic Unification and Type Inference.  Colloquium at
the {\em {Computer Science Department, State University of
New York at Stony Brook, USA, Apr 2002}}.

Aspect-Oriented Programming (Poster).  {\em {DARPA Demo
Days, Baltimore Maryland, USA, July 1998}}.  With Eric
Hilsdale and Taher Haveliwala.

Diagnosis of Ill-Typed Programs. {\em {ACM Midwest
Programming Language conference, Iowa City, Iowa, USA, April
1994}}.


\section{\sc Workshops and Tutorials}


Venkatesh Choppella and T B Dinesh.  {\bf FOSS, Web 2.0 and
  Mashups as a Natural Learning Management Infrastructure}.
Tutorial proposal.  {\em 2nd IEEE Conference on Technology
  for Education, Mumbai, Jul 2010}.  Accepted for
presentation.

Principles of Programming for Web 2.0.  2 week Summer School
at IIIT Hyderabad for college teachers.  May 3rd--14th 2010.
IIIT Hyderabad.  With T B Dinesh.

Introduction to Aspect-Oriented Design and Programming.
{\bf Invited Tutorial}.  {\em Thirteenth International
  Conference on Advanced Computing and Communications,
  Coimbatore, India, Dec~14th,~2005}.  With Anurag
Mendhekar.




\section{\sc Patents}

Cristina Lopes, Gregor Kiczales, John Lamping, Erik Hilsdale,
Venkatesh Choppella, and Taher Haveliwala.
\href{http://www.patentstorm.us/patents/6473895.html}
{Aspect-Oriented System Monitoring and Tracing}.  {\em {U.S.
    Patent No. 09/357,508.}} Filed 1999. Awarded April 2002.

% \newpage
\section{\sc Awards}

\href{http://www.manthanaward.com/winner2005.asp}{All India
Manthan Awards} Second Prize in the E-Governance category
for coordinating the Trivandrum City Police Portal for
Community Interaction \url{www.tvmcitypolice.org}, New
Delhi, India,~2005.

%% \section{\sc Awards Contd.}

Best Paper Award. International Parallel and Distributed
Processing Symposium (IPDPS), Albuquerque, New Mexico,
USA,~2004.

% \newpage

Best Paper Award  (Systems). Internationl Conference on High
Performance Computing (HiPC), Hyderabad, India,~2003.

DuPont Fellowship, Computer Science Dept., Indiana
University,~1988.

Merit Certificate for performance in the Joint Entrance
Examination, IIT Kanpur,~1981.

%\newpage

\section{\sc Grants}

European Commission.  Erasmus Mundus External Cooperation
Window. European Research and Educational Collaboration with
Asia Project for Academic Mobility 2008-2009.
\url{http://www.mrtc.mdh.se/eureca/}.  Institute
Coordinator.  10,000 Euro.

Govt. of Kerala, Dept. of Information Technology.  Centre
for Computational Biology.  January 2009.  Rs. 30 Lakhs.
Co-investigator.

Govt. of Kerala, Dept. of Information Technology.  Centre
for Computational Chemistry.  January 2009.  Rs. 25 Lakhs.
Co-investigator.

Govt. of Kerala, Dept. of Information Technology.
Innovation Centre for Open Standards in Software Engineering
Education.  January 2009.  Rs. 40 Lakhs.  Co-investigator.

%% IIITM-K internal grant.  Principles of Programming.
%% 2008-2009.  Principal Investigator.  Rs. 500,000 (pending).

Ministry of Information Technology, Government of India.
Information Security Education and Awareness (ISEA)
2006-2011.  Principal Investigator at Participating
Institution IIITM-K.  Rs. 300,000 per annum. 

Government of Kerala.  Police Portal for Community
Interaction 2004-2005.  Rs. 2,900,000.  Co-investigator.



\section{\sc Consulting}

Cognizant Technologies.  Chennai, India.  December
2007-January 2008.

Satyam Computers.  Hyderabad, India.  April 2008.

Servelots Infotech.  Bangalore, India.  Since December 2007

Indian Postal Service.   New Delhi, India.  May 2009

Terumo Penpol Ltd.  Trivandrum, India.  December 2008


\newpage

\section{\sc Professional Service}


Reviewer and Examiner, Design and Development of Robot
Design Platform and Database Research Platform and Database
based Robot Software Development Framework.  MS Thesis by
Subhash S. IIIT Hyderbad.  May 2010

Programme committee member, 11th International Conference on
Distributed Computing and Internet Technologies (ICDCIT)
February, 2011, Bhubaneswar, India.

Reviewer,  Formal Methods in System Design.  Springer, 2010.

Programme committee member, India Software Engineering
Conference (ISEC) February, 2011, Thiruvananthapuram, India.

Programme committee member, International Conference on
Distributed Computing and Internet Technologies (ICDCIT)
February, 2010, Bhubaneswar, India.

Programme committee member and reviewer, IEEE International
Conference on Software Engineering Education and Training
(CSEET) Feb 17-19, 2009, Hyderabad, India.

% Heap Reference Analysis by Amay Karkare

External reviewer for PhD Thesis, Computer Science, Indian
Institute of Technology, Bombay, Jan 2009.

Convener, Institute Information Technology and Services
Committee, IIITM-K.  Mar 2008 - Jan 2009.


Member, committee appointed by Kerala State IT Mission,
Govt.~of Kerala, to implement Public Key Infrastructure in
state government offices, 2008.


Reviewer, Journal of Parallel and Distributed Computing.
Elsevier. Aug 2008.

Reviewer, International Journal of Formal Computing.
Springer. Aug 2008.

Organizer, Free Map Workshop, IIITM-K, Feb 2008. 

Organizer, One day workshop on Pantoto: Building Information
Management Systems for Small and Medium Enterprises. 
IIITM-K, Feb 2008.

Programme and Organizing Committee member, IEEE workshop on
Connecting Computer Science to Domains, IIITM-K.  Dec~2007.

Convener, Institute Academic Affairs Committee, IIITM-K.
Aug 2006 - July 2007.

Moderator, Panel on Computer Science Teaching, National
workshop on Technology Enhanced Learning, Trivandrum,
Kerala, India.  Jan 5th-6th, 2007.

Panelist, Scilab in Science and Engineering Education.
Indo-French Workshop on Scilab, IIT Bombay.  Mar~8th,~2007.

\newpage

\section{\sc Professional Service contd.}

Organizer, IEEE/CSI Workshop on Community Wireless
Networking, Trivandrum, Kerala, India, Dec 1st-2nd, 2006.

Programme Committee member, International Conference on
Advanced Computing and Communications (ADCOMM~05),
Coimbatore, India, Dec 14-17, 2005.

Programme Committee member, Free Software, Free Society: Three
nation conference on Free, Libre and Open Source Software,
Trivandrum India, May 2005.

Board Member, SPACE (Society for Promotion of Alternative
Computing and Employment), NGO for promoting Free and Open
Source Software in Kerala. Since 2004.

Reviewer, Fifth International Conference on Parallel and
Distributed Computing, Applications and Technologies
(PDCAT~04), 2004.

Acknowledged by the authors of the following books for
proof reading, comments  and technical assistance:\\
\begin{list2}
 \item G. Springer and D. P. Friedman.  {Scheme and the Art of
 Programming}.  {\em {MIT Press,~1989}}.
%%  \vspace{0.5em}
 \item D.P. Friedman, M. Wand and C.T. Haynes.  Essentials of
 Programming Languages, 1st Edition  {\em {MIT Press,~1991}}.
%%  \vspace{0.5em}
 \item Raymond Smullyan. G\"{o}del's Incompleteness Theorems.
 {\em {Oxford University Press,~1992}}.
%%  \vspace{0.5em}
 \item Jon Barwise and Lawrence S. Moss. Vicious Circles
 {\em {CSLI Stanford,~1997}}.
\end{list2}

% \section{\sc Programming Experience} 
% \begin{list2}

% \item Languages: Fortran, Lisp, C, Scheme, ML, Haskell,
%   Prolog, Java, Python, Javascript

% \item Internet Technologies:  J2EE, Spring, Ruby on Rails, Web 2.0

% \item Theorem Provers:  PVS

% \item Operating Systems:  Unix/Linux, Windows
% \end{list2}

\section{\sc Professional Membership}

Computer Society of India

Indian Association for Research in Computer Science

\section{\sc Citizenship}
Citizen of India

\section{\sc Date of Birth}
Available upon request.

\section{\sc Languages}
English, Hindi, Telugu

\section{\sc References}
Available upon request.



\end{resume}
\end{document}




